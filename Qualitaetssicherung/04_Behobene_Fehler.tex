\section{Behobene Fehler}

\subsection{Zuvor gespeicherte Ergebnisse nicht sichtbar}

\textbf{Ursache} ShowResultsHandler versucht das Ergebnis aus der Simulation zu parsen und nicht das gespeicherte Ergebnis zu öffnen\\
\textbf{Lösung} ShowResultView lädt VMResult aus VMConfiguration-Objekten

\subsection{Ausnahme bei Simulation von Mehrfachkonfiguration}

\textbf{Ursache} \emph{NullPointerException} beim Parsen von Mehrfachkonfigurationen\\
\textbf{Lösung} Anpassen des ModelParser, sodass die Ausnahme vermieden wird

\subsection{Ausnahme bei negativen Punktzahlen}

\textbf{Ursache} Ergebnisdarstellung erwartet ausschließlich positive Punktzahlen\\
\textbf{Lösung} Darstellung von negativen Punktzahlen ermöglicht

\subsection{Ausnahme bei Bewertung mittels Durchschnittsrang}

\textbf{Ursache} Beim Parsen wird der Parameter \emph{windowsize} nicht übergeben\\
\textbf{Lösung} Anpassen in View und Handlers

\subsection{Ausnahme bei der Angabe von Anteilen}

\textbf{Ursache} InputValidator und ModelParser hat nutzen verschiedene Formate für die Repräsentation von Anteilen\\
\textbf{Lösung} Vereinheitlichung der beiden Formate

\subsection{Verzögerte Darstellung von erstellten Konfigurationen}

\textbf{Ursache} Nach Erstellen einer Konfiguration wird die Anzeige nicht aktualisiert\\
\textbf{Lösung} Aktualisieren der Anzeige nach jedem Erstellen einer Konfiguration

\subsection{Speichern falscher Strategienamen}

\textbf{Ursache} In VMGroup wird unter gewissen Umständen ein falscher Name gesetzt\\
\textbf{Lösung} Richtiges Setzen des Strategienamen erzwungen

\subsection{Fehlerhafte Darstellung von Umlauten}

\textbf{Ursache} Falsches Encodieren des Java-Compilers\\
\textbf{Lösung} Encodieren mittels UTF-8-Standard

\subsection{Fehlerhafter Dateipfad}

\textbf{Ursache} Annahme eines festen Dateipfads\\
\textbf{Lösung} Speichern im Ordner \emph{./saves} relativ zur JAR-Datei

\subsection{Ungünstige Fenstergröße}

\textbf{Ursache} Einige Fenster erlauben das Hinzufügen von variablen Elementen, die nicht vollständig angezeigt werden können\\
\textbf{Lösung} Anpassen der GUI-Forms

