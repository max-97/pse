\section{Behobene Fehler}

\subsection{Zuvor gespeicherte Ergebnisse nicht sichtbar}

\textbf{Ursache} ShowResultsHandler versucht das Ergebnis aus der Simulation zu parsen und nicht das gespeicherte Ergebnis zu öffnen.\\
\textbf{Lösung} ShowResultView lädt VMResult aus VMConfiguration-Objekten.

\subsection{Ausnahme bei Simulation von Mehrfachkonfiguration}

\textbf{Ursache} \emph{NullPointerException} beim Parsen von Mehrfachkonfigurationen.\\
\textbf{Lösung} Anpassen des ModelParser, sodass die Ausnahme vermieden wird.

\subsection{Ausnahme bei negativen Punktzahlen}

\textbf{Ursache} Ergebnisdarstellung erwartet ausschließlich positive Punktzahlen.\\
\textbf{Lösung} Darstellung von negativen Punktzahlen ermöglicht.

\subsection{Ausnahme bei Bewertung mittels Durchschnittsrang}

\textbf{Ursache} Beim Parsen wird der Parameter \emph{windowsize} nicht übergeben.\\
\textbf{Lösung} Anpassungen in View und Handlers.

\subsection{Ausnahme bei der Angabe von Anteilen}

\textbf{Ursache} InputValidator und ModelParser hat nutzen verschiedene Formate für die Repräsentation von Anteilen.\\
\textbf{Lösung} Vereinheitlichung der beiden Formate.

\subsection{Verzögerte Darstellung von erstellten Konfigurationen}

\textbf{Ursache} Nach Erstellen einer Konfiguration über \emph{Neu $>$ Konfiguration} wird die Konfiguration nicht im Hauptfenster angezeigt.\\
\textbf{Lösung} Aktualisieren des Hauptfenster nach jedem Erstellen einer Konfiguration.

\subsection{Speichern falscher Strategienamen}

\textbf{Ursache} In VMGroup wird unter gewissen Umständen ein falscher Name gesetzt.\\
\textbf{Lösung} Richtiges Setzen des Strategienamen erzwungen.

\subsection{Fehlerhafte Darstellung von Umlauten}

\textbf{Ursache} Falsches Encodieren des Java-Compilers\\
\textbf{Lösung} Encodieren mittels UTF-8-Standard

\subsection{Speichern der Dateien in der Jar-Datei}
Ursprünglich war es geplant die Json-Dateien innerhalb der Jar-Datei zu speichern.\\
\textbf{Ursache} Dateien und Verzeichnisse innerhalb einer Jar-Datei können nicht als File Objekt dargestellt werden.\\
\textbf{Lösung} Speichern im Ordner \emph{./saves} relativ zur JAR-Datei. Das Speichern der Dateien innerhalb der Jar-Datei ist nicht möglich. Beim Starten des Programms muss das Speicherverzeichnis nach allen vorhandenen Dateien durchsucht werden. Dies ist mit der Methode \emph{Files.listFiles()} möglich. Verzeichnisse in der Jar-Datei lassen sich aber nicht als File Objekt darstellen und es gibt keine andere Möglichkeit sämtliche Dateien in einer Jar aufgelistet zu bekommen.

\subsection{Ungünstige Fenstergröße}

\textbf{Ursache} Einige Fenster erlauben das Hinzufügen von variablen Elementen, die nicht vollständig angezeigt werden.\\
\textbf{Lösung} Anpassen der GUI-Größe nachdem neue GUI-Komponenten hinzugefügt wurden.

