%TODO vor Fertigstellung des Entwurfsdokuments geschrieben; nicht Umgesetztes/Fehlendes ggf. streichen/ändern/ergänzen
%TODO Einbindung in Hauptdokument
\section{Einleitung}

Sswis ist ein forschungsorientiertes Softwareprodukt, mit dem wiederholte Spiele ("`repeated games"') als Teilgebiet der Spieltheorie näher untersucht werden können.

Im Prozess, dieses Projekt zu realisieren, wurden durch das Pflichtenheft als Artefakt der Planungsphase die geforderten sowie die gewünschten Anforderungen an das Softwareprodukt detailliert. Ziel dieses Entwurfsdokuments ist es, hieran anzuschließen und die Ergebnisse der Entwurfsphase offenzulegen. Das bedeutet, dem Leser werden die groben und feinen Zusammenhänge des Programms verdeutlicht, beispielhafte Objektinteraktionen präsentiert und gewisse Aspekte der Programmlogik in Form von konkreten Algorithmen erläutert.

Im Entwurf galt als eines der obersten Design-Kriterien weiterhin die Modularität des Softwareprodukts. Neu ist im Vergleich zum Pflichtenheft ein Fokus auch auf einen Entwurf, der eine einfache Parallelisierung zulässt.\\
Diesen beiden Anforderungen wird im Grobentwurf resp. Feinentwurf Rechnung getragen, in denen auf die Paket- bzw. Klassenstruktur eingegangen wird. An entsprechender Stelle wird die Umsetzung hervorgehoben.

Durch Sequenzdiagramme wird ein beispielhafter Simulationsablauf beschrieben.

In den Ausführungen zur Programmlogik wurde den Paarungs- und Bewertungsalgorithmen besondere Aufmerksamkeit gewidmet.

Schließlich findet sich im Anhang ein UML-Diagramm des gesamten Programms.


Dieses Dokument wurde verfasst, um als Grundlage für die Implementierung von Sswis zu dienen, sodass Entscheidungen, die die grundlegende Struktur betreffen, in dieser Phase nicht mehr getroffen werden müssen.