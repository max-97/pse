\chapter*{Class Hierarchy}{
\thispagestyle{empty}
\markboth{Class Hierarchy}{Class Hierarchy}
\addcontentsline{toc}{chapter}{Class Hierarchy}
\section*{Classes}
{\raggedright
\hspace{0.0cm} $\bullet$ java.lang.Object {\tiny \refdefined{java.lang.Object}} \\
\hspace{1.0cm} $\bullet$ de.sswis.controller.FileManager {\tiny \refdefined{de.sswis.controller.FileManager}} \\
\hspace{1.0cm} $\bullet$ de.sswis.controller.ModelProvider {\tiny \refdefined{de.sswis.controller.ModelProvider}} \\
\hspace{1.0cm} $\bullet$ de.sswis.controller.SwingGuiFactory {\tiny \refdefined{de.sswis.controller.SwingGuiFactory}} \\
\hspace{1.0cm} $\bullet$ de.sswis.controller.handlers.CancleHandler {\tiny \refdefined{de.sswis.controller.handlers.CancleHandler}} \\
\hspace{1.0cm} $\bullet$ de.sswis.controller.handlers.CompareResultsHandler {\tiny \refdefined{de.sswis.controller.handlers.CompareResultsHandler}} \\
\hspace{1.0cm} $\bullet$ de.sswis.controller.handlers.DeleteConfigurationHandler {\tiny \refdefined{de.sswis.controller.handlers.DeleteConfigurationHandler}} \\
\hspace{1.0cm} $\bullet$ de.sswis.controller.handlers.DeleteGameHandler {\tiny \refdefined{de.sswis.controller.handlers.DeleteGameHandler}} \\
\hspace{1.0cm} $\bullet$ de.sswis.controller.handlers.DeleteInitializationHandler {\tiny \refdefined{de.sswis.controller.handlers.DeleteInitializationHandler}} \\
\hspace{1.0cm} $\bullet$ de.sswis.controller.handlers.DeleteResultHandler {\tiny \refdefined{de.sswis.controller.handlers.DeleteResultHandler}} \\
\hspace{1.0cm} $\bullet$ de.sswis.controller.handlers.DeleteStrategyHandler {\tiny \refdefined{de.sswis.controller.handlers.DeleteStrategyHandler}} \\
\hspace{1.0cm} $\bullet$ de.sswis.controller.handlers.EditConfigurationHandler {\tiny \refdefined{de.sswis.controller.handlers.EditConfigurationHandler}} \\
\hspace{1.0cm} $\bullet$ de.sswis.controller.handlers.EditGameHandler {\tiny \refdefined{de.sswis.controller.handlers.EditGameHandler}} \\
\hspace{1.0cm} $\bullet$ de.sswis.controller.handlers.EditInitializationHandler {\tiny \refdefined{de.sswis.controller.handlers.EditInitializationHandler}} \\
\hspace{1.0cm} $\bullet$ de.sswis.controller.handlers.EditStrategyHandler {\tiny \refdefined{de.sswis.controller.handlers.EditStrategyHandler}} \\
\hspace{1.0cm} $\bullet$ de.sswis.controller.handlers.ManageConfigurationsHandler {\tiny \refdefined{de.sswis.controller.handlers.ManageConfigurationsHandler}} \\
\hspace{1.0cm} $\bullet$ de.sswis.controller.handlers.ManageGamesHandler {\tiny \refdefined{de.sswis.controller.handlers.ManageGamesHandler}} \\
\hspace{1.0cm} $\bullet$ de.sswis.controller.handlers.ManageInitializationHandler {\tiny \refdefined{de.sswis.controller.handlers.ManageInitializationHandler}} \\
\hspace{1.0cm} $\bullet$ de.sswis.controller.handlers.ManageResultsHandler {\tiny \refdefined{de.sswis.controller.handlers.ManageResultsHandler}} \\
\hspace{1.0cm} $\bullet$ de.sswis.controller.handlers.ManageStrategiesHandler {\tiny \refdefined{de.sswis.controller.handlers.ManageStrategiesHandler}} \\
\hspace{1.0cm} $\bullet$ de.sswis.controller.handlers.NewConfigurationHandler {\tiny \refdefined{de.sswis.controller.handlers.NewConfigurationHandler}} \\
\hspace{1.0cm} $\bullet$ de.sswis.controller.handlers.NewConfigurationViewHandler {\tiny \refdefined{de.sswis.controller.handlers.NewConfigurationViewHandler}} \\
\hspace{1.0cm} $\bullet$ de.sswis.controller.handlers.NewGameHandler {\tiny \refdefined{de.sswis.controller.handlers.NewGameHandler}} \\
\hspace{1.0cm} $\bullet$ de.sswis.controller.handlers.NewGameViewHandler {\tiny \refdefined{de.sswis.controller.handlers.NewGameViewHandler}} \\
\hspace{1.0cm} $\bullet$ de.sswis.controller.handlers.NewInitializationHandler {\tiny \refdefined{de.sswis.controller.handlers.NewInitializationHandler}} \\
\hspace{1.0cm} $\bullet$ de.sswis.controller.handlers.NewInitializationViewHandler {\tiny \refdefined{de.sswis.controller.handlers.NewInitializationViewHandler}} \\
\hspace{1.0cm} $\bullet$ de.sswis.controller.handlers.NewStrategyHandler {\tiny \refdefined{de.sswis.controller.handlers.NewStrategyHandler}} \\
\hspace{1.0cm} $\bullet$ de.sswis.controller.handlers.NewStrategyViewHandler {\tiny \refdefined{de.sswis.controller.handlers.NewStrategyViewHandler}} \\
\hspace{1.0cm} $\bullet$ de.sswis.controller.handlers.SaveAndQuitHandler {\tiny \refdefined{de.sswis.controller.handlers.SaveAndQuitHandler}} \\
\hspace{1.0cm} $\bullet$ de.sswis.controller.handlers.SaveConfigurationsHandler {\tiny \refdefined{de.sswis.controller.handlers.SaveConfigurationsHandler}} \\
\hspace{1.0cm} $\bullet$ de.sswis.controller.handlers.SaveGamesHandler {\tiny \refdefined{de.sswis.controller.handlers.SaveGamesHandler}} \\
\hspace{1.0cm} $\bullet$ de.sswis.controller.handlers.SaveInitializationsHandler {\tiny \refdefined{de.sswis.controller.handlers.SaveInitializationsHandler}} \\
\hspace{1.0cm} $\bullet$ de.sswis.controller.handlers.SaveResultsHandler {\tiny \refdefined{de.sswis.controller.handlers.SaveResultsHandler}} \\
\hspace{1.0cm} $\bullet$ de.sswis.controller.handlers.SaveStrategiesHandler {\tiny \refdefined{de.sswis.controller.handlers.SaveStrategiesHandler}} \\
\hspace{1.0cm} $\bullet$ de.sswis.controller.handlers.ShowResultsHandler {\tiny \refdefined{de.sswis.controller.handlers.ShowResultsHandler}} \\
\hspace{1.0cm} $\bullet$ de.sswis.controller.handlers.StartSimulationHandler {\tiny \refdefined{de.sswis.controller.handlers.StartSimulationHandler}} \\
\hspace{1.0cm} $\bullet$ de.sswis.model.Agent {\tiny \refdefined{de.sswis.model.Agent}} \\
\hspace{1.0cm} $\bullet$ de.sswis.model.CombinedStrategy {\tiny \refdefined{de.sswis.model.CombinedStrategy}} \\
\hspace{1.0cm} $\bullet$ de.sswis.model.Configuration {\tiny \refdefined{de.sswis.model.Configuration}} \\
\hspace{1.0cm} $\bullet$ de.sswis.model.Game {\tiny \refdefined{de.sswis.model.Game}} \\
\hspace{1.0cm} $\bullet$ de.sswis.model.Game.Tuple {\tiny \refdefined{de.sswis.model.Game.Tuple}} \\
\hspace{1.0cm} $\bullet$ de.sswis.model.Group {\tiny \refdefined{de.sswis.model.Group}} \\
\hspace{1.0cm} $\bullet$ de.sswis.model.History {\tiny \refdefined{de.sswis.model.History}} \\
\hspace{1.0cm} $\bullet$ de.sswis.model.Initialization {\tiny \refdefined{de.sswis.model.Initialization}} \\
\hspace{1.0cm} $\bullet$ de.sswis.model.Pair {\tiny \refdefined{de.sswis.model.Pair}} \\
\hspace{1.0cm} $\bullet$ de.sswis.model.Strategy {\tiny \refdefined{de.sswis.model.Strategy}} \\
\hspace{1.0cm} $\bullet$ de.sswis.model.algorithms.adaptation.MixedLinearInterpolation {\tiny \refdefined{de.sswis.model.algorithms.adaptation.MixedLinearInterpolation}} \\
\hspace{1.0cm} $\bullet$ de.sswis.model.algorithms.adaptation.MixedSum {\tiny \refdefined{de.sswis.model.algorithms.adaptation.MixedSum}} \\
\hspace{1.0cm} $\bullet$ de.sswis.model.algorithms.adaptation.RandomAdaptation {\tiny \refdefined{de.sswis.model.algorithms.adaptation.RandomAdaptation}} \\
\hspace{1.0cm} $\bullet$ de.sswis.model.algorithms.adaptation.RankPercentage {\tiny \refdefined{de.sswis.model.algorithms.adaptation.RankPercentage}} \\
\hspace{1.0cm} $\bullet$ de.sswis.model.algorithms.adaptation.ReplicatorDynamicRank {\tiny \refdefined{de.sswis.model.algorithms.adaptation.ReplicatorDynamicRank}} \\
\hspace{1.0cm} $\bullet$ de.sswis.model.algorithms.adaptation.ReplicatorDynamicScore {\tiny \refdefined{de.sswis.model.algorithms.adaptation.ReplicatorDynamicScore}} \\
\hspace{1.0cm} $\bullet$ de.sswis.model.algorithms.ranking.AverageRank {\tiny \refdefined{de.sswis.model.algorithms.ranking.AverageRank}} \\
\hspace{1.0cm} $\bullet$ de.sswis.model.algorithms.ranking.CurrentCycleScore {\tiny \refdefined{de.sswis.model.algorithms.ranking.CurrentCycleScore}} \\
\hspace{1.0cm} $\bullet$ de.sswis.model.algorithms.ranking.CustomCycleScore {\tiny \refdefined{de.sswis.model.algorithms.ranking.CustomCycleScore}} \\
\hspace{1.0cm} $\bullet$ de.sswis.model.algorithms.ranking.Score {\tiny \refdefined{de.sswis.model.algorithms.ranking.Score}} \\
\hspace{1.0cm} $\bullet$ de.sswis.model.conditions.Condition {\tiny \refdefined{de.sswis.model.conditions.Condition}} \\
\hspace{2.0cm} $\bullet$ de.sswis.model.conditions.Always {\tiny \refdefined{de.sswis.model.conditions.Always}} \\
\hspace{2.0cm} $\bullet$ de.sswis.model.conditions.Delta {\tiny \refdefined{de.sswis.model.conditions.Delta}} \\
\hspace{2.0cm} $\bullet$ de.sswis.model.conditions.OwnGroup {\tiny \refdefined{de.sswis.model.conditions.OwnGroup}} \\
\hspace{2.0cm} $\bullet$ de.sswis.model.conditions.Poorer {\tiny \refdefined{de.sswis.model.conditions.Poorer}} \\
\hspace{2.0cm} $\bullet$ de.sswis.model.conditions.Probability {\tiny \refdefined{de.sswis.model.conditions.Probability}} \\
\hspace{2.0cm} $\bullet$ de.sswis.model.conditions.Richer {\tiny \refdefined{de.sswis.model.conditions.Richer}} \\
\hspace{2.0cm} $\bullet$ de.sswis.model.conditions.SpecificGroup {\tiny \refdefined{de.sswis.model.conditions.SpecificGroup}} \\
\hspace{1.0cm} $\bullet$ de.sswis.model.strategies.BaseStrategy {\tiny \refdefined{de.sswis.model.strategies.BaseStrategy}} \\
\hspace{2.0cm} $\bullet$ de.sswis.model.strategies.AlwaysCooperate {\tiny \refdefined{de.sswis.model.strategies.AlwaysCooperate}} \\
\hspace{2.0cm} $\bullet$ de.sswis.model.strategies.Grim1 {\tiny \refdefined{de.sswis.model.strategies.Grim1}} \\
\hspace{2.0cm} $\bullet$ de.sswis.model.strategies.Grim2 {\tiny \refdefined{de.sswis.model.strategies.Grim2}} \\
\hspace{2.0cm} $\bullet$ de.sswis.model.strategies.GroupGrim {\tiny \refdefined{de.sswis.model.strategies.GroupGrim}} \\
\hspace{2.0cm} $\bullet$ de.sswis.model.strategies.GroupTitForTat {\tiny \refdefined{de.sswis.model.strategies.GroupTitForTat}} \\
\hspace{2.0cm} $\bullet$ de.sswis.model.strategies.NeverCooperate {\tiny \refdefined{de.sswis.model.strategies.NeverCooperate}} \\
\hspace{2.0cm} $\bullet$ de.sswis.model.strategies.Random {\tiny \refdefined{de.sswis.model.strategies.Random}} \\
\hspace{2.0cm} $\bullet$ de.sswis.model.strategies.TitForTat1 {\tiny \refdefined{de.sswis.model.strategies.TitForTat1}} \\
\hspace{2.0cm} $\bullet$ de.sswis.model.strategies.TitForTat2 {\tiny \refdefined{de.sswis.model.strategies.TitForTat2}} \\
\hspace{1.0cm} $\bullet$ de.sswis.view.MainView {\tiny \refdefined{de.sswis.view.MainView}} \\
\hspace{1.0cm} $\bullet$ de.sswis.view.ManageGamesView {\tiny \refdefined{de.sswis.view.ManageGamesView}} \\
\hspace{1.0cm} $\bullet$ de.sswis.view.ManageStrategiesView {\tiny \refdefined{de.sswis.view.ManageStrategiesView}} \\
\hspace{1.0cm} $\bullet$ de.sswis.view.NewConfigurationView {\tiny \refdefined{de.sswis.view.NewConfigurationView}} \\
\hspace{1.0cm} $\bullet$ de.sswis.view.NewInitializationView {\tiny \refdefined{de.sswis.view.NewInitializationView}} \\
\hspace{1.0cm} $\bullet$ de.sswis.view.ShowCompareView {\tiny \refdefined{de.sswis.view.ShowCompareView}} \\
\hspace{1.0cm} $\bullet$ de.sswis.view.ShowMultiResultView {\tiny \refdefined{de.sswis.view.ShowMultiResultView}} \\
\hspace{1.0cm} $\bullet$ de.sswis.view.ShowResultView {\tiny \refdefined{de.sswis.view.ShowResultView}} \\
\hspace{1.0cm} $\bullet$ de.sswis.view.model.CombinedStrategy {\tiny \refdefined{de.sswis.view.model.CombinedStrategy}} \\
\hspace{1.0cm} $\bullet$ de.sswis.view.model.Configuration {\tiny \refdefined{de.sswis.view.model.Configuration}} \\
\hspace{1.0cm} $\bullet$ de.sswis.view.model.Game {\tiny \refdefined{de.sswis.view.model.Game}} \\
\hspace{1.0cm} $\bullet$ de.sswis.view.model.Group {\tiny \refdefined{de.sswis.view.model.Group}} \\
\hspace{1.0cm} $\bullet$ de.sswis.view.model.Initialization {\tiny \refdefined{de.sswis.view.model.Initialization}} \\
\hspace{1.0cm} $\bullet$ de.sswis.view.model.Result {\tiny \refdefined{de.sswis.view.model.Result}} \\
\hspace{1.0cm} $\bullet$ de.sswis.view.model.Strategy {\tiny \refdefined{de.sswis.view.model.Strategy}} \\
\hspace{1.0cm} $\bullet$ java.lang.Enum {\tiny \refdefined{java.lang.Enum}} \\
\hspace{2.0cm} $\bullet$ de.sswis.model.Action {\tiny \refdefined{de.sswis.model.Action}} \\
\hspace{1.0cm} $\bullet$ java.util.Observable {\tiny \refdefined{java.util.Observable}} \\
\hspace{2.0cm} $\bullet$ de.sswis.model.Simulation {\tiny \refdefined{de.sswis.model.Simulation}} \\
}
\section*{Interfaces}
\hspace{0.0cm} $\bullet$ de.sswis.controller.AbstractGuiFactory {\tiny \refdefined{de.sswis.controller.AbstractGuiFactory}} \\
\hspace{0.0cm} $\bullet$ de.sswis.model.algorithms.adaptation.AdaptationAlgorithm {\tiny \refdefined{de.sswis.model.algorithms.adaptation.AdaptationAlgorithm}} \\
\hspace{0.0cm} $\bullet$ de.sswis.model.algorithms.pairing.PairingAlgorithm {\tiny \refdefined{de.sswis.model.algorithms.pairing.PairingAlgorithm}} \\
\hspace{0.0cm} $\bullet$ de.sswis.model.algorithms.ranking.RankingAlgorithm {\tiny \refdefined{de.sswis.model.algorithms.ranking.RankingAlgorithm}} \\
\hspace{0.0cm} $\bullet$ de.sswis.view.AbstractMainView {\tiny \refdefined{de.sswis.view.AbstractMainView}} \\
\hspace{0.0cm} $\bullet$ de.sswis.view.AbstractManageCominedStartegiesView {\tiny \refdefined{de.sswis.view.AbstractManageCominedStartegiesView}} \\
\hspace{0.0cm} $\bullet$ de.sswis.view.AbstractManageConfigurationsView {\tiny \refdefined{de.sswis.view.AbstractManageConfigurationsView}} \\
\hspace{0.0cm} $\bullet$ de.sswis.view.AbstractManageGamesView {\tiny \refdefined{de.sswis.view.AbstractManageGamesView}} \\
\hspace{0.0cm} $\bullet$ de.sswis.view.AbstractManageInitializationsView {\tiny \refdefined{de.sswis.view.AbstractManageInitializationsView}} \\
\hspace{0.0cm} $\bullet$ de.sswis.view.AbstractManageResultsView {\tiny \refdefined{de.sswis.view.AbstractManageResultsView}} \\
\hspace{0.0cm} $\bullet$ de.sswis.view.AbstractManageStrategiesView {\tiny \refdefined{de.sswis.view.AbstractManageStrategiesView}} \\
\hspace{0.0cm} $\bullet$ de.sswis.view.AbstractNewCombinedStrategyView {\tiny \refdefined{de.sswis.view.AbstractNewCombinedStrategyView}} \\
\hspace{0.0cm} $\bullet$ de.sswis.view.AbstractNewConfigurationView {\tiny \refdefined{de.sswis.view.AbstractNewConfigurationView}} \\
\hspace{0.0cm} $\bullet$ de.sswis.view.AbstractNewGameView {\tiny \refdefined{de.sswis.view.AbstractNewGameView}} \\
\hspace{0.0cm} $\bullet$ de.sswis.view.AbstractNewInitializationView {\tiny \refdefined{de.sswis.view.AbstractNewInitializationView}} \\
\hspace{0.0cm} $\bullet$ de.sswis.view.AbstractNewStrategyView {\tiny \refdefined{de.sswis.view.AbstractNewStrategyView}} \\
\hspace{0.0cm} $\bullet$ de.sswis.view.AbstractShowCompareView {\tiny \refdefined{de.sswis.view.AbstractShowCompareView}} \\
\hspace{0.0cm} $\bullet$ de.sswis.view.AbstractShowMultiResultView {\tiny \refdefined{de.sswis.view.AbstractShowMultiResultView}} \\
\hspace{0.0cm} $\bullet$ de.sswis.view.AbstractShowResultView {\tiny \refdefined{de.sswis.view.AbstractShowResultView}} \\
}
\chapter{Package de.sswis.controller}{
\label{de.sswis.controller}\hypertarget{de.sswis.controller}{}
\hskip -.05in
\hbox to \hsize{\textit{ Package Contents\hfil Page}}
\vskip .13in
\hbox{{\bf  Interfaces}}
\entityintro{AbstractGuiFactory}{de.sswis.controller.AbstractGuiFactory}{Eine Fabrik zum Erzeugen von GUIs.}
\vskip .13in
\hbox{{\bf  Classes}}
\entityintro{FileManager}{de.sswis.controller.FileManager}{}
\entityintro{ModelProvider}{de.sswis.controller.ModelProvider}{}
\entityintro{SwingGuiFactory}{de.sswis.controller.SwingGuiFactory}{Eine Fabrik zum Erzeugen von GUIs mit \texttt{\small Swing}.}
\vskip .1in
\vskip .1in
\section{\label{de.sswis.controller.AbstractGuiFactory}Interface AbstractGuiFactory}{
\hypertarget{de.sswis.controller.AbstractGuiFactory}{}\vskip .1in 
Eine Fabrik zum Erzeugen von GUIs. Der Nutzer dieser Schnittstelle kann sich alle Benutzeroberflächen erzeugen lassen, die in Sswis verwendet werden können. Die Methoden liefern alle eine Schnittstelle zurück, welches das zu erzeugende Fenster beschreibt. In den Methoden werden die Instanzen der jeweiligen Klassen mit den entsprechenden Schnittstellen erzeugt und die benötigten \texttt{\small ActionListener} gesetzt. Bestimmte Implementierungen von Benutzeroberflächen können weitere oder andere Parameter benötigen.\mbox{}\newline Das \texttt{\small AbstractGuiFactory} Interface enthält eine Methode, um das Hauptfenster zu erzeugen.\mbox{}\newline Das \texttt{\small AbstractGuiFactory} Interface enthält eine Methode, um das Ergebnisansichtsfenster zu erzeugen.\mbox{}\newline Das \texttt{\small AbstractGuiFactory} Interface enthält fünf Methoden, um die Fenster zum Verwalten von Konfigurationen, Initialisierungen, Strategien, Spielen und Ergebnissen zu erzeugen.\mbox{}\newline Das \texttt{\small AbstractGuiFactory} Interface enthält vier Methoden, um die Fenster zum Erstellen von neuen Konfigurationen, Initialisierungen, Strategien und Spielen zu erzeugen.\vskip .1in 
\subsection{Declaration}{
\begin{lstlisting}[frame=none]
public interface AbstractGuiFactory
\end{lstlisting}
\subsection{All known subinterfaces}{SwingGuiFactory\small{\refdefined{de.sswis.controller.SwingGuiFactory}}}
\subsection{All classes known to implement interface}{SwingGuiFactory\small{\refdefined{de.sswis.controller.SwingGuiFactory}}}
\subsection{Methods}{
\vskip -2em
\begin{itemize}
\item{ 
\index{createCompareResultsView()}
\hypertarget{de.sswis.controller.AbstractGuiFactory.createCompareResultsView()}{{\bf  createCompareResultsView}\\}
\begin{lstlisting}[frame=none]
de.sswis.view.AbstractShowCompareView createCompareResultsView()\end{lstlisting} %end signature
\begin{itemize}
\item{
{\bf  Description}

Erstellt eine Ergebnisansicht zum Verlgichen von Simulationen.
}
\item{{\bf  Returns} -- 
eine Ergebnisansicht zum Verlgichen von Simulationen 
}%end item
\end{itemize}
}%end item
\item{ 
\index{createMainView()}
\hypertarget{de.sswis.controller.AbstractGuiFactory.createMainView()}{{\bf  createMainView}\\}
\begin{lstlisting}[frame=none]
de.sswis.view.AbstractMainView createMainView()\end{lstlisting} %end signature
\begin{itemize}
\item{
{\bf  Description}

Erstellt ein Hauptfenster.
}
\item{{\bf  Returns} -- 
ein Hauptfenster 
}%end item
\end{itemize}
}%end item
\item{ 
\index{createManageConfigurationsView()}
\hypertarget{de.sswis.controller.AbstractGuiFactory.createManageConfigurationsView()}{{\bf  createManageConfigurationsView}\\}
\begin{lstlisting}[frame=none]
de.sswis.view.AbstractManageConfigurationsView createManageConfigurationsView()\end{lstlisting} %end signature
\begin{itemize}
\item{
{\bf  Description}

Erstellt ein Konfigurationsverwaltungsfenster.
}
\item{{\bf  Returns} -- 
ein Konfigurationsverwaltungsfenster 
}%end item
\end{itemize}
}%end item
\item{ 
\index{createManageGamesView()}
\hypertarget{de.sswis.controller.AbstractGuiFactory.createManageGamesView()}{{\bf  createManageGamesView}\\}
\begin{lstlisting}[frame=none]
de.sswis.view.AbstractManageGamesView createManageGamesView()\end{lstlisting} %end signature
\begin{itemize}
\item{
{\bf  Description}

Erstellt ein Spieleverwaltungsfenster.
}
\item{{\bf  Returns} -- 
ein Spieleverwaltungsfenster 
}%end item
\end{itemize}
}%end item
\item{ 
\index{createManageInitializationsView()}
\hypertarget{de.sswis.controller.AbstractGuiFactory.createManageInitializationsView()}{{\bf  createManageInitializationsView}\\}
\begin{lstlisting}[frame=none]
de.sswis.view.AbstractManageInitializationsView createManageInitializationsView()\end{lstlisting} %end signature
\begin{itemize}
\item{
{\bf  Description}

Erstellt ein Initialisierungsverwaltungsfenster.
}
\item{{\bf  Returns} -- 
ein Initialisierungsverwaltungsfenster 
}%end item
\end{itemize}
}%end item
\item{ 
\index{createManageResultsView()}
\hypertarget{de.sswis.controller.AbstractGuiFactory.createManageResultsView()}{{\bf  createManageResultsView}\\}
\begin{lstlisting}[frame=none]
de.sswis.view.AbstractManageResultsView createManageResultsView()\end{lstlisting} %end signature
\begin{itemize}
\item{
{\bf  Description}

Erstellt ein Ergebnisverwaltungsfenster.
}
\item{{\bf  Returns} -- 
ein Ergebnisverwaltungsfenster 
}%end item
\end{itemize}
}%end item
\item{ 
\index{createManageStrategiesView()}
\hypertarget{de.sswis.controller.AbstractGuiFactory.createManageStrategiesView()}{{\bf  createManageStrategiesView}\\}
\begin{lstlisting}[frame=none]
de.sswis.view.AbstractManageStrategiesView createManageStrategiesView()\end{lstlisting} %end signature
\begin{itemize}
\item{
{\bf  Description}

Erstellt ein Strategieverwaltungsfenster.
}
\item{{\bf  Returns} -- 
ein Strategieverwaltungsfenster 
}%end item
\end{itemize}
}%end item
\item{ 
\index{createMultiResultsView()}
\hypertarget{de.sswis.controller.AbstractGuiFactory.createMultiResultsView()}{{\bf  createMultiResultsView}\\}
\begin{lstlisting}[frame=none]
de.sswis.view.AbstractShowMultiResultView createMultiResultsView()\end{lstlisting} %end signature
\begin{itemize}
\item{
{\bf  Description}

Erstellt eine Ergebnisansicht mit allen Simulationen einer Multikonfiguration.
}
\item{{\bf  Returns} -- 
eine Ergebnisansicht mit allen Simulationen einer Multikonfiguration 
}%end item
\end{itemize}
}%end item
\item{ 
\index{createNewConfigurationView()}
\hypertarget{de.sswis.controller.AbstractGuiFactory.createNewConfigurationView()}{{\bf  createNewConfigurationView}\\}
\begin{lstlisting}[frame=none]
de.sswis.view.AbstractNewConfigurationView createNewConfigurationView()\end{lstlisting} %end signature
\begin{itemize}
\item{
{\bf  Description}

Erstellt ein Fenster zum Erstellen von Konfigurationen.
}
\item{{\bf  Returns} -- 
ein Fenster zum Erstellen von Konfigurationen 
}%end item
\end{itemize}
}%end item
\item{ 
\index{createNewGameView()}
\hypertarget{de.sswis.controller.AbstractGuiFactory.createNewGameView()}{{\bf  createNewGameView}\\}
\begin{lstlisting}[frame=none]
de.sswis.view.AbstractNewGameView createNewGameView()\end{lstlisting} %end signature
\begin{itemize}
\item{
{\bf  Description}

Erstellt ein Fenster zum Erstellen von Spielen.
}
\item{{\bf  Returns} -- 
ein Fenster zum Erstellen von Spielen 
}%end item
\end{itemize}
}%end item
\item{ 
\index{createNewInitializationView()}
\hypertarget{de.sswis.controller.AbstractGuiFactory.createNewInitializationView()}{{\bf  createNewInitializationView}\\}
\begin{lstlisting}[frame=none]
de.sswis.view.AbstractNewInitializationView createNewInitializationView()\end{lstlisting} %end signature
\begin{itemize}
\item{
{\bf  Description}

Erstellt ein Fenster zum Erstellen von Initialisierungen.
}
\item{{\bf  Returns} -- 
ein Fenster zum Erstellen von Initialisierungen 
}%end item
\end{itemize}
}%end item
\item{ 
\index{createNewStrategyView()}
\hypertarget{de.sswis.controller.AbstractGuiFactory.createNewStrategyView()}{{\bf  createNewStrategyView}\\}
\begin{lstlisting}[frame=none]
de.sswis.view.AbstractNewStrategyView createNewStrategyView()\end{lstlisting} %end signature
\begin{itemize}
\item{
{\bf  Description}

Erstellt ein Fenster zum Erstellen von Strategien.
}
\item{{\bf  Returns} -- 
ein Fenster zum Erstellen von Strategien 
}%end item
\end{itemize}
}%end item
\item{ 
\index{createShowResultView()}
\hypertarget{de.sswis.controller.AbstractGuiFactory.createShowResultView()}{{\bf  createShowResultView}\\}
\begin{lstlisting}[frame=none]
de.sswis.view.AbstractShowResultView createShowResultView()\end{lstlisting} %end signature
\begin{itemize}
\item{
{\bf  Description}

Erstellt ein Ergebnisansichtsfenster.
}
\item{{\bf  Returns} -- 
ein Ergebnisansichtsfenster 
}%end item
\end{itemize}
}%end item
\end{itemize}
}
}
\section{\label{de.sswis.controller.FileManager}Class FileManager}{
\hypertarget{de.sswis.controller.FileManager}{}\vskip .1in 
\subsection{Declaration}{
\begin{lstlisting}[frame=none]
public class FileManager
 extends java.lang.Object\end{lstlisting}
\subsection{Constructors}{
\vskip -2em
\begin{itemize}
\item{ 
\index{FileManager()}
\hypertarget{de.sswis.controller.FileManager()}{{\bf  FileManager}\\}
\begin{lstlisting}[frame=none]
public FileManager()\end{lstlisting} %end signature
}%end item
\end{itemize}
}
\subsection{Methods}{
\vskip -2em
\begin{itemize}
\item{ 
\index{deleteCombinedStrategy(String)}
\hypertarget{de.sswis.controller.FileManager.deleteCombinedStrategy(java.lang.String)}{{\bf  deleteCombinedStrategy}\\}
\begin{lstlisting}[frame=none]
public void deleteCombinedStrategy(java.lang.String name)\end{lstlisting} %end signature
}%end item
\item{ 
\index{deleteConfigurations(String)}
\hypertarget{de.sswis.controller.FileManager.deleteConfigurations(java.lang.String)}{{\bf  deleteConfigurations}\\}
\begin{lstlisting}[frame=none]
public void deleteConfigurations(java.lang.String name)\end{lstlisting} %end signature
}%end item
\item{ 
\index{deleteGame(String)}
\hypertarget{de.sswis.controller.FileManager.deleteGame(java.lang.String)}{{\bf  deleteGame}\\}
\begin{lstlisting}[frame=none]
public void deleteGame(java.lang.String name)\end{lstlisting} %end signature
}%end item
\item{ 
\index{deleteInitalization(String)}
\hypertarget{de.sswis.controller.FileManager.deleteInitalization(java.lang.String)}{{\bf  deleteInitalization}\\}
\begin{lstlisting}[frame=none]
public void deleteInitalization(java.lang.String name)\end{lstlisting} %end signature
}%end item
\item{ 
\index{deleteResult(Object)}
\hypertarget{de.sswis.controller.FileManager.deleteResult(java.lang.Object)}{{\bf  deleteResult}\\}
\begin{lstlisting}[frame=none]
public void deleteResult(java.lang.Object obj)\end{lstlisting} %end signature
}%end item
\item{ 
\index{loadCombinedStrategy(String)}
\hypertarget{de.sswis.controller.FileManager.loadCombinedStrategy(java.lang.String)}{{\bf  loadCombinedStrategy}\\}
\begin{lstlisting}[frame=none]
public de.sswis.model.CombinedStrategy loadCombinedStrategy(java.lang.String name)\end{lstlisting} %end signature
}%end item
\item{ 
\index{loadConfigurations(String)}
\hypertarget{de.sswis.controller.FileManager.loadConfigurations(java.lang.String)}{{\bf  loadConfigurations}\\}
\begin{lstlisting}[frame=none]
public java.util.Collection loadConfigurations(java.lang.String name)\end{lstlisting} %end signature
}%end item
\item{ 
\index{loadGame(String)}
\hypertarget{de.sswis.controller.FileManager.loadGame(java.lang.String)}{{\bf  loadGame}\\}
\begin{lstlisting}[frame=none]
public de.sswis.model.Game loadGame(java.lang.String name)\end{lstlisting} %end signature
}%end item
\item{ 
\index{loadInitalization(String)}
\hypertarget{de.sswis.controller.FileManager.loadInitalization(java.lang.String)}{{\bf  loadInitalization}\\}
\begin{lstlisting}[frame=none]
public de.sswis.model.Initialization loadInitalization(java.lang.String name)\end{lstlisting} %end signature
}%end item
\item{ 
\index{loadResult(String)}
\hypertarget{de.sswis.controller.FileManager.loadResult(java.lang.String)}{{\bf  loadResult}\\}
\begin{lstlisting}[frame=none]
public java.lang.Object loadResult(java.lang.String name)\end{lstlisting} %end signature
}%end item
\item{ 
\index{saveCombinedStrategy(CombinedStrategy)}
\hypertarget{de.sswis.controller.FileManager.saveCombinedStrategy(de.sswis.model.CombinedStrategy)}{{\bf  saveCombinedStrategy}\\}
\begin{lstlisting}[frame=none]
public void saveCombinedStrategy(de.sswis.model.CombinedStrategy combinedStrategy)\end{lstlisting} %end signature
}%end item
\item{ 
\index{saveConfigurations(Collection)}
\hypertarget{de.sswis.controller.FileManager.saveConfigurations(java.util.Collection)}{{\bf  saveConfigurations}\\}
\begin{lstlisting}[frame=none]
public void saveConfigurations(java.util.Collection configurations)\end{lstlisting} %end signature
}%end item
\item{ 
\index{saveGame(Game)}
\hypertarget{de.sswis.controller.FileManager.saveGame(de.sswis.model.Game)}{{\bf  saveGame}\\}
\begin{lstlisting}[frame=none]
public void saveGame(de.sswis.model.Game game)\end{lstlisting} %end signature
}%end item
\item{ 
\index{saveInitalization(Initialization)}
\hypertarget{de.sswis.controller.FileManager.saveInitalization(de.sswis.model.Initialization)}{{\bf  saveInitalization}\\}
\begin{lstlisting}[frame=none]
public void saveInitalization(de.sswis.model.Initialization initialization)\end{lstlisting} %end signature
}%end item
\item{ 
\index{saveResult(Object)}
\hypertarget{de.sswis.controller.FileManager.saveResult(java.lang.Object)}{{\bf  saveResult}\\}
\begin{lstlisting}[frame=none]
public void saveResult(java.lang.Object obj)\end{lstlisting} %end signature
}%end item
\end{itemize}
}
}
\section{\label{de.sswis.controller.ModelProvider}Class ModelProvider}{
\hypertarget{de.sswis.controller.ModelProvider}{}\vskip .1in 
\subsection{Declaration}{
\begin{lstlisting}[frame=none]
public class ModelProvider
 extends java.lang.Object\end{lstlisting}
\subsection{Methods}{
\vskip -2em
\begin{itemize}
\item{ 
\index{addAgent(Agent)}
\hypertarget{de.sswis.controller.ModelProvider.addAgent(de.sswis.model.Agent)}{{\bf  addAgent}\\}
\begin{lstlisting}[frame=none]
public void addAgent(de.sswis.model.Agent agent)\end{lstlisting} %end signature
}%end item
\item{ 
\index{addCombinedStrategy(CombinedStrategy)}
\hypertarget{de.sswis.controller.ModelProvider.addCombinedStrategy(de.sswis.model.CombinedStrategy)}{{\bf  addCombinedStrategy}\\}
\begin{lstlisting}[frame=none]
public void addCombinedStrategy(de.sswis.model.CombinedStrategy combStrategy)\end{lstlisting} %end signature
}%end item
\item{ 
\index{addConfigurations(Collection)}
\hypertarget{de.sswis.controller.ModelProvider.addConfigurations(java.util.Collection)}{{\bf  addConfigurations}\\}
\begin{lstlisting}[frame=none]
public void addConfigurations(java.util.Collection configurations)\end{lstlisting} %end signature
}%end item
\item{ 
\index{addGame(Game)}
\hypertarget{de.sswis.controller.ModelProvider.addGame(de.sswis.model.Game)}{{\bf  addGame}\\}
\begin{lstlisting}[frame=none]
public void addGame(de.sswis.model.Game game)\end{lstlisting} %end signature
}%end item
\item{ 
\index{addInitialization(Initialization)}
\hypertarget{de.sswis.controller.ModelProvider.addInitialization(de.sswis.model.Initialization)}{{\bf  addInitialization}\\}
\begin{lstlisting}[frame=none]
public void addInitialization(de.sswis.model.Initialization initialization)\end{lstlisting} %end signature
}%end item
\item{ 
\index{deleteAgents(int)}
\hypertarget{de.sswis.controller.ModelProvider.deleteAgents(int)}{{\bf  deleteAgents}\\}
\begin{lstlisting}[frame=none]
public void deleteAgents(int id)\end{lstlisting} %end signature
}%end item
\item{ 
\index{deleteCombinedStrategy(String)}
\hypertarget{de.sswis.controller.ModelProvider.deleteCombinedStrategy(java.lang.String)}{{\bf  deleteCombinedStrategy}\\}
\begin{lstlisting}[frame=none]
public void deleteCombinedStrategy(java.lang.String name)\end{lstlisting} %end signature
}%end item
\item{ 
\index{deleteConfigurations(String)}
\hypertarget{de.sswis.controller.ModelProvider.deleteConfigurations(java.lang.String)}{{\bf  deleteConfigurations}\\}
\begin{lstlisting}[frame=none]
public void deleteConfigurations(java.lang.String name)\end{lstlisting} %end signature
}%end item
\item{ 
\index{deleteGame(String)}
\hypertarget{de.sswis.controller.ModelProvider.deleteGame(java.lang.String)}{{\bf  deleteGame}\\}
\begin{lstlisting}[frame=none]
public void deleteGame(java.lang.String name)\end{lstlisting} %end signature
}%end item
\item{ 
\index{deleteInitialization(String)}
\hypertarget{de.sswis.controller.ModelProvider.deleteInitialization(java.lang.String)}{{\bf  deleteInitialization}\\}
\begin{lstlisting}[frame=none]
public void deleteInitialization(java.lang.String name)\end{lstlisting} %end signature
}%end item
\item{ 
\index{getAgent(int)}
\hypertarget{de.sswis.controller.ModelProvider.getAgent(int)}{{\bf  getAgent}\\}
\begin{lstlisting}[frame=none]
public de.sswis.model.Agent getAgent(int id)\end{lstlisting} %end signature
}%end item
\item{ 
\index{getAgents()}
\hypertarget{de.sswis.controller.ModelProvider.getAgents()}{{\bf  getAgents}\\}
\begin{lstlisting}[frame=none]
public java.util.Map getAgents()\end{lstlisting} %end signature
}%end item
\item{ 
\index{getCombinedStrategies()}
\hypertarget{de.sswis.controller.ModelProvider.getCombinedStrategies()}{{\bf  getCombinedStrategies}\\}
\begin{lstlisting}[frame=none]
public java.util.Map getCombinedStrategies()\end{lstlisting} %end signature
}%end item
\item{ 
\index{getCombinedStrategy(String)}
\hypertarget{de.sswis.controller.ModelProvider.getCombinedStrategy(java.lang.String)}{{\bf  getCombinedStrategy}\\}
\begin{lstlisting}[frame=none]
public de.sswis.model.CombinedStrategy getCombinedStrategy(java.lang.String name)\end{lstlisting} %end signature
}%end item
\item{ 
\index{getConfiguration(String)}
\hypertarget{de.sswis.controller.ModelProvider.getConfiguration(java.lang.String)}{{\bf  getConfiguration}\\}
\begin{lstlisting}[frame=none]
public de.sswis.model.Configuration getConfiguration(java.lang.String name)\end{lstlisting} %end signature
}%end item
\item{ 
\index{getConfigurations()}
\hypertarget{de.sswis.controller.ModelProvider.getConfigurations()}{{\bf  getConfigurations}\\}
\begin{lstlisting}[frame=none]
public java.util.Map getConfigurations()\end{lstlisting} %end signature
}%end item
\item{ 
\index{getGame(String)}
\hypertarget{de.sswis.controller.ModelProvider.getGame(java.lang.String)}{{\bf  getGame}\\}
\begin{lstlisting}[frame=none]
public de.sswis.model.Game getGame(java.lang.String name)\end{lstlisting} %end signature
}%end item
\item{ 
\index{getGames()}
\hypertarget{de.sswis.controller.ModelProvider.getGames()}{{\bf  getGames}\\}
\begin{lstlisting}[frame=none]
public java.util.Map getGames()\end{lstlisting} %end signature
}%end item
\item{ 
\index{getInitialization(String)}
\hypertarget{de.sswis.controller.ModelProvider.getInitialization(java.lang.String)}{{\bf  getInitialization}\\}
\begin{lstlisting}[frame=none]
public de.sswis.model.Initialization getInitialization(java.lang.String name)\end{lstlisting} %end signature
}%end item
\item{ 
\index{getInitializations()}
\hypertarget{de.sswis.controller.ModelProvider.getInitializations()}{{\bf  getInitializations}\\}
\begin{lstlisting}[frame=none]
public java.util.Map getInitializations()\end{lstlisting} %end signature
}%end item
\item{ 
\index{getInstance()}
\hypertarget{de.sswis.controller.ModelProvider.getInstance()}{{\bf  getInstance}\\}
\begin{lstlisting}[frame=none]
public static ModelProvider getInstance()\end{lstlisting} %end signature
}%end item
\end{itemize}
}
}
\section{\label{de.sswis.controller.SwingGuiFactory}Class SwingGuiFactory}{
\hypertarget{de.sswis.controller.SwingGuiFactory}{}\vskip .1in 
Eine Fabrik zum Erzeugen von GUIs mit \texttt{\small Swing}. Diese Fabrik erzeugt Benutzeroberflächen, welche ausschließlich aus Swing-Elementen bestehen. Die Benutzeroberflächen erhalten \texttt{\small \hyperlink{java.awt.event.ActionListener}{ActionListener}}{\small 
\refdefined{java.awt.event.ActionListener}}, die das Verhalten der GUI-Elemente beschreiben.\vskip .1in 
\subsection{See also}{}

  \begin{list}{-- }{\setlength{\itemsep}{0cm}\setlength{\parsep}{0cm}}
\item{ \texttt{\hyperlink{de.sswis.controller.AbstractGuiFactory}{AbstractGuiFactory}} {\small 
\refdefined{de.sswis.controller.AbstractGuiFactory}}%end
} 
  \end{list}
\subsection{Declaration}{
\begin{lstlisting}[frame=none]
public class SwingGuiFactory
 extends java.lang.Object implements AbstractGuiFactory\end{lstlisting}
\subsection{Constructors}{
\vskip -2em
\begin{itemize}
\item{ 
\index{SwingGuiFactory()}
\hypertarget{de.sswis.controller.SwingGuiFactory()}{{\bf  SwingGuiFactory}\\}
\begin{lstlisting}[frame=none]
public SwingGuiFactory()\end{lstlisting} %end signature
}%end item
\end{itemize}
}
\subsection{Methods}{
\vskip -2em
\begin{itemize}
\item{ 
\index{createCompareResultsView()}
\hypertarget{de.sswis.controller.SwingGuiFactory.createCompareResultsView()}{{\bf  createCompareResultsView}\\}
\begin{lstlisting}[frame=none]
de.sswis.view.AbstractShowCompareView createCompareResultsView()\end{lstlisting} %end signature
\begin{itemize}
\item{
{\bf  Description copied from \hyperlink{de.sswis.controller.AbstractGuiFactory}{AbstractGuiFactory}{\small \refdefined{de.sswis.controller.AbstractGuiFactory}} }

Erstellt eine Ergebnisansicht zum Verlgichen von Simulationen.
}
\item{{\bf  Returns} -- 
eine Ergebnisansicht zum Verlgichen von Simulationen 
}%end item
\end{itemize}
}%end item
\item{ 
\index{createMainView()}
\hypertarget{de.sswis.controller.SwingGuiFactory.createMainView()}{{\bf  createMainView}\\}
\begin{lstlisting}[frame=none]
de.sswis.view.AbstractMainView createMainView()\end{lstlisting} %end signature
\begin{itemize}
\item{
{\bf  Description copied from \hyperlink{de.sswis.controller.AbstractGuiFactory}{AbstractGuiFactory}{\small \refdefined{de.sswis.controller.AbstractGuiFactory}} }

Erstellt ein Hauptfenster.
}
\item{{\bf  Returns} -- 
ein Hauptfenster 
}%end item
\end{itemize}
}%end item
\item{ 
\index{createManageConfigurationsView()}
\hypertarget{de.sswis.controller.SwingGuiFactory.createManageConfigurationsView()}{{\bf  createManageConfigurationsView}\\}
\begin{lstlisting}[frame=none]
de.sswis.view.AbstractManageConfigurationsView createManageConfigurationsView()\end{lstlisting} %end signature
\begin{itemize}
\item{
{\bf  Description copied from \hyperlink{de.sswis.controller.AbstractGuiFactory}{AbstractGuiFactory}{\small \refdefined{de.sswis.controller.AbstractGuiFactory}} }

Erstellt ein Konfigurationsverwaltungsfenster.
}
\item{{\bf  Returns} -- 
ein Konfigurationsverwaltungsfenster 
}%end item
\end{itemize}
}%end item
\item{ 
\index{createManageGamesView()}
\hypertarget{de.sswis.controller.SwingGuiFactory.createManageGamesView()}{{\bf  createManageGamesView}\\}
\begin{lstlisting}[frame=none]
de.sswis.view.AbstractManageGamesView createManageGamesView()\end{lstlisting} %end signature
\begin{itemize}
\item{
{\bf  Description copied from \hyperlink{de.sswis.controller.AbstractGuiFactory}{AbstractGuiFactory}{\small \refdefined{de.sswis.controller.AbstractGuiFactory}} }

Erstellt ein Spieleverwaltungsfenster.
}
\item{{\bf  Returns} -- 
ein Spieleverwaltungsfenster 
}%end item
\end{itemize}
}%end item
\item{ 
\index{createManageInitializationsView()}
\hypertarget{de.sswis.controller.SwingGuiFactory.createManageInitializationsView()}{{\bf  createManageInitializationsView}\\}
\begin{lstlisting}[frame=none]
de.sswis.view.AbstractManageInitializationsView createManageInitializationsView()\end{lstlisting} %end signature
\begin{itemize}
\item{
{\bf  Description copied from \hyperlink{de.sswis.controller.AbstractGuiFactory}{AbstractGuiFactory}{\small \refdefined{de.sswis.controller.AbstractGuiFactory}} }

Erstellt ein Initialisierungsverwaltungsfenster.
}
\item{{\bf  Returns} -- 
ein Initialisierungsverwaltungsfenster 
}%end item
\end{itemize}
}%end item
\item{ 
\index{createManageResultsView()}
\hypertarget{de.sswis.controller.SwingGuiFactory.createManageResultsView()}{{\bf  createManageResultsView}\\}
\begin{lstlisting}[frame=none]
de.sswis.view.AbstractManageResultsView createManageResultsView()\end{lstlisting} %end signature
\begin{itemize}
\item{
{\bf  Description copied from \hyperlink{de.sswis.controller.AbstractGuiFactory}{AbstractGuiFactory}{\small \refdefined{de.sswis.controller.AbstractGuiFactory}} }

Erstellt ein Ergebnisverwaltungsfenster.
}
\item{{\bf  Returns} -- 
ein Ergebnisverwaltungsfenster 
}%end item
\end{itemize}
}%end item
\item{ 
\index{createManageStrategiesView()}
\hypertarget{de.sswis.controller.SwingGuiFactory.createManageStrategiesView()}{{\bf  createManageStrategiesView}\\}
\begin{lstlisting}[frame=none]
de.sswis.view.AbstractManageStrategiesView createManageStrategiesView()\end{lstlisting} %end signature
\begin{itemize}
\item{
{\bf  Description copied from \hyperlink{de.sswis.controller.AbstractGuiFactory}{AbstractGuiFactory}{\small \refdefined{de.sswis.controller.AbstractGuiFactory}} }

Erstellt ein Strategieverwaltungsfenster.
}
\item{{\bf  Returns} -- 
ein Strategieverwaltungsfenster 
}%end item
\end{itemize}
}%end item
\item{ 
\index{createMultiResultsView()}
\hypertarget{de.sswis.controller.SwingGuiFactory.createMultiResultsView()}{{\bf  createMultiResultsView}\\}
\begin{lstlisting}[frame=none]
de.sswis.view.AbstractShowMultiResultView createMultiResultsView()\end{lstlisting} %end signature
\begin{itemize}
\item{
{\bf  Description copied from \hyperlink{de.sswis.controller.AbstractGuiFactory}{AbstractGuiFactory}{\small \refdefined{de.sswis.controller.AbstractGuiFactory}} }

Erstellt eine Ergebnisansicht mit allen Simulationen einer Multikonfiguration.
}
\item{{\bf  Returns} -- 
eine Ergebnisansicht mit allen Simulationen einer Multikonfiguration 
}%end item
\end{itemize}
}%end item
\item{ 
\index{createNewConfigurationView()}
\hypertarget{de.sswis.controller.SwingGuiFactory.createNewConfigurationView()}{{\bf  createNewConfigurationView}\\}
\begin{lstlisting}[frame=none]
de.sswis.view.AbstractNewConfigurationView createNewConfigurationView()\end{lstlisting} %end signature
\begin{itemize}
\item{
{\bf  Description copied from \hyperlink{de.sswis.controller.AbstractGuiFactory}{AbstractGuiFactory}{\small \refdefined{de.sswis.controller.AbstractGuiFactory}} }

Erstellt ein Fenster zum Erstellen von Konfigurationen.
}
\item{{\bf  Returns} -- 
ein Fenster zum Erstellen von Konfigurationen 
}%end item
\end{itemize}
}%end item
\item{ 
\index{createNewGameView()}
\hypertarget{de.sswis.controller.SwingGuiFactory.createNewGameView()}{{\bf  createNewGameView}\\}
\begin{lstlisting}[frame=none]
de.sswis.view.AbstractNewGameView createNewGameView()\end{lstlisting} %end signature
\begin{itemize}
\item{
{\bf  Description copied from \hyperlink{de.sswis.controller.AbstractGuiFactory}{AbstractGuiFactory}{\small \refdefined{de.sswis.controller.AbstractGuiFactory}} }

Erstellt ein Fenster zum Erstellen von Spielen.
}
\item{{\bf  Returns} -- 
ein Fenster zum Erstellen von Spielen 
}%end item
\end{itemize}
}%end item
\item{ 
\index{createNewInitializationView()}
\hypertarget{de.sswis.controller.SwingGuiFactory.createNewInitializationView()}{{\bf  createNewInitializationView}\\}
\begin{lstlisting}[frame=none]
de.sswis.view.AbstractNewInitializationView createNewInitializationView()\end{lstlisting} %end signature
\begin{itemize}
\item{
{\bf  Description copied from \hyperlink{de.sswis.controller.AbstractGuiFactory}{AbstractGuiFactory}{\small \refdefined{de.sswis.controller.AbstractGuiFactory}} }

Erstellt ein Fenster zum Erstellen von Initialisierungen.
}
\item{{\bf  Returns} -- 
ein Fenster zum Erstellen von Initialisierungen 
}%end item
\end{itemize}
}%end item
\item{ 
\index{createNewStrategyView()}
\hypertarget{de.sswis.controller.SwingGuiFactory.createNewStrategyView()}{{\bf  createNewStrategyView}\\}
\begin{lstlisting}[frame=none]
de.sswis.view.AbstractNewStrategyView createNewStrategyView()\end{lstlisting} %end signature
\begin{itemize}
\item{
{\bf  Description copied from \hyperlink{de.sswis.controller.AbstractGuiFactory}{AbstractGuiFactory}{\small \refdefined{de.sswis.controller.AbstractGuiFactory}} }

Erstellt ein Fenster zum Erstellen von Strategien.
}
\item{{\bf  Returns} -- 
ein Fenster zum Erstellen von Strategien 
}%end item
\end{itemize}
}%end item
\item{ 
\index{createShowResultView()}
\hypertarget{de.sswis.controller.SwingGuiFactory.createShowResultView()}{{\bf  createShowResultView}\\}
\begin{lstlisting}[frame=none]
de.sswis.view.AbstractShowResultView createShowResultView()\end{lstlisting} %end signature
\begin{itemize}
\item{
{\bf  Description copied from \hyperlink{de.sswis.controller.AbstractGuiFactory}{AbstractGuiFactory}{\small \refdefined{de.sswis.controller.AbstractGuiFactory}} }

Erstellt ein Ergebnisansichtsfenster.
}
\item{{\bf  Returns} -- 
ein Ergebnisansichtsfenster 
}%end item
\end{itemize}
}%end item
\end{itemize}
}
}
}
\chapter{Package de.sswis.controller.handlers}{
\label{de.sswis.controller.handlers}\hypertarget{de.sswis.controller.handlers}{}
\hskip -.05in
\hbox to \hsize{\textit{ Package Contents\hfil Page}}
\vskip .13in
\hbox{{\bf  Classes}}
\entityintro{CancleHandler}{de.sswis.controller.handlers.CancleHandler}{Ein \texttt{\small Hanlder} zum Schließen einer View.}
\entityintro{CompareResultsHandler}{de.sswis.controller.handlers.CompareResultsHandler}{Öffnet die View zum Vergleichen von Ergebnissen.}
\entityintro{DeleteConfigurationHandler}{de.sswis.controller.handlers.DeleteConfigurationHandler}{Löscht die ausgewählte \texttt{\small Konfiguration}.}
\entityintro{DeleteGameHandler}{de.sswis.controller.handlers.DeleteGameHandler}{Löscht das ausgewählte \texttt{\small Spiel}.}
\entityintro{DeleteInitializationHandler}{de.sswis.controller.handlers.DeleteInitializationHandler}{Löscht die ausgewählte \texttt{\small Initialisierung}.}
\entityintro{DeleteResultHandler}{de.sswis.controller.handlers.DeleteResultHandler}{Löscht das ausgewählte \texttt{\small Ergebnis}.}
\entityintro{DeleteStrategyHandler}{de.sswis.controller.handlers.DeleteStrategyHandler}{Löscht die ausgewählte \texttt{\small Strategie}.}
\entityintro{EditConfigurationHandler}{de.sswis.controller.handlers.EditConfigurationHandler}{Öffnet die View zum Bearbeiten einer \texttt{\small Konfiguration}.}
\entityintro{EditGameHandler}{de.sswis.controller.handlers.EditGameHandler}{Öffnet die View zum Bearbeiten eines \texttt{\small Spiels}.}
\entityintro{EditInitializationHandler}{de.sswis.controller.handlers.EditInitializationHandler}{Öffnet die View zum Bearbeiten einer \texttt{\small Initialisierung}.}
\entityintro{EditStrategyHandler}{de.sswis.controller.handlers.EditStrategyHandler}{Öffnet die View zum Bearbeiten einer \texttt{\small Strategie}.}
\entityintro{ManageConfigurationsHandler}{de.sswis.controller.handlers.ManageConfigurationsHandler}{Öffnet die View zum Verwalten der \texttt{\small Konfigurationen}.}
\entityintro{ManageGamesHandler}{de.sswis.controller.handlers.ManageGamesHandler}{Öffnet die View zum Verwalten der \texttt{\small Spiele}.}
\entityintro{ManageInitializationHandler}{de.sswis.controller.handlers.ManageInitializationHandler}{Öffnet die View zum Verwalten der \texttt{\small Initialisierungen}.}
\entityintro{ManageResultsHandler}{de.sswis.controller.handlers.ManageResultsHandler}{Öffnet die View zum Verwalten der \texttt{\small Ergebnisse}.}
\entityintro{ManageStrategiesHandler}{de.sswis.controller.handlers.ManageStrategiesHandler}{Öffnet die View zum Verwalten der \texttt{\small Strategien}.}
\entityintro{NewConfigurationHandler}{de.sswis.controller.handlers.NewConfigurationHandler}{Erstellt eine neuen Konfiguration.}
\entityintro{NewConfigurationViewHandler}{de.sswis.controller.handlers.NewConfigurationViewHandler}{Öffnet eine View zum Erstellen einer \texttt{\small Konfiguration}.}
\entityintro{NewGameHandler}{de.sswis.controller.handlers.NewGameHandler}{Erstellt ein neues Spiel.}
\entityintro{NewGameViewHandler}{de.sswis.controller.handlers.NewGameViewHandler}{Öffnet eine View zum Erstellen eines \texttt{\small Spiels}.}
\entityintro{NewInitializationHandler}{de.sswis.controller.handlers.NewInitializationHandler}{Erstellt eine neue Initialisierung.}
\entityintro{NewInitializationViewHandler}{de.sswis.controller.handlers.NewInitializationViewHandler}{Öffnet eine View zum Erstellen einer \texttt{\small Initialisierung}.}
\entityintro{NewStrategyHandler}{de.sswis.controller.handlers.NewStrategyHandler}{Erstellt eine neue kombinierte Strategie.}
\entityintro{NewStrategyViewHandler}{de.sswis.controller.handlers.NewStrategyViewHandler}{Öffnet eine View zum Erstellen einer \texttt{\small kombinierten Strategie}.}
\entityintro{SaveAndQuitHandler}{de.sswis.controller.handlers.SaveAndQuitHandler}{Speichert die Objekte in der View und schließt die View.}
\entityintro{SaveConfigurationsHandler}{de.sswis.controller.handlers.SaveConfigurationsHandler}{Speichert die erstellte \texttt{\small Konfiguration}.}
\entityintro{SaveGamesHandler}{de.sswis.controller.handlers.SaveGamesHandler}{Speichert das erstellte \texttt{\small Spiel}.}
\entityintro{SaveInitializationsHandler}{de.sswis.controller.handlers.SaveInitializationsHandler}{Speichert die erstellte \texttt{\small Initialisierung}.}
\entityintro{SaveResultsHandler}{de.sswis.controller.handlers.SaveResultsHandler}{Speichert die erstellten \texttt{\small Ergebnisse}.}
\entityintro{SaveStrategiesHandler}{de.sswis.controller.handlers.SaveStrategiesHandler}{Speichert die erstellte \texttt{\small kombinierte Strategie}.}
\entityintro{ShowResultsHandler}{de.sswis.controller.handlers.ShowResultsHandler}{Öffnet die View mit der Ergebnissansicht der ausgewählten \texttt{\small Konfigurationen}.}
\entityintro{StartSimulationHandler}{de.sswis.controller.handlers.StartSimulationHandler}{Startet die ausgewählten \texttt{\small Simulationen}.}
\vskip .1in
\vskip .1in
\section{\label{de.sswis.controller.handlers.CancleHandler}Class CancleHandler}{
\hypertarget{de.sswis.controller.handlers.CancleHandler}{}\vskip .1in 
Ein \texttt{\small Hanlder} zum Schließen einer View.\vskip .1in 
\subsection{Declaration}{
\begin{lstlisting}[frame=none]
public class CancleHandler
 extends java.lang.Object implements java.awt.event.ActionListener\end{lstlisting}
\subsection{Constructors}{
\vskip -2em
\begin{itemize}
\item{ 
\index{CancleHandler()}
\hypertarget{de.sswis.controller.handlers.CancleHandler()}{{\bf  CancleHandler}\\}
\begin{lstlisting}[frame=none]
public CancleHandler()\end{lstlisting} %end signature
}%end item
\end{itemize}
}
\subsection{Methods}{
\vskip -2em
\begin{itemize}
\item{ 
\index{actionPerformed(ActionEvent)}
\hypertarget{de.sswis.controller.handlers.CancleHandler.actionPerformed(java.awt.event.ActionEvent)}{{\bf  actionPerformed}\\}
\begin{lstlisting}[frame=none]
void actionPerformed(java.awt.event.ActionEvent arg0)\end{lstlisting} %end signature
}%end item
\end{itemize}
}
}
\section{\label{de.sswis.controller.handlers.CompareResultsHandler}Class CompareResultsHandler}{
\hypertarget{de.sswis.controller.handlers.CompareResultsHandler}{}\vskip .1in 
Öffnet die View zum Vergleichen von Ergebnissen.\vskip .1in 
\subsection{Declaration}{
\begin{lstlisting}[frame=none]
public class CompareResultsHandler
 extends java.lang.Object implements java.awt.event.ActionListener\end{lstlisting}
\subsection{Constructors}{
\vskip -2em
\begin{itemize}
\item{ 
\index{CompareResultsHandler(AbstractGuiFactory)}
\hypertarget{de.sswis.controller.handlers.CompareResultsHandler(de.sswis.controller.AbstractGuiFactory)}{{\bf  CompareResultsHandler}\\}
\begin{lstlisting}[frame=none]
public CompareResultsHandler(de.sswis.controller.AbstractGuiFactory factory)\end{lstlisting} %end signature
\begin{itemize}
\item{
{\bf  Parameters}
  \begin{itemize}
   \item{
\texttt{factory} -- Fabrik zum Erstellen der View}
  \end{itemize}
}%end item
\end{itemize}
}%end item
\end{itemize}
}
\subsection{Methods}{
\vskip -2em
\begin{itemize}
\item{ 
\index{actionPerformed(ActionEvent)}
\hypertarget{de.sswis.controller.handlers.CompareResultsHandler.actionPerformed(java.awt.event.ActionEvent)}{{\bf  actionPerformed}\\}
\begin{lstlisting}[frame=none]
void actionPerformed(java.awt.event.ActionEvent arg0)\end{lstlisting} %end signature
}%end item
\end{itemize}
}
}
\section{\label{de.sswis.controller.handlers.DeleteConfigurationHandler}Class DeleteConfigurationHandler}{
\hypertarget{de.sswis.controller.handlers.DeleteConfigurationHandler}{}\vskip .1in 
Löscht die ausgewählte \texttt{\small Konfiguration}.\vskip .1in 
\subsection{Declaration}{
\begin{lstlisting}[frame=none]
public class DeleteConfigurationHandler
 extends java.lang.Object implements java.awt.event.ActionListener\end{lstlisting}
\subsection{Constructors}{
\vskip -2em
\begin{itemize}
\item{ 
\index{DeleteConfigurationHandler()}
\hypertarget{de.sswis.controller.handlers.DeleteConfigurationHandler()}{{\bf  DeleteConfigurationHandler}\\}
\begin{lstlisting}[frame=none]
public DeleteConfigurationHandler()\end{lstlisting} %end signature
}%end item
\end{itemize}
}
\subsection{Methods}{
\vskip -2em
\begin{itemize}
\item{ 
\index{actionPerformed(ActionEvent)}
\hypertarget{de.sswis.controller.handlers.DeleteConfigurationHandler.actionPerformed(java.awt.event.ActionEvent)}{{\bf  actionPerformed}\\}
\begin{lstlisting}[frame=none]
void actionPerformed(java.awt.event.ActionEvent arg0)\end{lstlisting} %end signature
}%end item
\end{itemize}
}
}
\section{\label{de.sswis.controller.handlers.DeleteGameHandler}Class DeleteGameHandler}{
\hypertarget{de.sswis.controller.handlers.DeleteGameHandler}{}\vskip .1in 
Löscht das ausgewählte \texttt{\small Spiel}.\vskip .1in 
\subsection{Declaration}{
\begin{lstlisting}[frame=none]
public class DeleteGameHandler
 extends java.lang.Object implements java.awt.event.ActionListener\end{lstlisting}
\subsection{Constructors}{
\vskip -2em
\begin{itemize}
\item{ 
\index{DeleteGameHandler()}
\hypertarget{de.sswis.controller.handlers.DeleteGameHandler()}{{\bf  DeleteGameHandler}\\}
\begin{lstlisting}[frame=none]
public DeleteGameHandler()\end{lstlisting} %end signature
}%end item
\end{itemize}
}
\subsection{Methods}{
\vskip -2em
\begin{itemize}
\item{ 
\index{actionPerformed(ActionEvent)}
\hypertarget{de.sswis.controller.handlers.DeleteGameHandler.actionPerformed(java.awt.event.ActionEvent)}{{\bf  actionPerformed}\\}
\begin{lstlisting}[frame=none]
void actionPerformed(java.awt.event.ActionEvent arg0)\end{lstlisting} %end signature
}%end item
\end{itemize}
}
}
\section{\label{de.sswis.controller.handlers.DeleteInitializationHandler}Class DeleteInitializationHandler}{
\hypertarget{de.sswis.controller.handlers.DeleteInitializationHandler}{}\vskip .1in 
Löscht die ausgewählte \texttt{\small Initialisierung}.\vskip .1in 
\subsection{Declaration}{
\begin{lstlisting}[frame=none]
public class DeleteInitializationHandler
 extends java.lang.Object implements java.awt.event.ActionListener\end{lstlisting}
\subsection{Constructors}{
\vskip -2em
\begin{itemize}
\item{ 
\index{DeleteInitializationHandler()}
\hypertarget{de.sswis.controller.handlers.DeleteInitializationHandler()}{{\bf  DeleteInitializationHandler}\\}
\begin{lstlisting}[frame=none]
public DeleteInitializationHandler()\end{lstlisting} %end signature
}%end item
\end{itemize}
}
\subsection{Methods}{
\vskip -2em
\begin{itemize}
\item{ 
\index{actionPerformed(ActionEvent)}
\hypertarget{de.sswis.controller.handlers.DeleteInitializationHandler.actionPerformed(java.awt.event.ActionEvent)}{{\bf  actionPerformed}\\}
\begin{lstlisting}[frame=none]
void actionPerformed(java.awt.event.ActionEvent arg0)\end{lstlisting} %end signature
}%end item
\end{itemize}
}
}
\section{\label{de.sswis.controller.handlers.DeleteResultHandler}Class DeleteResultHandler}{
\hypertarget{de.sswis.controller.handlers.DeleteResultHandler}{}\vskip .1in 
Löscht das ausgewählte \texttt{\small Ergebnis}.\vskip .1in 
\subsection{Declaration}{
\begin{lstlisting}[frame=none]
public class DeleteResultHandler
 extends java.lang.Object implements java.awt.event.ActionListener\end{lstlisting}
\subsection{Constructors}{
\vskip -2em
\begin{itemize}
\item{ 
\index{DeleteResultHandler()}
\hypertarget{de.sswis.controller.handlers.DeleteResultHandler()}{{\bf  DeleteResultHandler}\\}
\begin{lstlisting}[frame=none]
public DeleteResultHandler()\end{lstlisting} %end signature
}%end item
\end{itemize}
}
\subsection{Methods}{
\vskip -2em
\begin{itemize}
\item{ 
\index{actionPerformed(ActionEvent)}
\hypertarget{de.sswis.controller.handlers.DeleteResultHandler.actionPerformed(java.awt.event.ActionEvent)}{{\bf  actionPerformed}\\}
\begin{lstlisting}[frame=none]
void actionPerformed(java.awt.event.ActionEvent arg0)\end{lstlisting} %end signature
}%end item
\end{itemize}
}
}
\section{\label{de.sswis.controller.handlers.DeleteStrategyHandler}Class DeleteStrategyHandler}{
\hypertarget{de.sswis.controller.handlers.DeleteStrategyHandler}{}\vskip .1in 
Löscht die ausgewählte \texttt{\small Strategie}.\vskip .1in 
\subsection{Declaration}{
\begin{lstlisting}[frame=none]
public class DeleteStrategyHandler
 extends java.lang.Object implements java.awt.event.ActionListener\end{lstlisting}
\subsection{Constructors}{
\vskip -2em
\begin{itemize}
\item{ 
\index{DeleteStrategyHandler()}
\hypertarget{de.sswis.controller.handlers.DeleteStrategyHandler()}{{\bf  DeleteStrategyHandler}\\}
\begin{lstlisting}[frame=none]
public DeleteStrategyHandler()\end{lstlisting} %end signature
}%end item
\end{itemize}
}
\subsection{Methods}{
\vskip -2em
\begin{itemize}
\item{ 
\index{actionPerformed(ActionEvent)}
\hypertarget{de.sswis.controller.handlers.DeleteStrategyHandler.actionPerformed(java.awt.event.ActionEvent)}{{\bf  actionPerformed}\\}
\begin{lstlisting}[frame=none]
void actionPerformed(java.awt.event.ActionEvent arg0)\end{lstlisting} %end signature
}%end item
\end{itemize}
}
}
\section{\label{de.sswis.controller.handlers.EditConfigurationHandler}Class EditConfigurationHandler}{
\hypertarget{de.sswis.controller.handlers.EditConfigurationHandler}{}\vskip .1in 
Öffnet die View zum Bearbeiten einer \texttt{\small Konfiguration}.\vskip .1in 
\subsection{Declaration}{
\begin{lstlisting}[frame=none]
public class EditConfigurationHandler
 extends java.lang.Object implements java.awt.event.ActionListener\end{lstlisting}
\subsection{Constructors}{
\vskip -2em
\begin{itemize}
\item{ 
\index{EditConfigurationHandler(AbstractGuiFactory)}
\hypertarget{de.sswis.controller.handlers.EditConfigurationHandler(de.sswis.controller.AbstractGuiFactory)}{{\bf  EditConfigurationHandler}\\}
\begin{lstlisting}[frame=none]
public EditConfigurationHandler(de.sswis.controller.AbstractGuiFactory factory)\end{lstlisting} %end signature
\begin{itemize}
\item{
{\bf  Parameters}
  \begin{itemize}
   \item{
\texttt{factory} -- Fabrik zum Erstellen der View}
  \end{itemize}
}%end item
\end{itemize}
}%end item
\end{itemize}
}
\subsection{Methods}{
\vskip -2em
\begin{itemize}
\item{ 
\index{actionPerformed(ActionEvent)}
\hypertarget{de.sswis.controller.handlers.EditConfigurationHandler.actionPerformed(java.awt.event.ActionEvent)}{{\bf  actionPerformed}\\}
\begin{lstlisting}[frame=none]
void actionPerformed(java.awt.event.ActionEvent arg0)\end{lstlisting} %end signature
}%end item
\end{itemize}
}
}
\section{\label{de.sswis.controller.handlers.EditGameHandler}Class EditGameHandler}{
\hypertarget{de.sswis.controller.handlers.EditGameHandler}{}\vskip .1in 
Öffnet die View zum Bearbeiten eines \texttt{\small Spiels}.\vskip .1in 
\subsection{Declaration}{
\begin{lstlisting}[frame=none]
public class EditGameHandler
 extends java.lang.Object implements java.awt.event.ActionListener\end{lstlisting}
\subsection{Constructors}{
\vskip -2em
\begin{itemize}
\item{ 
\index{EditGameHandler(AbstractGuiFactory)}
\hypertarget{de.sswis.controller.handlers.EditGameHandler(de.sswis.controller.AbstractGuiFactory)}{{\bf  EditGameHandler}\\}
\begin{lstlisting}[frame=none]
public EditGameHandler(de.sswis.controller.AbstractGuiFactory factory)\end{lstlisting} %end signature
\begin{itemize}
\item{
{\bf  Parameters}
  \begin{itemize}
   \item{
\texttt{factory} -- Fabrik zum Erstellen der View}
  \end{itemize}
}%end item
\end{itemize}
}%end item
\end{itemize}
}
\subsection{Methods}{
\vskip -2em
\begin{itemize}
\item{ 
\index{actionPerformed(ActionEvent)}
\hypertarget{de.sswis.controller.handlers.EditGameHandler.actionPerformed(java.awt.event.ActionEvent)}{{\bf  actionPerformed}\\}
\begin{lstlisting}[frame=none]
void actionPerformed(java.awt.event.ActionEvent arg0)\end{lstlisting} %end signature
}%end item
\end{itemize}
}
}
\section{\label{de.sswis.controller.handlers.EditInitializationHandler}Class EditInitializationHandler}{
\hypertarget{de.sswis.controller.handlers.EditInitializationHandler}{}\vskip .1in 
Öffnet die View zum Bearbeiten einer \texttt{\small Initialisierung}.\vskip .1in 
\subsection{Declaration}{
\begin{lstlisting}[frame=none]
public class EditInitializationHandler
 extends java.lang.Object implements java.awt.event.ActionListener\end{lstlisting}
\subsection{Constructors}{
\vskip -2em
\begin{itemize}
\item{ 
\index{EditInitializationHandler(AbstractGuiFactory)}
\hypertarget{de.sswis.controller.handlers.EditInitializationHandler(de.sswis.controller.AbstractGuiFactory)}{{\bf  EditInitializationHandler}\\}
\begin{lstlisting}[frame=none]
public EditInitializationHandler(de.sswis.controller.AbstractGuiFactory factory)\end{lstlisting} %end signature
\begin{itemize}
\item{
{\bf  Parameters}
  \begin{itemize}
   \item{
\texttt{factory} -- Fabrik zum Erstellen der View}
  \end{itemize}
}%end item
\end{itemize}
}%end item
\end{itemize}
}
\subsection{Methods}{
\vskip -2em
\begin{itemize}
\item{ 
\index{actionPerformed(ActionEvent)}
\hypertarget{de.sswis.controller.handlers.EditInitializationHandler.actionPerformed(java.awt.event.ActionEvent)}{{\bf  actionPerformed}\\}
\begin{lstlisting}[frame=none]
void actionPerformed(java.awt.event.ActionEvent arg0)\end{lstlisting} %end signature
}%end item
\end{itemize}
}
}
\section{\label{de.sswis.controller.handlers.EditStrategyHandler}Class EditStrategyHandler}{
\hypertarget{de.sswis.controller.handlers.EditStrategyHandler}{}\vskip .1in 
Öffnet die View zum Bearbeiten einer \texttt{\small Strategie}.\vskip .1in 
\subsection{Declaration}{
\begin{lstlisting}[frame=none]
public class EditStrategyHandler
 extends java.lang.Object implements java.awt.event.ActionListener\end{lstlisting}
\subsection{Constructors}{
\vskip -2em
\begin{itemize}
\item{ 
\index{EditStrategyHandler(AbstractGuiFactory)}
\hypertarget{de.sswis.controller.handlers.EditStrategyHandler(de.sswis.controller.AbstractGuiFactory)}{{\bf  EditStrategyHandler}\\}
\begin{lstlisting}[frame=none]
public EditStrategyHandler(de.sswis.controller.AbstractGuiFactory factory)\end{lstlisting} %end signature
\begin{itemize}
\item{
{\bf  Parameters}
  \begin{itemize}
   \item{
\texttt{factory} -- Fabrik zum Erstellen der View}
  \end{itemize}
}%end item
\end{itemize}
}%end item
\end{itemize}
}
\subsection{Methods}{
\vskip -2em
\begin{itemize}
\item{ 
\index{actionPerformed(ActionEvent)}
\hypertarget{de.sswis.controller.handlers.EditStrategyHandler.actionPerformed(java.awt.event.ActionEvent)}{{\bf  actionPerformed}\\}
\begin{lstlisting}[frame=none]
void actionPerformed(java.awt.event.ActionEvent arg0)\end{lstlisting} %end signature
}%end item
\end{itemize}
}
}
\section{\label{de.sswis.controller.handlers.ManageConfigurationsHandler}Class ManageConfigurationsHandler}{
\hypertarget{de.sswis.controller.handlers.ManageConfigurationsHandler}{}\vskip .1in 
Öffnet die View zum Verwalten der \texttt{\small Konfigurationen}.\vskip .1in 
\subsection{Declaration}{
\begin{lstlisting}[frame=none]
public class ManageConfigurationsHandler
 extends java.lang.Object implements java.awt.event.ActionListener\end{lstlisting}
\subsection{Constructors}{
\vskip -2em
\begin{itemize}
\item{ 
\index{ManageConfigurationsHandler(AbstractGuiFactory)}
\hypertarget{de.sswis.controller.handlers.ManageConfigurationsHandler(de.sswis.controller.AbstractGuiFactory)}{{\bf  ManageConfigurationsHandler}\\}
\begin{lstlisting}[frame=none]
public ManageConfigurationsHandler(de.sswis.controller.AbstractGuiFactory factory)\end{lstlisting} %end signature
\begin{itemize}
\item{
{\bf  Parameters}
  \begin{itemize}
   \item{
\texttt{factory} -- Fabrik zum Erstellen der View}
  \end{itemize}
}%end item
\end{itemize}
}%end item
\end{itemize}
}
\subsection{Methods}{
\vskip -2em
\begin{itemize}
\item{ 
\index{actionPerformed(ActionEvent)}
\hypertarget{de.sswis.controller.handlers.ManageConfigurationsHandler.actionPerformed(java.awt.event.ActionEvent)}{{\bf  actionPerformed}\\}
\begin{lstlisting}[frame=none]
void actionPerformed(java.awt.event.ActionEvent arg0)\end{lstlisting} %end signature
}%end item
\end{itemize}
}
}
\section{\label{de.sswis.controller.handlers.ManageGamesHandler}Class ManageGamesHandler}{
\hypertarget{de.sswis.controller.handlers.ManageGamesHandler}{}\vskip .1in 
Öffnet die View zum Verwalten der \texttt{\small Spiele}.\vskip .1in 
\subsection{Declaration}{
\begin{lstlisting}[frame=none]
public class ManageGamesHandler
 extends java.lang.Object implements java.awt.event.ActionListener\end{lstlisting}
\subsection{Constructors}{
\vskip -2em
\begin{itemize}
\item{ 
\index{ManageGamesHandler(AbstractGuiFactory)}
\hypertarget{de.sswis.controller.handlers.ManageGamesHandler(de.sswis.controller.AbstractGuiFactory)}{{\bf  ManageGamesHandler}\\}
\begin{lstlisting}[frame=none]
public ManageGamesHandler(de.sswis.controller.AbstractGuiFactory factory)\end{lstlisting} %end signature
\begin{itemize}
\item{
{\bf  Parameters}
  \begin{itemize}
   \item{
\texttt{factory} -- Fabrik zum Erstellen der View}
  \end{itemize}
}%end item
\end{itemize}
}%end item
\end{itemize}
}
\subsection{Methods}{
\vskip -2em
\begin{itemize}
\item{ 
\index{actionPerformed(ActionEvent)}
\hypertarget{de.sswis.controller.handlers.ManageGamesHandler.actionPerformed(java.awt.event.ActionEvent)}{{\bf  actionPerformed}\\}
\begin{lstlisting}[frame=none]
void actionPerformed(java.awt.event.ActionEvent arg0)\end{lstlisting} %end signature
}%end item
\end{itemize}
}
}
\section{\label{de.sswis.controller.handlers.ManageInitializationHandler}Class ManageInitializationHandler}{
\hypertarget{de.sswis.controller.handlers.ManageInitializationHandler}{}\vskip .1in 
Öffnet die View zum Verwalten der \texttt{\small Initialisierungen}.\vskip .1in 
\subsection{Declaration}{
\begin{lstlisting}[frame=none]
public class ManageInitializationHandler
 extends java.lang.Object implements java.awt.event.ActionListener\end{lstlisting}
\subsection{Constructors}{
\vskip -2em
\begin{itemize}
\item{ 
\index{ManageInitializationHandler(AbstractGuiFactory)}
\hypertarget{de.sswis.controller.handlers.ManageInitializationHandler(de.sswis.controller.AbstractGuiFactory)}{{\bf  ManageInitializationHandler}\\}
\begin{lstlisting}[frame=none]
public ManageInitializationHandler(de.sswis.controller.AbstractGuiFactory factory)\end{lstlisting} %end signature
\begin{itemize}
\item{
{\bf  Parameters}
  \begin{itemize}
   \item{
\texttt{factory} -- Fabrik zum Erstellen der View}
  \end{itemize}
}%end item
\end{itemize}
}%end item
\end{itemize}
}
\subsection{Methods}{
\vskip -2em
\begin{itemize}
\item{ 
\index{actionPerformed(ActionEvent)}
\hypertarget{de.sswis.controller.handlers.ManageInitializationHandler.actionPerformed(java.awt.event.ActionEvent)}{{\bf  actionPerformed}\\}
\begin{lstlisting}[frame=none]
void actionPerformed(java.awt.event.ActionEvent arg0)\end{lstlisting} %end signature
}%end item
\end{itemize}
}
}
\section{\label{de.sswis.controller.handlers.ManageResultsHandler}Class ManageResultsHandler}{
\hypertarget{de.sswis.controller.handlers.ManageResultsHandler}{}\vskip .1in 
Öffnet die View zum Verwalten der \texttt{\small Ergebnisse}.\vskip .1in 
\subsection{Declaration}{
\begin{lstlisting}[frame=none]
public class ManageResultsHandler
 extends java.lang.Object implements java.awt.event.ActionListener\end{lstlisting}
\subsection{Constructors}{
\vskip -2em
\begin{itemize}
\item{ 
\index{ManageResultsHandler(AbstractGuiFactory)}
\hypertarget{de.sswis.controller.handlers.ManageResultsHandler(de.sswis.controller.AbstractGuiFactory)}{{\bf  ManageResultsHandler}\\}
\begin{lstlisting}[frame=none]
public ManageResultsHandler(de.sswis.controller.AbstractGuiFactory factory)\end{lstlisting} %end signature
\begin{itemize}
\item{
{\bf  Parameters}
  \begin{itemize}
   \item{
\texttt{factory} -- Fabrik zum Erstellen der View}
  \end{itemize}
}%end item
\end{itemize}
}%end item
\end{itemize}
}
\subsection{Methods}{
\vskip -2em
\begin{itemize}
\item{ 
\index{actionPerformed(ActionEvent)}
\hypertarget{de.sswis.controller.handlers.ManageResultsHandler.actionPerformed(java.awt.event.ActionEvent)}{{\bf  actionPerformed}\\}
\begin{lstlisting}[frame=none]
void actionPerformed(java.awt.event.ActionEvent arg0)\end{lstlisting} %end signature
}%end item
\end{itemize}
}
}
\section{\label{de.sswis.controller.handlers.ManageStrategiesHandler}Class ManageStrategiesHandler}{
\hypertarget{de.sswis.controller.handlers.ManageStrategiesHandler}{}\vskip .1in 
Öffnet die View zum Verwalten der \texttt{\small Strategien}.\vskip .1in 
\subsection{Declaration}{
\begin{lstlisting}[frame=none]
public class ManageStrategiesHandler
 extends java.lang.Object implements java.awt.event.ActionListener\end{lstlisting}
\subsection{Constructors}{
\vskip -2em
\begin{itemize}
\item{ 
\index{ManageStrategiesHandler(AbstractGuiFactory)}
\hypertarget{de.sswis.controller.handlers.ManageStrategiesHandler(de.sswis.controller.AbstractGuiFactory)}{{\bf  ManageStrategiesHandler}\\}
\begin{lstlisting}[frame=none]
public ManageStrategiesHandler(de.sswis.controller.AbstractGuiFactory factory)\end{lstlisting} %end signature
\begin{itemize}
\item{
{\bf  Parameters}
  \begin{itemize}
   \item{
\texttt{factory} -- Fabrik zum Erstellen der View}
  \end{itemize}
}%end item
\end{itemize}
}%end item
\end{itemize}
}
\subsection{Methods}{
\vskip -2em
\begin{itemize}
\item{ 
\index{actionPerformed(ActionEvent)}
\hypertarget{de.sswis.controller.handlers.ManageStrategiesHandler.actionPerformed(java.awt.event.ActionEvent)}{{\bf  actionPerformed}\\}
\begin{lstlisting}[frame=none]
void actionPerformed(java.awt.event.ActionEvent arg0)\end{lstlisting} %end signature
}%end item
\end{itemize}
}
}
\section{\label{de.sswis.controller.handlers.NewConfigurationHandler}Class NewConfigurationHandler}{
\hypertarget{de.sswis.controller.handlers.NewConfigurationHandler}{}\vskip .1in 
Erstellt eine neuen Konfiguration. In der View zum Verwalten der \texttt{\small Konfigurationen} wird eine neue \texttt{\small Konfiguration} hinzugefügt und es öffnet sich die View zum Bearbeiten der neuen \texttt{\small Konfiguration}.\vskip .1in 
\subsection{Declaration}{
\begin{lstlisting}[frame=none]
public class NewConfigurationHandler
 extends java.lang.Object implements java.awt.event.ActionListener\end{lstlisting}
\subsection{Constructors}{
\vskip -2em
\begin{itemize}
\item{ 
\index{NewConfigurationHandler(AbstractGuiFactory)}
\hypertarget{de.sswis.controller.handlers.NewConfigurationHandler(de.sswis.controller.AbstractGuiFactory)}{{\bf  NewConfigurationHandler}\\}
\begin{lstlisting}[frame=none]
public NewConfigurationHandler(de.sswis.controller.AbstractGuiFactory factory)\end{lstlisting} %end signature
\begin{itemize}
\item{
{\bf  Parameters}
  \begin{itemize}
   \item{
\texttt{factory} -- Fabrik zum Erstellen der View}
  \end{itemize}
}%end item
\end{itemize}
}%end item
\end{itemize}
}
\subsection{Methods}{
\vskip -2em
\begin{itemize}
\item{ 
\index{actionPerformed(ActionEvent)}
\hypertarget{de.sswis.controller.handlers.NewConfigurationHandler.actionPerformed(java.awt.event.ActionEvent)}{{\bf  actionPerformed}\\}
\begin{lstlisting}[frame=none]
void actionPerformed(java.awt.event.ActionEvent arg0)\end{lstlisting} %end signature
}%end item
\end{itemize}
}
}
\section{\label{de.sswis.controller.handlers.NewConfigurationViewHandler}Class NewConfigurationViewHandler}{
\hypertarget{de.sswis.controller.handlers.NewConfigurationViewHandler}{}\vskip .1in 
Öffnet eine View zum Erstellen einer \texttt{\small Konfiguration}.\vskip .1in 
\subsection{Declaration}{
\begin{lstlisting}[frame=none]
public class NewConfigurationViewHandler
 extends java.lang.Object implements java.awt.event.ActionListener\end{lstlisting}
\subsection{Constructors}{
\vskip -2em
\begin{itemize}
\item{ 
\index{NewConfigurationViewHandler(AbstractGuiFactory)}
\hypertarget{de.sswis.controller.handlers.NewConfigurationViewHandler(de.sswis.controller.AbstractGuiFactory)}{{\bf  NewConfigurationViewHandler}\\}
\begin{lstlisting}[frame=none]
public NewConfigurationViewHandler(de.sswis.controller.AbstractGuiFactory factory)\end{lstlisting} %end signature
\begin{itemize}
\item{
{\bf  Parameters}
  \begin{itemize}
   \item{
\texttt{factory} -- Fabrik zum Erstellen der View}
  \end{itemize}
}%end item
\end{itemize}
}%end item
\end{itemize}
}
\subsection{Methods}{
\vskip -2em
\begin{itemize}
\item{ 
\index{actionPerformed(ActionEvent)}
\hypertarget{de.sswis.controller.handlers.NewConfigurationViewHandler.actionPerformed(java.awt.event.ActionEvent)}{{\bf  actionPerformed}\\}
\begin{lstlisting}[frame=none]
void actionPerformed(java.awt.event.ActionEvent arg0)\end{lstlisting} %end signature
}%end item
\end{itemize}
}
}
\section{\label{de.sswis.controller.handlers.NewGameHandler}Class NewGameHandler}{
\hypertarget{de.sswis.controller.handlers.NewGameHandler}{}\vskip .1in 
Erstellt ein neues Spiel. In der View zum Verwalten der \texttt{\small Spiele} wird ein neues \texttt{\small Spiel} hinzugefügt und es öffnet sich die View zum Bearbeiten der neuen \texttt{\small Spiele}.\vskip .1in 
\subsection{Declaration}{
\begin{lstlisting}[frame=none]
public class NewGameHandler
 extends java.lang.Object implements java.awt.event.ActionListener\end{lstlisting}
\subsection{Constructors}{
\vskip -2em
\begin{itemize}
\item{ 
\index{NewGameHandler(AbstractGuiFactory)}
\hypertarget{de.sswis.controller.handlers.NewGameHandler(de.sswis.controller.AbstractGuiFactory)}{{\bf  NewGameHandler}\\}
\begin{lstlisting}[frame=none]
public NewGameHandler(de.sswis.controller.AbstractGuiFactory factory)\end{lstlisting} %end signature
\begin{itemize}
\item{
{\bf  Parameters}
  \begin{itemize}
   \item{
\texttt{factory} -- Fabrik zum Erstellen der View}
  \end{itemize}
}%end item
\end{itemize}
}%end item
\end{itemize}
}
\subsection{Methods}{
\vskip -2em
\begin{itemize}
\item{ 
\index{actionPerformed(ActionEvent)}
\hypertarget{de.sswis.controller.handlers.NewGameHandler.actionPerformed(java.awt.event.ActionEvent)}{{\bf  actionPerformed}\\}
\begin{lstlisting}[frame=none]
void actionPerformed(java.awt.event.ActionEvent arg0)\end{lstlisting} %end signature
}%end item
\end{itemize}
}
}
\section{\label{de.sswis.controller.handlers.NewGameViewHandler}Class NewGameViewHandler}{
\hypertarget{de.sswis.controller.handlers.NewGameViewHandler}{}\vskip .1in 
Öffnet eine View zum Erstellen eines \texttt{\small Spiels}.\vskip .1in 
\subsection{Declaration}{
\begin{lstlisting}[frame=none]
public class NewGameViewHandler
 extends java.lang.Object implements java.awt.event.ActionListener\end{lstlisting}
\subsection{Constructors}{
\vskip -2em
\begin{itemize}
\item{ 
\index{NewGameViewHandler()}
\hypertarget{de.sswis.controller.handlers.NewGameViewHandler()}{{\bf  NewGameViewHandler}\\}
\begin{lstlisting}[frame=none]
public NewGameViewHandler()\end{lstlisting} %end signature
}%end item
\end{itemize}
}
\subsection{Methods}{
\vskip -2em
\begin{itemize}
\item{ 
\index{actionPerformed(ActionEvent)}
\hypertarget{de.sswis.controller.handlers.NewGameViewHandler.actionPerformed(java.awt.event.ActionEvent)}{{\bf  actionPerformed}\\}
\begin{lstlisting}[frame=none]
void actionPerformed(java.awt.event.ActionEvent arg0)\end{lstlisting} %end signature
}%end item
\end{itemize}
}
}
\section{\label{de.sswis.controller.handlers.NewInitializationHandler}Class NewInitializationHandler}{
\hypertarget{de.sswis.controller.handlers.NewInitializationHandler}{}\vskip .1in 
Erstellt eine neue Initialisierung. In der View zum Verwalten der \texttt{\small Initialisierungen} wird eine neue \texttt{\small Initialisierung} hinzugefügt und es öffnet sich die View zum Bearbeiten der neuen \texttt{\small Initialisierung}.\vskip .1in 
\subsection{Declaration}{
\begin{lstlisting}[frame=none]
public class NewInitializationHandler
 extends java.lang.Object implements java.awt.event.ActionListener\end{lstlisting}
\subsection{Constructors}{
\vskip -2em
\begin{itemize}
\item{ 
\index{NewInitializationHandler(AbstractGuiFactory)}
\hypertarget{de.sswis.controller.handlers.NewInitializationHandler(de.sswis.controller.AbstractGuiFactory)}{{\bf  NewInitializationHandler}\\}
\begin{lstlisting}[frame=none]
public NewInitializationHandler(de.sswis.controller.AbstractGuiFactory factory)\end{lstlisting} %end signature
\begin{itemize}
\item{
{\bf  Parameters}
  \begin{itemize}
   \item{
\texttt{factory} -- Fabrik zum Erstellen der View}
  \end{itemize}
}%end item
\end{itemize}
}%end item
\end{itemize}
}
\subsection{Methods}{
\vskip -2em
\begin{itemize}
\item{ 
\index{actionPerformed(ActionEvent)}
\hypertarget{de.sswis.controller.handlers.NewInitializationHandler.actionPerformed(java.awt.event.ActionEvent)}{{\bf  actionPerformed}\\}
\begin{lstlisting}[frame=none]
void actionPerformed(java.awt.event.ActionEvent arg0)\end{lstlisting} %end signature
}%end item
\end{itemize}
}
}
\section{\label{de.sswis.controller.handlers.NewInitializationViewHandler}Class NewInitializationViewHandler}{
\hypertarget{de.sswis.controller.handlers.NewInitializationViewHandler}{}\vskip .1in 
Öffnet eine View zum Erstellen einer \texttt{\small Initialisierung}.\vskip .1in 
\subsection{Declaration}{
\begin{lstlisting}[frame=none]
public class NewInitializationViewHandler
 extends java.lang.Object implements java.awt.event.ActionListener\end{lstlisting}
\subsection{Constructors}{
\vskip -2em
\begin{itemize}
\item{ 
\index{NewInitializationViewHandler(AbstractGuiFactory)}
\hypertarget{de.sswis.controller.handlers.NewInitializationViewHandler(de.sswis.controller.AbstractGuiFactory)}{{\bf  NewInitializationViewHandler}\\}
\begin{lstlisting}[frame=none]
public NewInitializationViewHandler(de.sswis.controller.AbstractGuiFactory factory)\end{lstlisting} %end signature
\begin{itemize}
\item{
{\bf  Parameters}
  \begin{itemize}
   \item{
\texttt{factory} -- Fabrik zum Erstellen der View}
  \end{itemize}
}%end item
\end{itemize}
}%end item
\end{itemize}
}
\subsection{Methods}{
\vskip -2em
\begin{itemize}
\item{ 
\index{actionPerformed(ActionEvent)}
\hypertarget{de.sswis.controller.handlers.NewInitializationViewHandler.actionPerformed(java.awt.event.ActionEvent)}{{\bf  actionPerformed}\\}
\begin{lstlisting}[frame=none]
void actionPerformed(java.awt.event.ActionEvent arg0)\end{lstlisting} %end signature
}%end item
\end{itemize}
}
}
\section{\label{de.sswis.controller.handlers.NewStrategyHandler}Class NewStrategyHandler}{
\hypertarget{de.sswis.controller.handlers.NewStrategyHandler}{}\vskip .1in 
Erstellt eine neue kombinierte Strategie. In der View zum Verwalten der \texttt{\small kombinierten Strategien} wird eine neue \texttt{\small kombinierte Strategie} hinzugefügt und es öffnet sich die View zum Bearbeiten der neuen \texttt{\small kombinierten Strategie}.\vskip .1in 
\subsection{Declaration}{
\begin{lstlisting}[frame=none]
public class NewStrategyHandler
 extends java.lang.Object implements java.awt.event.ActionListener\end{lstlisting}
\subsection{Constructors}{
\vskip -2em
\begin{itemize}
\item{ 
\index{NewStrategyHandler(AbstractGuiFactory)}
\hypertarget{de.sswis.controller.handlers.NewStrategyHandler(de.sswis.controller.AbstractGuiFactory)}{{\bf  NewStrategyHandler}\\}
\begin{lstlisting}[frame=none]
public NewStrategyHandler(de.sswis.controller.AbstractGuiFactory factory)\end{lstlisting} %end signature
\begin{itemize}
\item{
{\bf  Parameters}
  \begin{itemize}
   \item{
\texttt{factory} -- Fabrik zum Erstellen der View}
  \end{itemize}
}%end item
\end{itemize}
}%end item
\end{itemize}
}
\subsection{Methods}{
\vskip -2em
\begin{itemize}
\item{ 
\index{actionPerformed(ActionEvent)}
\hypertarget{de.sswis.controller.handlers.NewStrategyHandler.actionPerformed(java.awt.event.ActionEvent)}{{\bf  actionPerformed}\\}
\begin{lstlisting}[frame=none]
void actionPerformed(java.awt.event.ActionEvent arg0)\end{lstlisting} %end signature
}%end item
\end{itemize}
}
}
\section{\label{de.sswis.controller.handlers.NewStrategyViewHandler}Class NewStrategyViewHandler}{
\hypertarget{de.sswis.controller.handlers.NewStrategyViewHandler}{}\vskip .1in 
Öffnet eine View zum Erstellen einer \texttt{\small kombinierten Strategie}.\vskip .1in 
\subsection{Declaration}{
\begin{lstlisting}[frame=none]
public class NewStrategyViewHandler
 extends java.lang.Object implements java.awt.event.ActionListener\end{lstlisting}
\subsection{Constructors}{
\vskip -2em
\begin{itemize}
\item{ 
\index{NewStrategyViewHandler(AbstractGuiFactory)}
\hypertarget{de.sswis.controller.handlers.NewStrategyViewHandler(de.sswis.controller.AbstractGuiFactory)}{{\bf  NewStrategyViewHandler}\\}
\begin{lstlisting}[frame=none]
public NewStrategyViewHandler(de.sswis.controller.AbstractGuiFactory factory)\end{lstlisting} %end signature
\begin{itemize}
\item{
{\bf  Parameters}
  \begin{itemize}
   \item{
\texttt{factory} -- Fabrik zum Erstellen der View}
  \end{itemize}
}%end item
\end{itemize}
}%end item
\end{itemize}
}
\subsection{Methods}{
\vskip -2em
\begin{itemize}
\item{ 
\index{actionPerformed(ActionEvent)}
\hypertarget{de.sswis.controller.handlers.NewStrategyViewHandler.actionPerformed(java.awt.event.ActionEvent)}{{\bf  actionPerformed}\\}
\begin{lstlisting}[frame=none]
void actionPerformed(java.awt.event.ActionEvent arg0)\end{lstlisting} %end signature
}%end item
\end{itemize}
}
}
\section{\label{de.sswis.controller.handlers.SaveAndQuitHandler}Class SaveAndQuitHandler}{
\hypertarget{de.sswis.controller.handlers.SaveAndQuitHandler}{}\vskip .1in 
Speichert die Objekte in der View und schließt die View.\vskip .1in 
\subsection{Declaration}{
\begin{lstlisting}[frame=none]
public class SaveAndQuitHandler
 extends java.lang.Object implements java.awt.event.ActionListener\end{lstlisting}
\subsection{Constructors}{
\vskip -2em
\begin{itemize}
\item{ 
\index{SaveAndQuitHandler()}
\hypertarget{de.sswis.controller.handlers.SaveAndQuitHandler()}{{\bf  SaveAndQuitHandler}\\}
\begin{lstlisting}[frame=none]
public SaveAndQuitHandler()\end{lstlisting} %end signature
}%end item
\end{itemize}
}
\subsection{Methods}{
\vskip -2em
\begin{itemize}
\item{ 
\index{actionPerformed(ActionEvent)}
\hypertarget{de.sswis.controller.handlers.SaveAndQuitHandler.actionPerformed(java.awt.event.ActionEvent)}{{\bf  actionPerformed}\\}
\begin{lstlisting}[frame=none]
void actionPerformed(java.awt.event.ActionEvent arg0)\end{lstlisting} %end signature
}%end item
\end{itemize}
}
}
\section{\label{de.sswis.controller.handlers.SaveConfigurationsHandler}Class SaveConfigurationsHandler}{
\hypertarget{de.sswis.controller.handlers.SaveConfigurationsHandler}{}\vskip .1in 
Speichert die erstellte \texttt{\small Konfiguration}. Die View, die diesen \texttt{\small ActionListener} verwendet muss eine \texttt{\small Konfiguration} besitzen.\vskip .1in 
\subsection{Declaration}{
\begin{lstlisting}[frame=none]
public class SaveConfigurationsHandler
 extends java.lang.Object implements java.awt.event.ActionListener\end{lstlisting}
\subsection{Constructors}{
\vskip -2em
\begin{itemize}
\item{ 
\index{SaveConfigurationsHandler(AbstractNewConfigurationView)}
\hypertarget{de.sswis.controller.handlers.SaveConfigurationsHandler(de.sswis.view.AbstractNewConfigurationView)}{{\bf  SaveConfigurationsHandler}\\}
\begin{lstlisting}[frame=none]
public SaveConfigurationsHandler(de.sswis.view.AbstractNewConfigurationView configurationView)\end{lstlisting} %end signature
\begin{itemize}
\item{
{\bf  Parameters}
  \begin{itemize}
   \item{
\texttt{configurationView} -- die View mit der zu speichernden \texttt{\small Konfiguration}}
  \end{itemize}
}%end item
\end{itemize}
}%end item
\end{itemize}
}
\subsection{Methods}{
\vskip -2em
\begin{itemize}
\item{ 
\index{actionPerformed(ActionEvent)}
\hypertarget{de.sswis.controller.handlers.SaveConfigurationsHandler.actionPerformed(java.awt.event.ActionEvent)}{{\bf  actionPerformed}\\}
\begin{lstlisting}[frame=none]
void actionPerformed(java.awt.event.ActionEvent arg0)\end{lstlisting} %end signature
}%end item
\end{itemize}
}
}
\section{\label{de.sswis.controller.handlers.SaveGamesHandler}Class SaveGamesHandler}{
\hypertarget{de.sswis.controller.handlers.SaveGamesHandler}{}\vskip .1in 
Speichert das erstellte \texttt{\small Spiel}. Die View, die diesen \texttt{\small ActionListener} verwendet muss ein \texttt{\small Spiel} besitzen.\vskip .1in 
\subsection{Declaration}{
\begin{lstlisting}[frame=none]
public class SaveGamesHandler
 extends java.lang.Object implements java.awt.event.ActionListener\end{lstlisting}
\subsection{Constructors}{
\vskip -2em
\begin{itemize}
\item{ 
\index{SaveGamesHandler(AbstractNewGameView)}
\hypertarget{de.sswis.controller.handlers.SaveGamesHandler(de.sswis.view.AbstractNewGameView)}{{\bf  SaveGamesHandler}\\}
\begin{lstlisting}[frame=none]
public SaveGamesHandler(de.sswis.view.AbstractNewGameView gameView)\end{lstlisting} %end signature
\begin{itemize}
\item{
{\bf  Parameters}
  \begin{itemize}
   \item{
\texttt{gameView} -- die View mit dem zu speichernden \texttt{\small Spiel}}
  \end{itemize}
}%end item
\end{itemize}
}%end item
\end{itemize}
}
\subsection{Methods}{
\vskip -2em
\begin{itemize}
\item{ 
\index{actionPerformed(ActionEvent)}
\hypertarget{de.sswis.controller.handlers.SaveGamesHandler.actionPerformed(java.awt.event.ActionEvent)}{{\bf  actionPerformed}\\}
\begin{lstlisting}[frame=none]
void actionPerformed(java.awt.event.ActionEvent arg0)\end{lstlisting} %end signature
}%end item
\end{itemize}
}
}
\section{\label{de.sswis.controller.handlers.SaveInitializationsHandler}Class SaveInitializationsHandler}{
\hypertarget{de.sswis.controller.handlers.SaveInitializationsHandler}{}\vskip .1in 
Speichert die erstellte \texttt{\small Initialisierung}. Die View, die diesen \texttt{\small ActionListener} verwendet muss eine \texttt{\small Initialisierung} besitzen.\vskip .1in 
\subsection{Declaration}{
\begin{lstlisting}[frame=none]
public class SaveInitializationsHandler
 extends java.lang.Object implements java.awt.event.ActionListener\end{lstlisting}
\subsection{Constructors}{
\vskip -2em
\begin{itemize}
\item{ 
\index{SaveInitializationsHandler(AbstractNewInitializationView)}
\hypertarget{de.sswis.controller.handlers.SaveInitializationsHandler(de.sswis.view.AbstractNewInitializationView)}{{\bf  SaveInitializationsHandler}\\}
\begin{lstlisting}[frame=none]
public SaveInitializationsHandler(de.sswis.view.AbstractNewInitializationView initializationView)\end{lstlisting} %end signature
\begin{itemize}
\item{
{\bf  Parameters}
  \begin{itemize}
   \item{
\texttt{initializationView} -- die View mit der zu speichernden Initialisierung}
  \end{itemize}
}%end item
\end{itemize}
}%end item
\end{itemize}
}
\subsection{Methods}{
\vskip -2em
\begin{itemize}
\item{ 
\index{actionPerformed(ActionEvent)}
\hypertarget{de.sswis.controller.handlers.SaveInitializationsHandler.actionPerformed(java.awt.event.ActionEvent)}{{\bf  actionPerformed}\\}
\begin{lstlisting}[frame=none]
void actionPerformed(java.awt.event.ActionEvent arg0)\end{lstlisting} %end signature
}%end item
\end{itemize}
}
}
\section{\label{de.sswis.controller.handlers.SaveResultsHandler}Class SaveResultsHandler}{
\hypertarget{de.sswis.controller.handlers.SaveResultsHandler}{}\vskip .1in 
Speichert die erstellten \texttt{\small Ergebnisse}. Die View, die diesen \texttt{\small ActionListener} verwendet muss \texttt{\small Ergebnisse} besitzen.\vskip .1in 
\subsection{Declaration}{
\begin{lstlisting}[frame=none]
public class SaveResultsHandler
 extends java.lang.Object implements java.awt.event.ActionListener\end{lstlisting}
\subsection{Constructors}{
\vskip -2em
\begin{itemize}
\item{ 
\index{SaveResultsHandler(AbstractMainView)}
\hypertarget{de.sswis.controller.handlers.SaveResultsHandler(de.sswis.view.AbstractMainView)}{{\bf  SaveResultsHandler}\\}
\begin{lstlisting}[frame=none]
public SaveResultsHandler(de.sswis.view.AbstractMainView mainView)\end{lstlisting} %end signature
\begin{itemize}
\item{
{\bf  Parameters}
  \begin{itemize}
   \item{
\texttt{mainView} -- die View mit den zu speichernden \texttt{\small Ergebnissen}}
  \end{itemize}
}%end item
\end{itemize}
}%end item
\end{itemize}
}
\subsection{Methods}{
\vskip -2em
\begin{itemize}
\item{ 
\index{actionPerformed(ActionEvent)}
\hypertarget{de.sswis.controller.handlers.SaveResultsHandler.actionPerformed(java.awt.event.ActionEvent)}{{\bf  actionPerformed}\\}
\begin{lstlisting}[frame=none]
void actionPerformed(java.awt.event.ActionEvent arg0)\end{lstlisting} %end signature
}%end item
\end{itemize}
}
}
\section{\label{de.sswis.controller.handlers.SaveStrategiesHandler}Class SaveStrategiesHandler}{
\hypertarget{de.sswis.controller.handlers.SaveStrategiesHandler}{}\vskip .1in 
Speichert die erstellte \texttt{\small kombinierte Strategie}. Die View, die diesen \texttt{\small ActionListener} verwendet muss eine \texttt{\small kombinierte Strategie} besitzen.\vskip .1in 
\subsection{Declaration}{
\begin{lstlisting}[frame=none]
public class SaveStrategiesHandler
 extends java.lang.Object implements java.awt.event.ActionListener\end{lstlisting}
\subsection{Constructors}{
\vskip -2em
\begin{itemize}
\item{ 
\index{SaveStrategiesHandler(AbstractNewStrategyView)}
\hypertarget{de.sswis.controller.handlers.SaveStrategiesHandler(de.sswis.view.AbstractNewStrategyView)}{{\bf  SaveStrategiesHandler}\\}
\begin{lstlisting}[frame=none]
public SaveStrategiesHandler(de.sswis.view.AbstractNewStrategyView strategyView)\end{lstlisting} %end signature
\begin{itemize}
\item{
{\bf  Parameters}
  \begin{itemize}
   \item{
\texttt{strategyView} -- die View mit der zu speichernden \texttt{\small kombinierten Strategie}}
  \end{itemize}
}%end item
\end{itemize}
}%end item
\end{itemize}
}
\subsection{Methods}{
\vskip -2em
\begin{itemize}
\item{ 
\index{actionPerformed(ActionEvent)}
\hypertarget{de.sswis.controller.handlers.SaveStrategiesHandler.actionPerformed(java.awt.event.ActionEvent)}{{\bf  actionPerformed}\\}
\begin{lstlisting}[frame=none]
void actionPerformed(java.awt.event.ActionEvent arg0)\end{lstlisting} %end signature
}%end item
\end{itemize}
}
}
\section{\label{de.sswis.controller.handlers.ShowResultsHandler}Class ShowResultsHandler}{
\hypertarget{de.sswis.controller.handlers.ShowResultsHandler}{}\vskip .1in 
Öffnet die View mit der Ergebnissansicht der ausgewählten \texttt{\small Konfigurationen}. Die \texttt{\small Konfigurationen} werden im Hauptfenster ausgewählt.\vskip .1in 
\subsection{Declaration}{
\begin{lstlisting}[frame=none]
public class ShowResultsHandler
 extends java.lang.Object implements java.awt.event.ActionListener\end{lstlisting}
\subsection{Constructors}{
\vskip -2em
\begin{itemize}
\item{ 
\index{ShowResultsHandler(AbstractMainView, AbstractGuiFactory)}
\hypertarget{de.sswis.controller.handlers.ShowResultsHandler(de.sswis.view.AbstractMainView, de.sswis.controller.AbstractGuiFactory)}{{\bf  ShowResultsHandler}\\}
\begin{lstlisting}[frame=none]
public ShowResultsHandler(de.sswis.view.AbstractMainView mainView,de.sswis.controller.AbstractGuiFactory factory)\end{lstlisting} %end signature
\begin{itemize}
\item{
{\bf  Parameters}
  \begin{itemize}
   \item{
\texttt{mainView} -- Hauptfenster mit den ausgewählten \texttt{\small Konfigurationen}}
   \item{
\texttt{factory} -- Fabrik zum Erstellen der View}
  \end{itemize}
}%end item
\end{itemize}
}%end item
\end{itemize}
}
\subsection{Methods}{
\vskip -2em
\begin{itemize}
\item{ 
\index{actionPerformed(ActionEvent)}
\hypertarget{de.sswis.controller.handlers.ShowResultsHandler.actionPerformed(java.awt.event.ActionEvent)}{{\bf  actionPerformed}\\}
\begin{lstlisting}[frame=none]
void actionPerformed(java.awt.event.ActionEvent arg0)\end{lstlisting} %end signature
}%end item
\end{itemize}
}
}
\section{\label{de.sswis.controller.handlers.StartSimulationHandler}Class StartSimulationHandler}{
\hypertarget{de.sswis.controller.handlers.StartSimulationHandler}{}\vskip .1in 
Startet die ausgewählten \texttt{\small Simulationen}. Die \texttt{\small Simulationen} werden im Hauptfenster ausgewählt und sind durch ihre zugehörigen \texttt{\small Konfigurationen} identifiziert.\vskip .1in 
\subsection{Declaration}{
\begin{lstlisting}[frame=none]
public class StartSimulationHandler
 extends java.lang.Object implements java.awt.event.ActionListener\end{lstlisting}
\subsection{Constructors}{
\vskip -2em
\begin{itemize}
\item{ 
\index{StartSimulationHandler(AbstractMainView)}
\hypertarget{de.sswis.controller.handlers.StartSimulationHandler(de.sswis.view.AbstractMainView)}{{\bf  StartSimulationHandler}\\}
\begin{lstlisting}[frame=none]
public StartSimulationHandler(de.sswis.view.AbstractMainView mainView)\end{lstlisting} %end signature
\begin{itemize}
\item{
{\bf  Parameters}
  \begin{itemize}
   \item{
\texttt{mainView} -- Hauptfenster mit den ausgewählten \texttt{\small Simulationen}}
  \end{itemize}
}%end item
\end{itemize}
}%end item
\end{itemize}
}
\subsection{Methods}{
\vskip -2em
\begin{itemize}
\item{ 
\index{actionPerformed(ActionEvent)}
\hypertarget{de.sswis.controller.handlers.StartSimulationHandler.actionPerformed(java.awt.event.ActionEvent)}{{\bf  actionPerformed}\\}
\begin{lstlisting}[frame=none]
void actionPerformed(java.awt.event.ActionEvent arg0)\end{lstlisting} %end signature
}%end item
\end{itemize}
}
}
}
\chapter{Package de.sswis.model}{
\label{de.sswis.model}\hypertarget{de.sswis.model}{}
\hskip -.05in
\hbox to \hsize{\textit{ Package Contents\hfil Page}}
\vskip .13in
\hbox{{\bf  Classes}}
\entityintro{Action}{de.sswis.model.Action}{}
\entityintro{Agent}{de.sswis.model.Agent}{}
\entityintro{CombinedStrategy}{de.sswis.model.CombinedStrategy}{}
\entityintro{Configuration}{de.sswis.model.Configuration}{}
\entityintro{Game}{de.sswis.model.Game}{}
\entityintro{Game.Tuple}{de.sswis.model.Game.Tuple}{}
\entityintro{Group}{de.sswis.model.Group}{}
\entityintro{History}{de.sswis.model.History}{}
\entityintro{Initialization}{de.sswis.model.Initialization}{}
\entityintro{Pair}{de.sswis.model.Pair}{}
\entityintro{Simulation}{de.sswis.model.Simulation}{}
\entityintro{Strategy}{de.sswis.model.Strategy}{}
\vskip .1in
\vskip .1in
\section{\label{de.sswis.model.Action}Class Action}{
\hypertarget{de.sswis.model.Action}{}\vskip .1in 
\subsection{Declaration}{
\begin{lstlisting}[frame=none]
public final class Action
 extends java.lang.Enum\end{lstlisting}
\subsection{Fields}{
\begin{itemize}
\item{
\index{COOPERATION}
\label{de.sswis.model.Action.COOPERATION}\hypertarget{de.sswis.model.Action.COOPERATION}{\texttt{public static final Action\ {\bf  COOPERATION}}
}
}
\item{
\index{DEFECTION}
\label{de.sswis.model.Action.DEFECTION}\hypertarget{de.sswis.model.Action.DEFECTION}{\texttt{public static final Action\ {\bf  DEFECTION}}
}
}
\end{itemize}
}
\subsection{Methods}{
\vskip -2em
\begin{itemize}
\item{ 
\index{valueOf(String)}
\hypertarget{de.sswis.model.Action.valueOf(java.lang.String)}{{\bf  valueOf}\\}
\begin{lstlisting}[frame=none]
public static Action valueOf(java.lang.String name)\end{lstlisting} %end signature
}%end item
\item{ 
\index{values()}
\hypertarget{de.sswis.model.Action.values()}{{\bf  values}\\}
\begin{lstlisting}[frame=none]
public static Action[] values()\end{lstlisting} %end signature
}%end item
\end{itemize}
}
\subsection{Members inherited from class Enum }{
\texttt{java.lang.Enum} {\small 
\refdefined{java.lang.Enum}}
{\small 

clone, compareTo, equals, finalize, getDeclaringClass, hashCode, name, ordinal, toString, valueOf}
}
\section{\label{de.sswis.model.Agent}Class Agent}{
\hypertarget{de.sswis.model.Agent}{}\vskip .1in 
\subsection{Declaration}{
\begin{lstlisting}[frame=none]
public class Agent
 extends java.lang.Object\end{lstlisting}
\subsection{Constructors}{
\vskip -2em
\begin{itemize}
\item{ 
\index{Agent(int, int, Group, Strategy)}
\hypertarget{de.sswis.model.Agent(int, int, de.sswis.model.Group, de.sswis.model.Strategy)}{{\bf  Agent}\\}
\begin{lstlisting}[frame=none]
public Agent(int id,int initialScore,Group group,Strategy initialStrategy)\end{lstlisting} %end signature
}%end item
\end{itemize}
}
}
\section{\label{de.sswis.model.CombinedStrategy}Class CombinedStrategy}{
\hypertarget{de.sswis.model.CombinedStrategy}{}\vskip .1in 
\subsection{Declaration}{
\begin{lstlisting}[frame=none]
public class CombinedStrategy
 extends java.lang.Object\end{lstlisting}
\subsection{Constructors}{
\vskip -2em
\begin{itemize}
\item{ 
\index{CombinedStrategy(String, String, BaseStrategy\lbrack \rbrack , Condition\lbrack \rbrack )}
\hypertarget{de.sswis.model.CombinedStrategy(java.lang.String, java.lang.String, de.sswis.model.strategies.BaseStrategy[], de.sswis.model.conditions.Condition[])}{{\bf  CombinedStrategy}\\}
\begin{lstlisting}[frame=none]
public CombinedStrategy(java.lang.String name,java.lang.String description,strategies.BaseStrategy[] strategies,conditions.Condition[] conditions)\end{lstlisting} %end signature
}%end item
\end{itemize}
}
\subsection{Methods}{
\vskip -2em
\begin{itemize}
\item{ 
\index{calculateAction(Agent, Agent)}
\hypertarget{de.sswis.model.CombinedStrategy.calculateAction(de.sswis.model.Agent, de.sswis.model.Agent)}{{\bf  calculateAction}\\}
\begin{lstlisting}[frame=none]
public Action calculateAction(Agent agent1,Agent agent2)\end{lstlisting} %end signature
}%end item
\end{itemize}
}
}
\section{\label{de.sswis.model.Configuration}Class Configuration}{
\hypertarget{de.sswis.model.Configuration}{}\vskip .1in 
\subsection{Declaration}{
\begin{lstlisting}[frame=none]
public class Configuration
 extends java.lang.Object\end{lstlisting}
\subsection{Constructors}{
\vskip -2em
\begin{itemize}
\item{ 
\index{Configuration(Game, AdaptationAlgorithm, PairingAlgorithm, RankingAlgorithm, int, int, int, double, List)}
\hypertarget{de.sswis.model.Configuration(de.sswis.model.Game, de.sswis.model.algorithms.adaptation.AdaptationAlgorithm, de.sswis.model.algorithms.pairing.PairingAlgorithm, de.sswis.model.algorithms.ranking.RankingAlgorithm, int, int, int, double, java.util.List)}{{\bf  Configuration}\\}
\begin{lstlisting}[frame=none]
public Configuration(Game game,algorithms.adaptation.AdaptationAlgorithm adaptation,algorithms.pairing.PairingAlgorithm pairing,algorithms.ranking.RankingAlgorithm rangking,int rounds,int cycles,int cycleRoundCount,double adaptationProbability,java.util.List strategies)\end{lstlisting} %end signature
}%end item
\end{itemize}
}
\subsection{Methods}{
\vskip -2em
\begin{itemize}
\item{ 
\index{addStrategy(Strategy)}
\hypertarget{de.sswis.model.Configuration.addStrategy(de.sswis.model.Strategy)}{{\bf  addStrategy}\\}
\begin{lstlisting}[frame=none]
public void addStrategy(Strategy newStrategy)\end{lstlisting} %end signature
}%end item
\item{ 
\index{getAdaptationAlg()}
\hypertarget{de.sswis.model.Configuration.getAdaptationAlg()}{{\bf  getAdaptationAlg}\\}
\begin{lstlisting}[frame=none]
public algorithms.adaptation.AdaptationAlgorithm getAdaptationAlg()\end{lstlisting} %end signature
}%end item
\item{ 
\index{getAdaptationProbability()}
\hypertarget{de.sswis.model.Configuration.getAdaptationProbability()}{{\bf  getAdaptationProbability}\\}
\begin{lstlisting}[frame=none]
public double getAdaptationProbability()\end{lstlisting} %end signature
}%end item
\item{ 
\index{getCycleRoundCount()}
\hypertarget{de.sswis.model.Configuration.getCycleRoundCount()}{{\bf  getCycleRoundCount}\\}
\begin{lstlisting}[frame=none]
public int getCycleRoundCount()\end{lstlisting} %end signature
}%end item
\item{ 
\index{getCycles()}
\hypertarget{de.sswis.model.Configuration.getCycles()}{{\bf  getCycles}\\}
\begin{lstlisting}[frame=none]
public int getCycles()\end{lstlisting} %end signature
}%end item
\item{ 
\index{getGame()}
\hypertarget{de.sswis.model.Configuration.getGame()}{{\bf  getGame}\\}
\begin{lstlisting}[frame=none]
public Game getGame()\end{lstlisting} %end signature
}%end item
\item{ 
\index{getInit()}
\hypertarget{de.sswis.model.Configuration.getInit()}{{\bf  getInit}\\}
\begin{lstlisting}[frame=none]
public Initialization getInit()\end{lstlisting} %end signature
}%end item
\item{ 
\index{getPairingAlg()}
\hypertarget{de.sswis.model.Configuration.getPairingAlg()}{{\bf  getPairingAlg}\\}
\begin{lstlisting}[frame=none]
public algorithms.pairing.PairingAlgorithm getPairingAlg()\end{lstlisting} %end signature
}%end item
\item{ 
\index{getPossibleStrategies()}
\hypertarget{de.sswis.model.Configuration.getPossibleStrategies()}{{\bf  getPossibleStrategies}\\}
\begin{lstlisting}[frame=none]
public java.util.List getPossibleStrategies()\end{lstlisting} %end signature
}%end item
\item{ 
\index{getRounds()}
\hypertarget{de.sswis.model.Configuration.getRounds()}{{\bf  getRounds}\\}
\begin{lstlisting}[frame=none]
public int getRounds()\end{lstlisting} %end signature
}%end item
\item{ 
\index{simulate(int)}
\hypertarget{de.sswis.model.Configuration.simulate(int)}{{\bf  simulate}\\}
\begin{lstlisting}[frame=none]
public Simulation simulate(int repetitions)\end{lstlisting} %end signature
}%end item
\end{itemize}
}
}
\section{\label{de.sswis.model.Game}Class Game}{
\hypertarget{de.sswis.model.Game}{}\vskip .1in 
\subsection{Declaration}{
\begin{lstlisting}[frame=none]
public class Game
 extends java.lang.Object\end{lstlisting}
\subsection{Constructors}{
\vskip -2em
\begin{itemize}
\item{ 
\index{Game(String, String, Game.Tuple\lbrack \rbrack \lbrack \rbrack )}
\hypertarget{de.sswis.model.Game(java.lang.String, java.lang.String, de.sswis.model.Game.Tuple[][])}{{\bf  Game}\\}
\begin{lstlisting}[frame=none]
public Game(java.lang.String name,java.lang.String description,Game.Tuple[][] payoffs)\end{lstlisting} %end signature
}%end item
\end{itemize}
}
\subsection{Methods}{
\vskip -2em
\begin{itemize}
\item{ 
\index{getPayoffs(Action, Action)}
\hypertarget{de.sswis.model.Game.getPayoffs(de.sswis.model.Action, de.sswis.model.Action)}{{\bf  getPayoffs}\\}
\begin{lstlisting}[frame=none]
public Game.Tuple getPayoffs(Action a1,Action a2)\end{lstlisting} %end signature
}%end item
\end{itemize}
}
}
\section{\label{de.sswis.model.Game.Tuple}Class Game.Tuple}{
\hypertarget{de.sswis.model.Game.Tuple}{}\vskip .1in 
\subsection{Declaration}{
\begin{lstlisting}[frame=none]
public class Game.Tuple
 extends java.lang.Object\end{lstlisting}
\subsection{Fields}{
\begin{itemize}
\item{
\index{x}
\label{de.sswis.model.Game.Tuple.x}\hypertarget{de.sswis.model.Game.Tuple.x}{\texttt{public int\ {\bf  x}}
}
}
\item{
\index{y}
\label{de.sswis.model.Game.Tuple.y}\hypertarget{de.sswis.model.Game.Tuple.y}{\texttt{public int\ {\bf  y}}
}
}
\end{itemize}
}
\subsection{Constructors}{
\vskip -2em
\begin{itemize}
\item{ 
\index{Tuple(int, int)}
\hypertarget{de.sswis.model.Game.Tuple(int, int)}{{\bf  Tuple}\\}
\begin{lstlisting}[frame=none]
public Tuple(int x,int y)\end{lstlisting} %end signature
}%end item
\end{itemize}
}
}
\section{\label{de.sswis.model.Group}Class Group}{
\hypertarget{de.sswis.model.Group}{}\vskip .1in 
\subsection{Declaration}{
\begin{lstlisting}[frame=none]
public class Group
 extends java.lang.Object\end{lstlisting}
\subsection{Constructors}{
\vskip -2em
\begin{itemize}
\item{ 
\index{Group(int, String)}
\hypertarget{de.sswis.model.Group(int, java.lang.String)}{{\bf  Group}\\}
\begin{lstlisting}[frame=none]
public Group(int id,java.lang.String name)\end{lstlisting} %end signature
}%end item
\end{itemize}
}
\subsection{Methods}{
\vskip -2em
\begin{itemize}
\item{ 
\index{addMember(Agent)}
\hypertarget{de.sswis.model.Group.addMember(de.sswis.model.Agent)}{{\bf  addMember}\\}
\begin{lstlisting}[frame=none]
public void addMember(Agent newMember)\end{lstlisting} %end signature
}%end item
\end{itemize}
}
}
\section{\label{de.sswis.model.History}Class History}{
\hypertarget{de.sswis.model.History}{}\vskip .1in 
\subsection{Declaration}{
\begin{lstlisting}[frame=none]
public class History
 extends java.lang.Object\end{lstlisting}
\subsection{Constructors}{
\vskip -2em
\begin{itemize}
\item{ 
\index{History(int)}
\hypertarget{de.sswis.model.History(int)}{{\bf  History}\\}
\begin{lstlisting}[frame=none]
public History(int rounds)\end{lstlisting} %end signature
}%end item
\end{itemize}
}
\subsection{Methods}{
\vskip -2em
\begin{itemize}
\item{ 
\index{cooperatedEveryTime(Agent)}
\hypertarget{de.sswis.model.History.cooperatedEveryTime(de.sswis.model.Agent)}{{\bf  cooperatedEveryTime}\\}
\begin{lstlisting}[frame=none]
public boolean cooperatedEveryTime(Agent agent)\end{lstlisting} %end signature
}%end item
\item{ 
\index{cooperatedLastTime(Agent)}
\hypertarget{de.sswis.model.History.cooperatedLastTime(de.sswis.model.Agent)}{{\bf  cooperatedLastTime}\\}
\begin{lstlisting}[frame=none]
public boolean cooperatedLastTime(Agent agent)\end{lstlisting} %end signature
}%end item
\item{ 
\index{groupCooperatedEveryTime(Agent)}
\hypertarget{de.sswis.model.History.groupCooperatedEveryTime(de.sswis.model.Agent)}{{\bf  groupCooperatedEveryTime}\\}
\begin{lstlisting}[frame=none]
public boolean groupCooperatedEveryTime(Agent agent)\end{lstlisting} %end signature
}%end item
\item{ 
\index{groupCooperatedLastTime(Agent)}
\hypertarget{de.sswis.model.History.groupCooperatedLastTime(de.sswis.model.Agent)}{{\bf  groupCooperatedLastTime}\\}
\begin{lstlisting}[frame=none]
public boolean groupCooperatedLastTime(Agent agent)\end{lstlisting} %end signature
}%end item
\end{itemize}
}
}
\section{\label{de.sswis.model.Initialization}Class Initialization}{
\hypertarget{de.sswis.model.Initialization}{}\vskip .1in 
\subsection{Declaration}{
\begin{lstlisting}[frame=none]
public class Initialization
 extends java.lang.Object\end{lstlisting}
\subsection{Constructors}{
\vskip -2em
\begin{itemize}
\item{ 
\index{Initialization(Group\lbrack \rbrack , int, Agent\lbrack \rbrack )}
\hypertarget{de.sswis.model.Initialization(de.sswis.model.Group[], int, de.sswis.model.Agent[])}{{\bf  Initialization}\\}
\begin{lstlisting}[frame=none]
public Initialization(Group[] groups,int agentCount,Agent[] agents)\end{lstlisting} %end signature
}%end item
\end{itemize}
}
\subsection{Methods}{
\vskip -2em
\begin{itemize}
\item{ 
\index{getAgentCount()}
\hypertarget{de.sswis.model.Initialization.getAgentCount()}{{\bf  getAgentCount}\\}
\begin{lstlisting}[frame=none]
public int getAgentCount()\end{lstlisting} %end signature
}%end item
\item{ 
\index{getAgents()}
\hypertarget{de.sswis.model.Initialization.getAgents()}{{\bf  getAgents}\\}
\begin{lstlisting}[frame=none]
public Agent[] getAgents()\end{lstlisting} %end signature
}%end item
\item{ 
\index{getGroups()}
\hypertarget{de.sswis.model.Initialization.getGroups()}{{\bf  getGroups}\\}
\begin{lstlisting}[frame=none]
public Group[] getGroups()\end{lstlisting} %end signature
}%end item
\end{itemize}
}
}
\section{\label{de.sswis.model.Pair}Class Pair}{
\hypertarget{de.sswis.model.Pair}{}\vskip .1in 
\subsection{Declaration}{
\begin{lstlisting}[frame=none]
public class Pair
 extends java.lang.Object\end{lstlisting}
\subsection{Constructors}{
\vskip -2em
\begin{itemize}
\item{ 
\index{Pair(Agent, Agent)}
\hypertarget{de.sswis.model.Pair(de.sswis.model.Agent, de.sswis.model.Agent)}{{\bf  Pair}\\}
\begin{lstlisting}[frame=none]
public Pair(Agent agent1,Agent agent2)\end{lstlisting} %end signature
}%end item
\end{itemize}
}
\subsection{Methods}{
\vskip -2em
\begin{itemize}
\item{ 
\index{playGame()}
\hypertarget{de.sswis.model.Pair.playGame()}{{\bf  playGame}\\}
\begin{lstlisting}[frame=none]
public void playGame()\end{lstlisting} %end signature
}%end item
\end{itemize}
}
}
\section{\label{de.sswis.model.Simulation}Class Simulation}{
\hypertarget{de.sswis.model.Simulation}{}\vskip .1in 
\subsection{Declaration}{
\begin{lstlisting}[frame=none]
public class Simulation
 extends java.util.Observable implements java.lang.Runnable\end{lstlisting}
\subsection{Constructors}{
\vskip -2em
\begin{itemize}
\item{ 
\index{Simulation(Configuration)}
\hypertarget{de.sswis.model.Simulation(de.sswis.model.Configuration)}{{\bf  Simulation}\\}
\begin{lstlisting}[frame=none]
public Simulation(Configuration config)\end{lstlisting} %end signature
}%end item
\end{itemize}
}
\subsection{Methods}{
\vskip -2em
\begin{itemize}
\item{ 
\index{abort()}
\hypertarget{de.sswis.model.Simulation.abort()}{{\bf  abort}\\}
\begin{lstlisting}[frame=none]
public void abort()\end{lstlisting} %end signature
}%end item
\item{ 
\index{EquilibriumAchieved()}
\hypertarget{de.sswis.model.Simulation.EquilibriumAchieved()}{{\bf  EquilibriumAchieved}\\}
\begin{lstlisting}[frame=none]
public boolean EquilibriumAchieved()\end{lstlisting} %end signature
}%end item
\item{ 
\index{getCurrentPairs()}
\hypertarget{de.sswis.model.Simulation.getCurrentPairs()}{{\bf  getCurrentPairs}\\}
\begin{lstlisting}[frame=none]
public Pair[] getCurrentPairs()\end{lstlisting} %end signature
}%end item
\item{ 
\index{getCurrentRanking()}
\hypertarget{de.sswis.model.Simulation.getCurrentRanking()}{{\bf  getCurrentRanking}\\}
\begin{lstlisting}[frame=none]
public java.util.HashMap getCurrentRanking()\end{lstlisting} %end signature
}%end item
\item{ 
\index{getResults()}
\hypertarget{de.sswis.model.Simulation.getResults()}{{\bf  getResults}\\}
\begin{lstlisting}[frame=none]
public Agent[] getResults()\end{lstlisting} %end signature
}%end item
\item{ 
\index{restart()}
\hypertarget{de.sswis.model.Simulation.restart()}{{\bf  restart}\\}
\begin{lstlisting}[frame=none]
public void restart()\end{lstlisting} %end signature
}%end item
\item{ 
\index{run()}
\hypertarget{de.sswis.model.Simulation.run()}{{\bf  run}\\}
\begin{lstlisting}[frame=none]
void run()\end{lstlisting} %end signature
}%end item
\item{ 
\index{simulate()}
\hypertarget{de.sswis.model.Simulation.simulate()}{{\bf  simulate}\\}
\begin{lstlisting}[frame=none]
public void simulate()\end{lstlisting} %end signature
}%end item
\end{itemize}
}
\subsection{Members inherited from class Observable }{
\texttt{java.util.Observable} {\small 
\refdefined{java.util.Observable}}
{\small 

addObserver, clearChanged, countObservers, deleteObserver, deleteObservers, hasChanged, notifyObservers, notifyObservers, setChanged}
}
\section{\label{de.sswis.model.Strategy}Class Strategy}{
\hypertarget{de.sswis.model.Strategy}{}\vskip .1in 
\subsection{Declaration}{
\begin{lstlisting}[frame=none]
public class Strategy
 extends java.lang.Object\end{lstlisting}
\subsection{Constructors}{
\vskip -2em
\begin{itemize}
\item{ 
\index{Strategy(String, String, CombinedStrategy\lbrack \rbrack , double\lbrack \rbrack )}
\hypertarget{de.sswis.model.Strategy(java.lang.String, java.lang.String, de.sswis.model.CombinedStrategy[], double[])}{{\bf  Strategy}\\}
\begin{lstlisting}[frame=none]
public Strategy(java.lang.String name,java.lang.String description,CombinedStrategy[] combinedStrategies,double[] probabilities)\end{lstlisting} %end signature
}%end item
\end{itemize}
}
\subsection{Methods}{
\vskip -2em
\begin{itemize}
\item{ 
\index{calculateAction(Agent, Agent)}
\hypertarget{de.sswis.model.Strategy.calculateAction(de.sswis.model.Agent, de.sswis.model.Agent)}{{\bf  calculateAction}\\}
\begin{lstlisting}[frame=none]
public Action calculateAction(Agent a1,Agent a2)\end{lstlisting} %end signature
}%end item
\end{itemize}
}
}
}
\chapter{Package de.sswis.model.algorithms.adaptation}{
\label{de.sswis.model.algorithms.adaptation}\hypertarget{de.sswis.model.algorithms.adaptation}{}
\hskip -.05in
\hbox to \hsize{\textit{ Package Contents\hfil Page}}
\vskip .13in
\hbox{{\bf  Interfaces}}
\entityintro{AdaptationAlgorithm}{de.sswis.model.algorithms.adaptation.AdaptationAlgorithm}{Ein Algorithmus zum Anpassen der Strategien von Agenten einer Simulation.}
\vskip .13in
\hbox{{\bf  Classes}}
\entityintro{MixedLinearInterpolation}{de.sswis.model.algorithms.adaptation.MixedLinearInterpolation}{Ein Algorithmus der die Wahrscheinlichkeiten der gemischten Strategien der Agenten einer Simulation durch lineare Interpolation anpasst.}
\entityintro{MixedSum}{de.sswis.model.algorithms.adaptation.MixedSum}{Ein Algorithmus der die Wahrscheinlichkeiten der gemischten Strategien der Agenten einer Simulation durch Summenbildung und anschließende Normierung anpasst.}
\entityintro{RandomAdaptation}{de.sswis.model.algorithms.adaptation.RandomAdaptation}{Ein Algorithmus der die Strategien der Agenten einer Simulation mit einer gewissen Wahrscheinlichkeit zu einer zufälligen neuen Strategie anpasst.}
\entityintro{RankPercentage}{de.sswis.model.algorithms.adaptation.RankPercentage}{Ein Algorithmus der die Strategie eines Agenten anpasst, wenn der Agent mit dem verglichen wird zu den obersten \texttt{\small PERCENTAGE} Prozent der Rangliste gehört.}
\entityintro{ReplicatorDynamicRank}{de.sswis.model.algorithms.adaptation.ReplicatorDynamicRank}{Ein Algorithmus der die Strategie eines Agenten einer Simulation anpasst, in Abhängigkeit von der Differenz der Ränge zweier verglichener Agenten.}
\entityintro{ReplicatorDynamicScore}{de.sswis.model.algorithms.adaptation.ReplicatorDynamicScore}{Ein Algorithmus der die Strategie eines Agenten einer Simulation anpasst, in Abhängigkeit von der Differenz der Gesamtpunktzahlen zweier verglichener Agenten.}
\vskip .1in
\vskip .1in
\section{\label{de.sswis.model.algorithms.adaptation.AdaptationAlgorithm}Interface AdaptationAlgorithm}{
\hypertarget{de.sswis.model.algorithms.adaptation.AdaptationAlgorithm}{}\vskip .1in 
Ein Algorithmus zum Anpassen der Strategien von Agenten einer Simulation. Jeder Agent A wird mit einem zufälligen Agenten verglichen, ist letzterer erfolgreicher, so wird die Strategie von A durch den Algorithmus angepasst.\vskip .1in 
\subsection{Declaration}{
\begin{lstlisting}[frame=none]
public interface AdaptationAlgorithm
\end{lstlisting}
\subsection{All known subinterfaces}{ReplicatorDynamicScore\small{\refdefined{de.sswis.model.algorithms.adaptation.ReplicatorDynamicScore}}, ReplicatorDynamicRank\small{\refdefined{de.sswis.model.algorithms.adaptation.ReplicatorDynamicRank}}, RankPercentage\small{\refdefined{de.sswis.model.algorithms.adaptation.RankPercentage}}, RandomAdaptation\small{\refdefined{de.sswis.model.algorithms.adaptation.RandomAdaptation}}, MixedSum\small{\refdefined{de.sswis.model.algorithms.adaptation.MixedSum}}, MixedLinearInterpolation\small{\refdefined{de.sswis.model.algorithms.adaptation.MixedLinearInterpolation}}}
\subsection{All classes known to implement interface}{ReplicatorDynamicScore\small{\refdefined{de.sswis.model.algorithms.adaptation.ReplicatorDynamicScore}}, ReplicatorDynamicRank\small{\refdefined{de.sswis.model.algorithms.adaptation.ReplicatorDynamicRank}}, RankPercentage\small{\refdefined{de.sswis.model.algorithms.adaptation.RankPercentage}}, RandomAdaptation\small{\refdefined{de.sswis.model.algorithms.adaptation.RandomAdaptation}}, MixedSum\small{\refdefined{de.sswis.model.algorithms.adaptation.MixedSum}}, MixedLinearInterpolation\small{\refdefined{de.sswis.model.algorithms.adaptation.MixedLinearInterpolation}}}
\subsection{Methods}{
\vskip -2em
\begin{itemize}
\item{ 
\index{adapt(Simulation)}
\hypertarget{de.sswis.model.algorithms.adaptation.AdaptationAlgorithm.adapt(de.sswis.model.Simulation)}{{\bf  adapt}\\}
\begin{lstlisting}[frame=none]
void adapt(de.sswis.model.Simulation sim)\end{lstlisting} %end signature
\begin{itemize}
\item{
{\bf  Description}

Passt die Strategien der Agenten entsprechend des Algorithmus an.
}
\item{
{\bf  Parameters}
  \begin{itemize}
   \item{
\texttt{sim} -- die Simulation in der die Strategien der Ageten angepasst werden sollen}
  \end{itemize}
}%end item
\end{itemize}
}%end item
\end{itemize}
}
}
\section{\label{de.sswis.model.algorithms.adaptation.MixedLinearInterpolation}Class MixedLinearInterpolation}{
\hypertarget{de.sswis.model.algorithms.adaptation.MixedLinearInterpolation}{}\vskip .1in 
Ein Algorithmus der die Wahrscheinlichkeiten der gemischten Strategien der Agenten einer Simulation durch lineare Interpolation anpasst. Die Wahrscheinlichkeiten werden stärker angepasst desto größer die Differenz der Gesamtpunktzahlen der verglichenen Agenten ist.\vskip .1in 
\subsection{Declaration}{
\begin{lstlisting}[frame=none]
public class MixedLinearInterpolation
 extends java.lang.Object implements AdaptationAlgorithm\end{lstlisting}
\subsection{Fields}{
\begin{itemize}
\item{
\index{NAME}
\label{de.sswis.model.algorithms.adaptation.MixedLinearInterpolation.NAME}\hypertarget{de.sswis.model.algorithms.adaptation.MixedLinearInterpolation.NAME}{\texttt{public static final java.lang.String\ {\bf  NAME}}
}
}
\item{
\index{DESCRIPTION}
\label{de.sswis.model.algorithms.adaptation.MixedLinearInterpolation.DESCRIPTION}\hypertarget{de.sswis.model.algorithms.adaptation.MixedLinearInterpolation.DESCRIPTION}{\texttt{public static final java.lang.String\ {\bf  DESCRIPTION}}
}
}
\end{itemize}
}
\subsection{Constructors}{
\vskip -2em
\begin{itemize}
\item{ 
\index{MixedLinearInterpolation()}
\hypertarget{de.sswis.model.algorithms.adaptation.MixedLinearInterpolation()}{{\bf  MixedLinearInterpolation}\\}
\begin{lstlisting}[frame=none]
public MixedLinearInterpolation()\end{lstlisting} %end signature
}%end item
\end{itemize}
}
\subsection{Methods}{
\vskip -2em
\begin{itemize}
\item{ 
\index{adapt(Simulation)}
\hypertarget{de.sswis.model.algorithms.adaptation.MixedLinearInterpolation.adapt(de.sswis.model.Simulation)}{{\bf  adapt}\\}
\begin{lstlisting}[frame=none]
void adapt(de.sswis.model.Simulation sim)\end{lstlisting} %end signature
\begin{itemize}
\item{
{\bf  Description copied from \hyperlink{de.sswis.model.algorithms.adaptation.AdaptationAlgorithm}{AdaptationAlgorithm}{\small \refdefined{de.sswis.model.algorithms.adaptation.AdaptationAlgorithm}} }

Passt die Strategien der Agenten entsprechend des Algorithmus an.
}
\item{
{\bf  Parameters}
  \begin{itemize}
   \item{
\texttt{sim} -- die Simulation in der die Strategien der Ageten angepasst werden sollen}
  \end{itemize}
}%end item
\end{itemize}
}%end item
\end{itemize}
}
}
\section{\label{de.sswis.model.algorithms.adaptation.MixedSum}Class MixedSum}{
\hypertarget{de.sswis.model.algorithms.adaptation.MixedSum}{}\vskip .1in 
Ein Algorithmus der die Wahrscheinlichkeiten der gemischten Strategien der Agenten einer Simulation durch Summenbildung und anschließende Normierung anpasst. Die Wahrscheinlichkeiten der beiden gemischten Strategien werden addiert und anschließend normiert, so dass die Summe der neuen Wahrscheinlichkeiten wieder 1 ergibt.\vskip .1in 
\subsection{Declaration}{
\begin{lstlisting}[frame=none]
public class MixedSum
 extends java.lang.Object implements AdaptationAlgorithm\end{lstlisting}
\subsection{Fields}{
\begin{itemize}
\item{
\index{NAME}
\label{de.sswis.model.algorithms.adaptation.MixedSum.NAME}\hypertarget{de.sswis.model.algorithms.adaptation.MixedSum.NAME}{\texttt{public static final java.lang.String\ {\bf  NAME}}
}
}
\item{
\index{DESCRIPTION}
\label{de.sswis.model.algorithms.adaptation.MixedSum.DESCRIPTION}\hypertarget{de.sswis.model.algorithms.adaptation.MixedSum.DESCRIPTION}{\texttt{public static final java.lang.String\ {\bf  DESCRIPTION}}
}
}
\end{itemize}
}
\subsection{Constructors}{
\vskip -2em
\begin{itemize}
\item{ 
\index{MixedSum()}
\hypertarget{de.sswis.model.algorithms.adaptation.MixedSum()}{{\bf  MixedSum}\\}
\begin{lstlisting}[frame=none]
public MixedSum()\end{lstlisting} %end signature
}%end item
\end{itemize}
}
\subsection{Methods}{
\vskip -2em
\begin{itemize}
\item{ 
\index{adapt(Simulation)}
\hypertarget{de.sswis.model.algorithms.adaptation.MixedSum.adapt(de.sswis.model.Simulation)}{{\bf  adapt}\\}
\begin{lstlisting}[frame=none]
void adapt(de.sswis.model.Simulation sim)\end{lstlisting} %end signature
\begin{itemize}
\item{
{\bf  Description copied from \hyperlink{de.sswis.model.algorithms.adaptation.AdaptationAlgorithm}{AdaptationAlgorithm}{\small \refdefined{de.sswis.model.algorithms.adaptation.AdaptationAlgorithm}} }

Passt die Strategien der Agenten entsprechend des Algorithmus an.
}
\item{
{\bf  Parameters}
  \begin{itemize}
   \item{
\texttt{sim} -- die Simulation in der die Strategien der Ageten angepasst werden sollen}
  \end{itemize}
}%end item
\end{itemize}
}%end item
\end{itemize}
}
}
\section{\label{de.sswis.model.algorithms.adaptation.RandomAdaptation}Class RandomAdaptation}{
\hypertarget{de.sswis.model.algorithms.adaptation.RandomAdaptation}{}\vskip .1in 
Ein Algorithmus der die Strategien der Agenten einer Simulation mit einer gewissen Wahrscheinlichkeit zu einer zufälligen neuen Strategie anpasst. Die neue Strategie muss in der Konfiguration enthalten sein. Hat der Agent eine gemischte Strategie, so wird die Wahrscheinlichkeit der neuen zufälligen Strategie erhöht und anschließend alle Wahrscheinlichkeiten normiert, so dass ihre Summe wieder 1 ergibt.\vskip .1in 
\subsection{Declaration}{
\begin{lstlisting}[frame=none]
public class RandomAdaptation
 extends java.lang.Object implements AdaptationAlgorithm\end{lstlisting}
\subsection{Fields}{
\begin{itemize}
\item{
\index{NAME}
\label{de.sswis.model.algorithms.adaptation.RandomAdaptation.NAME}\hypertarget{de.sswis.model.algorithms.adaptation.RandomAdaptation.NAME}{\texttt{public static final java.lang.String\ {\bf  NAME}}
}
}
\item{
\index{DESCRIPTION}
\label{de.sswis.model.algorithms.adaptation.RandomAdaptation.DESCRIPTION}\hypertarget{de.sswis.model.algorithms.adaptation.RandomAdaptation.DESCRIPTION}{\texttt{public static final java.lang.String\ {\bf  DESCRIPTION}}
}
}
\end{itemize}
}
\subsection{Constructors}{
\vskip -2em
\begin{itemize}
\item{ 
\index{RandomAdaptation(int)}
\hypertarget{de.sswis.model.algorithms.adaptation.RandomAdaptation(int)}{{\bf  RandomAdaptation}\\}
\begin{lstlisting}[frame=none]
public RandomAdaptation(int PROBABILITY)\end{lstlisting} %end signature
\begin{itemize}
\item{
{\bf  Description}

Konstruktor
}
\item{
{\bf  Parameters}
  \begin{itemize}
   \item{
\texttt{PROBABILITY} -- Wahrscheinlichkeit mit der eine neue Strategie gewählt wird}
  \end{itemize}
}%end item
\end{itemize}
}%end item
\end{itemize}
}
\subsection{Methods}{
\vskip -2em
\begin{itemize}
\item{ 
\index{adapt(Simulation)}
\hypertarget{de.sswis.model.algorithms.adaptation.RandomAdaptation.adapt(de.sswis.model.Simulation)}{{\bf  adapt}\\}
\begin{lstlisting}[frame=none]
void adapt(de.sswis.model.Simulation sim)\end{lstlisting} %end signature
\begin{itemize}
\item{
{\bf  Description copied from \hyperlink{de.sswis.model.algorithms.adaptation.AdaptationAlgorithm}{AdaptationAlgorithm}{\small \refdefined{de.sswis.model.algorithms.adaptation.AdaptationAlgorithm}} }

Passt die Strategien der Agenten entsprechend des Algorithmus an.
}
\item{
{\bf  Parameters}
  \begin{itemize}
   \item{
\texttt{sim} -- die Simulation in der die Strategien der Ageten angepasst werden sollen}
  \end{itemize}
}%end item
\end{itemize}
}%end item
\end{itemize}
}
}
\section{\label{de.sswis.model.algorithms.adaptation.RankPercentage}Class RankPercentage}{
\hypertarget{de.sswis.model.algorithms.adaptation.RankPercentage}{}\vskip .1in 
Ein Algorithmus der die Strategie eines Agenten anpasst, wenn der Agent mit dem verglichen wird zu den obersten \texttt{\small PERCENTAGE} Prozent der Rangliste gehört. Der Agent übernimmt die Strategie des anderen, falls diese Kondition erfüllt ist. Der Rang eines Agenten wird nicht angepasst, wenn der Rang des Agenten mit dem verglichen wird tiefer ist.\vskip .1in 
\subsection{Declaration}{
\begin{lstlisting}[frame=none]
public class RankPercentage
 extends java.lang.Object implements AdaptationAlgorithm\end{lstlisting}
\subsection{Fields}{
\begin{itemize}
\item{
\index{NAME}
\label{de.sswis.model.algorithms.adaptation.RankPercentage.NAME}\hypertarget{de.sswis.model.algorithms.adaptation.RankPercentage.NAME}{\texttt{public static final java.lang.String\ {\bf  NAME}}
}
}
\item{
\index{DESCRIPTION}
\label{de.sswis.model.algorithms.adaptation.RankPercentage.DESCRIPTION}\hypertarget{de.sswis.model.algorithms.adaptation.RankPercentage.DESCRIPTION}{\texttt{public static final java.lang.String\ {\bf  DESCRIPTION}}
}
}
\end{itemize}
}
\subsection{Constructors}{
\vskip -2em
\begin{itemize}
\item{ 
\index{RankPercentage(int)}
\hypertarget{de.sswis.model.algorithms.adaptation.RankPercentage(int)}{{\bf  RankPercentage}\\}
\begin{lstlisting}[frame=none]
public RankPercentage(int PERCENTAGE)\end{lstlisting} %end signature
\begin{itemize}
\item{
{\bf  Description}

Konstruktor
}
\item{
{\bf  Parameters}
  \begin{itemize}
   \item{
\texttt{PERCENTAGE} -- Prozentsatz der angibt von welchen Agenten Stragegien übernommen werden}
  \end{itemize}
}%end item
\end{itemize}
}%end item
\end{itemize}
}
\subsection{Methods}{
\vskip -2em
\begin{itemize}
\item{ 
\index{adapt(Simulation)}
\hypertarget{de.sswis.model.algorithms.adaptation.RankPercentage.adapt(de.sswis.model.Simulation)}{{\bf  adapt}\\}
\begin{lstlisting}[frame=none]
void adapt(de.sswis.model.Simulation sim)\end{lstlisting} %end signature
\begin{itemize}
\item{
{\bf  Description copied from \hyperlink{de.sswis.model.algorithms.adaptation.AdaptationAlgorithm}{AdaptationAlgorithm}{\small \refdefined{de.sswis.model.algorithms.adaptation.AdaptationAlgorithm}} }

Passt die Strategien der Agenten entsprechend des Algorithmus an.
}
\item{
{\bf  Parameters}
  \begin{itemize}
   \item{
\texttt{sim} -- die Simulation in der die Strategien der Ageten angepasst werden sollen}
  \end{itemize}
}%end item
\end{itemize}
}%end item
\end{itemize}
}
}
\section{\label{de.sswis.model.algorithms.adaptation.ReplicatorDynamicRank}Class ReplicatorDynamicRank}{
\hypertarget{de.sswis.model.algorithms.adaptation.ReplicatorDynamicRank}{}\vskip .1in 
Ein Algorithmus der die Strategie eines Agenten einer Simulation anpasst, in Abhängigkeit von der Differenz der Ränge zweier verglichener Agenten. Ein Agent übernimmt die Strategie eines anderen mit einer Wahrscheinlichkeitlt.\vskip .1in 
\subsection{Declaration}{
\begin{lstlisting}[frame=none]
public class ReplicatorDynamicRank
 extends java.lang.Object implements AdaptationAlgorithm\end{lstlisting}
\subsection{Fields}{
\begin{itemize}
\item{
\index{NAME}
\label{de.sswis.model.algorithms.adaptation.ReplicatorDynamicRank.NAME}\hypertarget{de.sswis.model.algorithms.adaptation.ReplicatorDynamicRank.NAME}{\texttt{public static final java.lang.String\ {\bf  NAME}}
}
}
\item{
\index{DESCRIPTION}
\label{de.sswis.model.algorithms.adaptation.ReplicatorDynamicRank.DESCRIPTION}\hypertarget{de.sswis.model.algorithms.adaptation.ReplicatorDynamicRank.DESCRIPTION}{\texttt{public static final java.lang.String\ {\bf  DESCRIPTION}}
}
}
\end{itemize}
}
\subsection{Constructors}{
\vskip -2em
\begin{itemize}
\item{ 
\index{ReplicatorDynamicRank(double)}
\hypertarget{de.sswis.model.algorithms.adaptation.ReplicatorDynamicRank(double)}{{\bf  ReplicatorDynamicRank}\\}
\begin{lstlisting}[frame=none]
public ReplicatorDynamicRank(double BETA)\end{lstlisting} %end signature
\begin{itemize}
\item{
{\bf  Description}

Konstruktor
}
\item{
{\bf  Parameters}
  \begin{itemize}
   \item{
\texttt{BETA} -- Konstante gilt}
  \end{itemize}
}%end item
\end{itemize}
}%end item
\end{itemize}
}
\subsection{Methods}{
\vskip -2em
\begin{itemize}
\item{ 
\index{adapt(Simulation)}
\hypertarget{de.sswis.model.algorithms.adaptation.ReplicatorDynamicRank.adapt(de.sswis.model.Simulation)}{{\bf  adapt}\\}
\begin{lstlisting}[frame=none]
void adapt(de.sswis.model.Simulation sim)\end{lstlisting} %end signature
\begin{itemize}
\item{
{\bf  Description copied from \hyperlink{de.sswis.model.algorithms.adaptation.AdaptationAlgorithm}{AdaptationAlgorithm}{\small \refdefined{de.sswis.model.algorithms.adaptation.AdaptationAlgorithm}} }

Passt die Strategien der Agenten entsprechend des Algorithmus an.
}
\item{
{\bf  Parameters}
  \begin{itemize}
   \item{
\texttt{sim} -- die Simulation in der die Strategien der Ageten angepasst werden sollen}
  \end{itemize}
}%end item
\end{itemize}
}%end item
\end{itemize}
}
}
\section{\label{de.sswis.model.algorithms.adaptation.ReplicatorDynamicScore}Class ReplicatorDynamicScore}{
\hypertarget{de.sswis.model.algorithms.adaptation.ReplicatorDynamicScore}{}\vskip .1in 
Ein Algorithmus der die Strategie eines Agenten einer Simulation anpasst, in Abhängigkeit von der Differenz der Gesamtpunktzahlen zweier verglichener Agenten. Ein Agent übernimmt die Strategie eines anderen mit einer Wahrscheinlichkeit gilt.\vskip .1in 
\subsection{Declaration}{
\begin{lstlisting}[frame=none]
public class ReplicatorDynamicScore
 extends java.lang.Object implements AdaptationAlgorithm\end{lstlisting}
\subsection{Fields}{
\begin{itemize}
\item{
\index{NAME}
\label{de.sswis.model.algorithms.adaptation.ReplicatorDynamicScore.NAME}\hypertarget{de.sswis.model.algorithms.adaptation.ReplicatorDynamicScore.NAME}{\texttt{public static final java.lang.String\ {\bf  NAME}}
}
}
\item{
\index{DESCRIPTION}
\label{de.sswis.model.algorithms.adaptation.ReplicatorDynamicScore.DESCRIPTION}\hypertarget{de.sswis.model.algorithms.adaptation.ReplicatorDynamicScore.DESCRIPTION}{\texttt{public static final java.lang.String\ {\bf  DESCRIPTION}}
}
}
\end{itemize}
}
\subsection{Constructors}{
\vskip -2em
\begin{itemize}
\item{ 
\index{ReplicatorDynamicScore(double)}
\hypertarget{de.sswis.model.algorithms.adaptation.ReplicatorDynamicScore(double)}{{\bf  ReplicatorDynamicScore}\\}
\begin{lstlisting}[frame=none]
public ReplicatorDynamicScore(double BETA)\end{lstlisting} %end signature
\begin{itemize}
\item{
{\bf  Description}

Konstruktor
}
\item{
{\bf  Parameters}
  \begin{itemize}
   \item{
\texttt{BETA} -- Konstante  gilt}
  \end{itemize}
}%end item
\end{itemize}
}%end item
\end{itemize}
}
\subsection{Methods}{
\vskip -2em
\begin{itemize}
\item{ 
\index{adapt(Simulation)}
\hypertarget{de.sswis.model.algorithms.adaptation.ReplicatorDynamicScore.adapt(de.sswis.model.Simulation)}{{\bf  adapt}\\}
\begin{lstlisting}[frame=none]
void adapt(de.sswis.model.Simulation sim)\end{lstlisting} %end signature
\begin{itemize}
\item{
{\bf  Description copied from \hyperlink{de.sswis.model.algorithms.adaptation.AdaptationAlgorithm}{AdaptationAlgorithm}{\small \refdefined{de.sswis.model.algorithms.adaptation.AdaptationAlgorithm}} }

Passt die Strategien der Agenten entsprechend des Algorithmus an.
}
\item{
{\bf  Parameters}
  \begin{itemize}
   \item{
\texttt{sim} -- die Simulation in der die Strategien der Ageten angepasst werden sollen}
  \end{itemize}
}%end item
\end{itemize}
}%end item
\end{itemize}
}
}
}
\chapter{Package de.sswis.model.algorithms.pairing}{
\label{de.sswis.model.algorithms.pairing}\hypertarget{de.sswis.model.algorithms.pairing}{}
\hskip -.05in
\hbox to \hsize{\textit{ Package Contents\hfil Page}}
\vskip .13in
\hbox{{\bf  Interfaces}}
\entityintro{PairingAlgorithm}{de.sswis.model.algorithms.pairing.PairingAlgorithm}{Ein Algorithmus zum Bilden von Paaren von Agenten einer Simulation.}
\vskip .1in
\vskip .1in
\section{\label{de.sswis.model.algorithms.pairing.PairingAlgorithm}Interface PairingAlgorithm}{
\hypertarget{de.sswis.model.algorithms.pairing.PairingAlgorithm}{}\vskip .1in 
Ein Algorithmus zum Bilden von Paaren von Agenten einer Simulation.\vskip .1in 
\subsection{Declaration}{
\begin{lstlisting}[frame=none]
public interface PairingAlgorithm
\end{lstlisting}
\subsection{Methods}{
\vskip -2em
\begin{itemize}
\item{ 
\index{getPairing(Simulation)}
\hypertarget{de.sswis.model.algorithms.pairing.PairingAlgorithm.getPairing(de.sswis.model.Simulation)}{{\bf  getPairing}\\}
\begin{lstlisting}[frame=none]
de.sswis.model.Pair[] getPairing(de.sswis.model.Simulation sim)\end{lstlisting} %end signature
\begin{itemize}
\item{
{\bf  Description}

Paart die Agenten einer Simulation entsprechend des Algorithmus.
}
\item{
{\bf  Parameters}
  \begin{itemize}
   \item{
\texttt{sim} -- die Simulation deren Agenten gepaart werden sollen}
  \end{itemize}
}%end item
\item{{\bf  Returns} -- 
eine Menge von Agent-Paaren 
}%end item
\end{itemize}
}%end item
\end{itemize}
}
}
}
\chapter{Package de.sswis.model.algorithms.ranking}{
\label{de.sswis.model.algorithms.ranking}\hypertarget{de.sswis.model.algorithms.ranking}{}
\hskip -.05in
\hbox to \hsize{\textit{ Package Contents\hfil Page}}
\vskip .13in
\hbox{{\bf  Interfaces}}
\entityintro{RankingAlgorithm}{de.sswis.model.algorithms.ranking.RankingAlgorithm}{Ein Algorithmus zum Bewerten der Agenten einer Simulation.}
\vskip .13in
\hbox{{\bf  Classes}}
\entityintro{AverageRank}{de.sswis.model.algorithms.ranking.AverageRank}{Ein Algorithmus der die Agenten einer Simulation entsprechend ihres Durchschnittsranges über die letzten Zyklen bewertet.}
\entityintro{CurrentCycleScore}{de.sswis.model.algorithms.ranking.CurrentCycleScore}{Ein Algorithmus der die Agenten einer Simulation entsprechend ihrer Punktzahl im aktuellen Zyklus bewertet.}
\entityintro{CustomCycleScore}{de.sswis.model.algorithms.ranking.CustomCycleScore}{Ein Algorithmus der die Agenten einer Simulation entsprechend ihrer Punktzahl in den bisherigen \texttt{\small WINDOW\_SIZE} Zyklen bewertet.}
\entityintro{Score}{de.sswis.model.algorithms.ranking.Score}{Ein Algorithmus der die Agenten einer Simulation entsprechend ihrer bisherigen Gesamtpunktzahl bewertet.}
\vskip .1in
\vskip .1in
\section{\label{de.sswis.model.algorithms.ranking.RankingAlgorithm}Interface RankingAlgorithm}{
\hypertarget{de.sswis.model.algorithms.ranking.RankingAlgorithm}{}\vskip .1in 
Ein Algorithmus zum Bewerten der Agenten einer Simulation.\vskip .1in 
\subsection{Declaration}{
\begin{lstlisting}[frame=none]
public interface RankingAlgorithm
\end{lstlisting}
\subsection{All known subinterfaces}{Score\small{\refdefined{de.sswis.model.algorithms.ranking.Score}}, CustomCycleScore\small{\refdefined{de.sswis.model.algorithms.ranking.CustomCycleScore}}, CurrentCycleScore\small{\refdefined{de.sswis.model.algorithms.ranking.CurrentCycleScore}}, AverageRank\small{\refdefined{de.sswis.model.algorithms.ranking.AverageRank}}}
\subsection{All classes known to implement interface}{Score\small{\refdefined{de.sswis.model.algorithms.ranking.Score}}, CustomCycleScore\small{\refdefined{de.sswis.model.algorithms.ranking.CustomCycleScore}}, CurrentCycleScore\small{\refdefined{de.sswis.model.algorithms.ranking.CurrentCycleScore}}, AverageRank\small{\refdefined{de.sswis.model.algorithms.ranking.AverageRank}}}
\subsection{Methods}{
\vskip -2em
\begin{itemize}
\item{ 
\index{getRankings(Simulation)}
\hypertarget{de.sswis.model.algorithms.ranking.RankingAlgorithm.getRankings(de.sswis.model.Simulation)}{{\bf  getRankings}\\}
\begin{lstlisting}[frame=none]
java.util.HashMap getRankings(de.sswis.model.Simulation sim)\end{lstlisting} %end signature
\begin{itemize}
\item{
{\bf  Description}

Bewertet die Agenten entsprechend des Algorithmus.
}
\item{
{\bf  Parameters}
  \begin{itemize}
   \item{
\texttt{sim} -- die Simulation deren Agenten bewertet werden sollen}
  \end{itemize}
}%end item
\item{{\bf  Returns} -- 
eine \texttt{\small HashMap} die jedem Agenten seinen Rang zuordnet 
}%end item
\end{itemize}
}%end item
\end{itemize}
}
}
\section{\label{de.sswis.model.algorithms.ranking.AverageRank}Class AverageRank}{
\hypertarget{de.sswis.model.algorithms.ranking.AverageRank}{}\vskip .1in 
Ein Algorithmus der die Agenten einer Simulation entsprechend ihres Durchschnittsranges über die letzten Zyklen bewertet. Für jeden bisherigen Zyklus wird die Gesamtpunktzahl der vorherigen \texttt{\small WINDOW\_SIZE} Zyklen bestimmt und die Agenten erhalten entsprechend dieser Punktzahl einen Rang. Der Durchschnitt dieser Ränge über alle bisherigen Zyklen ist der finale Rang.\vskip .1in 
\subsection{Declaration}{
\begin{lstlisting}[frame=none]
public class AverageRank
 extends java.lang.Object implements RankingAlgorithm\end{lstlisting}
\subsection{Fields}{
\begin{itemize}
\item{
\index{NAME}
\label{de.sswis.model.algorithms.ranking.AverageRank.NAME}\hypertarget{de.sswis.model.algorithms.ranking.AverageRank.NAME}{\texttt{public static final java.lang.String\ {\bf  NAME}}
}
}
\item{
\index{DESCRIPTION}
\label{de.sswis.model.algorithms.ranking.AverageRank.DESCRIPTION}\hypertarget{de.sswis.model.algorithms.ranking.AverageRank.DESCRIPTION}{\texttt{public static final java.lang.String\ {\bf  DESCRIPTION}}
}
}
\end{itemize}
}
\subsection{Constructors}{
\vskip -2em
\begin{itemize}
\item{ 
\index{AverageRank(int)}
\hypertarget{de.sswis.model.algorithms.ranking.AverageRank(int)}{{\bf  AverageRank}\\}
\begin{lstlisting}[frame=none]
public AverageRank(int WINDOW_SIZE)\end{lstlisting} %end signature
\begin{itemize}
\item{
{\bf  Description}

Konstruktor
}
\item{
{\bf  Parameters}
  \begin{itemize}
   \item{
\texttt{WINDOW\_SIZE} -- Anzahl der zu betrachtenden Zyklen}
  \end{itemize}
}%end item
\end{itemize}
}%end item
\end{itemize}
}
\subsection{Methods}{
\vskip -2em
\begin{itemize}
\item{ 
\index{getRankings(Simulation)}
\hypertarget{de.sswis.model.algorithms.ranking.AverageRank.getRankings(de.sswis.model.Simulation)}{{\bf  getRankings}\\}
\begin{lstlisting}[frame=none]
java.util.HashMap getRankings(de.sswis.model.Simulation sim)\end{lstlisting} %end signature
\begin{itemize}
\item{
{\bf  Description copied from \hyperlink{de.sswis.model.algorithms.ranking.RankingAlgorithm}{RankingAlgorithm}{\small \refdefined{de.sswis.model.algorithms.ranking.RankingAlgorithm}} }

Bewertet die Agenten entsprechend des Algorithmus.
}
\item{
{\bf  Parameters}
  \begin{itemize}
   \item{
\texttt{sim} -- die Simulation deren Agenten bewertet werden sollen}
  \end{itemize}
}%end item
\item{{\bf  Returns} -- 
eine \texttt{\small HashMap} die jedem Agenten seinen Rang zuordnet 
}%end item
\end{itemize}
}%end item
\end{itemize}
}
}
\section{\label{de.sswis.model.algorithms.ranking.CurrentCycleScore}Class CurrentCycleScore}{
\hypertarget{de.sswis.model.algorithms.ranking.CurrentCycleScore}{}\vskip .1in 
Ein Algorithmus der die Agenten einer Simulation entsprechend ihrer Punktzahl im aktuellen Zyklus bewertet. Hat ein Agent eine höhere Punktzahl als ein anderer, so ist auch sein Rang höher.\vskip .1in 
\subsection{Declaration}{
\begin{lstlisting}[frame=none]
public class CurrentCycleScore
 extends java.lang.Object implements RankingAlgorithm\end{lstlisting}
\subsection{Fields}{
\begin{itemize}
\item{
\index{NAME}
\label{de.sswis.model.algorithms.ranking.CurrentCycleScore.NAME}\hypertarget{de.sswis.model.algorithms.ranking.CurrentCycleScore.NAME}{\texttt{public static final java.lang.String\ {\bf  NAME}}
}
}
\item{
\index{DESCRIPTION}
\label{de.sswis.model.algorithms.ranking.CurrentCycleScore.DESCRIPTION}\hypertarget{de.sswis.model.algorithms.ranking.CurrentCycleScore.DESCRIPTION}{\texttt{public static final java.lang.String\ {\bf  DESCRIPTION}}
}
}
\end{itemize}
}
\subsection{Constructors}{
\vskip -2em
\begin{itemize}
\item{ 
\index{CurrentCycleScore()}
\hypertarget{de.sswis.model.algorithms.ranking.CurrentCycleScore()}{{\bf  CurrentCycleScore}\\}
\begin{lstlisting}[frame=none]
public CurrentCycleScore()\end{lstlisting} %end signature
}%end item
\end{itemize}
}
\subsection{Methods}{
\vskip -2em
\begin{itemize}
\item{ 
\index{getRankings(Simulation)}
\hypertarget{de.sswis.model.algorithms.ranking.CurrentCycleScore.getRankings(de.sswis.model.Simulation)}{{\bf  getRankings}\\}
\begin{lstlisting}[frame=none]
java.util.HashMap getRankings(de.sswis.model.Simulation sim)\end{lstlisting} %end signature
\begin{itemize}
\item{
{\bf  Description copied from \hyperlink{de.sswis.model.algorithms.ranking.RankingAlgorithm}{RankingAlgorithm}{\small \refdefined{de.sswis.model.algorithms.ranking.RankingAlgorithm}} }

Bewertet die Agenten entsprechend des Algorithmus.
}
\item{
{\bf  Parameters}
  \begin{itemize}
   \item{
\texttt{sim} -- die Simulation deren Agenten bewertet werden sollen}
  \end{itemize}
}%end item
\item{{\bf  Returns} -- 
eine \texttt{\small HashMap} die jedem Agenten seinen Rang zuordnet 
}%end item
\end{itemize}
}%end item
\end{itemize}
}
}
\section{\label{de.sswis.model.algorithms.ranking.CustomCycleScore}Class CustomCycleScore}{
\hypertarget{de.sswis.model.algorithms.ranking.CustomCycleScore}{}\vskip .1in 
Ein Algorithmus der die Agenten einer Simulation entsprechend ihrer Punktzahl in den bisherigen \texttt{\small WINDOW\_SIZE} Zyklen bewertet. Hat ein Agent eine höhere Punktzahl als ein anderer, so ist auch sein Rang höher.\vskip .1in 
\subsection{Declaration}{
\begin{lstlisting}[frame=none]
public class CustomCycleScore
 extends java.lang.Object implements RankingAlgorithm\end{lstlisting}
\subsection{Fields}{
\begin{itemize}
\item{
\index{NAME}
\label{de.sswis.model.algorithms.ranking.CustomCycleScore.NAME}\hypertarget{de.sswis.model.algorithms.ranking.CustomCycleScore.NAME}{\texttt{public static final java.lang.String\ {\bf  NAME}}
}
}
\item{
\index{DESCRIPTION}
\label{de.sswis.model.algorithms.ranking.CustomCycleScore.DESCRIPTION}\hypertarget{de.sswis.model.algorithms.ranking.CustomCycleScore.DESCRIPTION}{\texttt{public static final java.lang.String\ {\bf  DESCRIPTION}}
}
}
\end{itemize}
}
\subsection{Constructors}{
\vskip -2em
\begin{itemize}
\item{ 
\index{CustomCycleScore(int)}
\hypertarget{de.sswis.model.algorithms.ranking.CustomCycleScore(int)}{{\bf  CustomCycleScore}\\}
\begin{lstlisting}[frame=none]
public CustomCycleScore(int WINDOW_SIZE)\end{lstlisting} %end signature
\begin{itemize}
\item{
{\bf  Description}

Konstruktor
}
\item{
{\bf  Parameters}
  \begin{itemize}
   \item{
\texttt{WINDOW\_SIZE} -- Anzahl der zu betrachtenden Zyklen}
  \end{itemize}
}%end item
\end{itemize}
}%end item
\end{itemize}
}
\subsection{Methods}{
\vskip -2em
\begin{itemize}
\item{ 
\index{getRankings(Simulation)}
\hypertarget{de.sswis.model.algorithms.ranking.CustomCycleScore.getRankings(de.sswis.model.Simulation)}{{\bf  getRankings}\\}
\begin{lstlisting}[frame=none]
java.util.HashMap getRankings(de.sswis.model.Simulation sim)\end{lstlisting} %end signature
\begin{itemize}
\item{
{\bf  Description copied from \hyperlink{de.sswis.model.algorithms.ranking.RankingAlgorithm}{RankingAlgorithm}{\small \refdefined{de.sswis.model.algorithms.ranking.RankingAlgorithm}} }

Bewertet die Agenten entsprechend des Algorithmus.
}
\item{
{\bf  Parameters}
  \begin{itemize}
   \item{
\texttt{sim} -- die Simulation deren Agenten bewertet werden sollen}
  \end{itemize}
}%end item
\item{{\bf  Returns} -- 
eine \texttt{\small HashMap} die jedem Agenten seinen Rang zuordnet 
}%end item
\end{itemize}
}%end item
\end{itemize}
}
}
\section{\label{de.sswis.model.algorithms.ranking.Score}Class Score}{
\hypertarget{de.sswis.model.algorithms.ranking.Score}{}\vskip .1in 
Ein Algorithmus der die Agenten einer Simulation entsprechend ihrer bisherigen Gesamtpunktzahl bewertet. Hat ein Agent eine höhere Punktzahl als ein anderer, so ist auch sein Rang höher.\vskip .1in 
\subsection{Declaration}{
\begin{lstlisting}[frame=none]
public class Score
 extends java.lang.Object implements RankingAlgorithm\end{lstlisting}
\subsection{Fields}{
\begin{itemize}
\item{
\index{NAME}
\label{de.sswis.model.algorithms.ranking.Score.NAME}\hypertarget{de.sswis.model.algorithms.ranking.Score.NAME}{\texttt{public static final java.lang.String\ {\bf  NAME}}
}
}
\item{
\index{DESCRIPTION}
\label{de.sswis.model.algorithms.ranking.Score.DESCRIPTION}\hypertarget{de.sswis.model.algorithms.ranking.Score.DESCRIPTION}{\texttt{public static final java.lang.String\ {\bf  DESCRIPTION}}
}
}
\end{itemize}
}
\subsection{Constructors}{
\vskip -2em
\begin{itemize}
\item{ 
\index{Score()}
\hypertarget{de.sswis.model.algorithms.ranking.Score()}{{\bf  Score}\\}
\begin{lstlisting}[frame=none]
public Score()\end{lstlisting} %end signature
}%end item
\end{itemize}
}
\subsection{Methods}{
\vskip -2em
\begin{itemize}
\item{ 
\index{getRankings(Simulation)}
\hypertarget{de.sswis.model.algorithms.ranking.Score.getRankings(de.sswis.model.Simulation)}{{\bf  getRankings}\\}
\begin{lstlisting}[frame=none]
java.util.HashMap getRankings(de.sswis.model.Simulation sim)\end{lstlisting} %end signature
\begin{itemize}
\item{
{\bf  Description copied from \hyperlink{de.sswis.model.algorithms.ranking.RankingAlgorithm}{RankingAlgorithm}{\small \refdefined{de.sswis.model.algorithms.ranking.RankingAlgorithm}} }

Bewertet die Agenten entsprechend des Algorithmus.
}
\item{
{\bf  Parameters}
  \begin{itemize}
   \item{
\texttt{sim} -- die Simulation deren Agenten bewertet werden sollen}
  \end{itemize}
}%end item
\item{{\bf  Returns} -- 
eine \texttt{\small HashMap} die jedem Agenten seinen Rang zuordnet 
}%end item
\end{itemize}
}%end item
\end{itemize}
}
}
}
\chapter{Package de.sswis.model.conditions}{
\label{de.sswis.model.conditions}\hypertarget{de.sswis.model.conditions}{}
\hskip -.05in
\hbox to \hsize{\textit{ Package Contents\hfil Page}}
\vskip .13in
\hbox{{\bf  Classes}}
\entityintro{Always}{de.sswis.model.conditions.Always}{Eine Bedingung die immer erfüllt ist.}
\entityintro{Condition}{de.sswis.model.conditions.Condition}{Eine Bedingung zur Auswahl einer Basisstrategie innerhalb einer kombinierten Strategie.}
\entityintro{Delta}{de.sswis.model.conditions.Delta}{Eine Bedingung die erfüllt ist, wenn beide Agenten ungefähr gleich reich sind.}
\entityintro{OwnGroup}{de.sswis.model.conditions.OwnGroup}{Eine Bedinung die erfüllt ist, wenn beide Agenten der gleichen Gruppe zugehörig sind.}
\entityintro{Poorer}{de.sswis.model.conditions.Poorer}{Eine Bedingung die erfüllt ist wenn der Gegenspieler ärmer ist.}
\entityintro{Probability}{de.sswis.model.conditions.Probability}{Eine Bedingung die mit einer gewissen Wahrscheinlichkeit \texttt{\small alpha} erfüllt ist.}
\entityintro{Richer}{de.sswis.model.conditions.Richer}{Eine Bedingung die erfüllt ist wenn der Gegenspieler reicher ist.}
\entityintro{SpecificGroup}{de.sswis.model.conditions.SpecificGroup}{Eine Bedingung die erfüllt ist, wenn der Gegenspieler Mitglied einer spezifischen Gruppe ist.}
\vskip .1in
\vskip .1in
\section{\label{de.sswis.model.conditions.Always}Class Always}{
\hypertarget{de.sswis.model.conditions.Always}{}\vskip .1in 
Eine Bedingung die immer erfüllt ist.\vskip .1in 
\subsection{Declaration}{
\begin{lstlisting}[frame=none]
public class Always
 extends de.sswis.model.conditions.Condition\end{lstlisting}
\subsection{Fields}{
\begin{itemize}
\item{
\index{NAME}
\label{de.sswis.model.conditions.Always.NAME}\hypertarget{de.sswis.model.conditions.Always.NAME}{\texttt{public static final java.lang.String\ {\bf  NAME}}
}
}
\item{
\index{DESCRIPTION}
\label{de.sswis.model.conditions.Always.DESCRIPTION}\hypertarget{de.sswis.model.conditions.Always.DESCRIPTION}{\texttt{public static final java.lang.String\ {\bf  DESCRIPTION}}
}
}
\end{itemize}
}
\subsection{Constructors}{
\vskip -2em
\begin{itemize}
\item{ 
\index{Always()}
\hypertarget{de.sswis.model.conditions.Always()}{{\bf  Always}\\}
\begin{lstlisting}[frame=none]
public Always()\end{lstlisting} %end signature
}%end item
\end{itemize}
}
\subsection{Methods}{
\vskip -2em
\begin{itemize}
\item{ 
\index{fulfillsCondition(Agent, Agent)}
\hypertarget{de.sswis.model.conditions.Always.fulfillsCondition(de.sswis.model.Agent, de.sswis.model.Agent)}{{\bf  fulfillsCondition}\\}
\begin{lstlisting}[frame=none]
public abstract boolean fulfillsCondition(de.sswis.model.Agent agent1,de.sswis.model.Agent agent2)\end{lstlisting} %end signature
\begin{itemize}
\item{
{\bf  Description copied from \hyperlink{de.sswis.model.conditions.Condition}{Condition}{\small \refdefined{de.sswis.model.conditions.Condition}} }

Überprüft ob die zwei Agenten die Bedingung erfüllen.
}
\item{
{\bf  Parameters}
  \begin{itemize}
   \item{
\texttt{agent1} -- Agent dessen kombinierte Strategie die Bedingung behinhaltet}
   \item{
\texttt{agent2} -- Gegenspieler}
  \end{itemize}
}%end item
\item{{\bf  Returns} -- 
\texttt{\small true}, wenn die Bedingung erfüllt ist, \texttt{\small false} sonst 
}%end item
\end{itemize}
}%end item
\end{itemize}
}
\subsection{Members inherited from class Condition }{
\texttt{de.sswis.model.conditions.Condition} {\small 
\refdefined{de.sswis.model.conditions.Condition}}
{\small 

fulfillsCondition}
}
\section{\label{de.sswis.model.conditions.Condition}Class Condition}{
\hypertarget{de.sswis.model.conditions.Condition}{}\vskip .1in 
Eine Bedingung zur Auswahl einer Basisstrategie innerhalb einer kombinierten Strategie. Die Bedingung kann sowohl vom Agenten abhängen dessen kombinierte Strategie sie beinhaltet, als auch von dessem Gegenspieler.\vskip .1in 
\subsection{Declaration}{
\begin{lstlisting}[frame=none]
public abstract class Condition
 extends java.lang.Object\end{lstlisting}
\subsection{All known subclasses}{SpecificGroup\small{\refdefined{de.sswis.model.conditions.SpecificGroup}}, Richer\small{\refdefined{de.sswis.model.conditions.Richer}}, Probability\small{\refdefined{de.sswis.model.conditions.Probability}}, Poorer\small{\refdefined{de.sswis.model.conditions.Poorer}}, OwnGroup\small{\refdefined{de.sswis.model.conditions.OwnGroup}}, Delta\small{\refdefined{de.sswis.model.conditions.Delta}}, Always\small{\refdefined{de.sswis.model.conditions.Always}}}
\subsection{Constructors}{
\vskip -2em
\begin{itemize}
\item{ 
\index{Condition()}
\hypertarget{de.sswis.model.conditions.Condition()}{{\bf  Condition}\\}
\begin{lstlisting}[frame=none]
public Condition()\end{lstlisting} %end signature
}%end item
\end{itemize}
}
\subsection{Methods}{
\vskip -2em
\begin{itemize}
\item{ 
\index{fulfillsCondition(Agent, Agent)}
\hypertarget{de.sswis.model.conditions.Condition.fulfillsCondition(de.sswis.model.Agent, de.sswis.model.Agent)}{{\bf  fulfillsCondition}\\}
\begin{lstlisting}[frame=none]
public abstract boolean fulfillsCondition(de.sswis.model.Agent agent1,de.sswis.model.Agent agent2)\end{lstlisting} %end signature
\begin{itemize}
\item{
{\bf  Description}

Überprüft ob die zwei Agenten die Bedingung erfüllen.
}
\item{
{\bf  Parameters}
  \begin{itemize}
   \item{
\texttt{agent1} -- Agent dessen kombinierte Strategie die Bedingung behinhaltet}
   \item{
\texttt{agent2} -- Gegenspieler}
  \end{itemize}
}%end item
\item{{\bf  Returns} -- 
\texttt{\small true}, wenn die Bedingung erfüllt ist, \texttt{\small false} sonst 
}%end item
\end{itemize}
}%end item
\end{itemize}
}
}
\section{\label{de.sswis.model.conditions.Delta}Class Delta}{
\hypertarget{de.sswis.model.conditions.Delta}{}\vskip .1in 
Eine Bedingung die erfüllt ist, wenn beide Agenten ungefähr gleich reich sind. Zwei Agenten sind ungefähr gleich reich, wenn der Betrag der Differenz ihrer Gesamtpunktzahlen kleiner oder gleich \texttt{\small delta} ist.\vskip .1in 
\subsection{Declaration}{
\begin{lstlisting}[frame=none]
public class Delta
 extends de.sswis.model.conditions.Condition\end{lstlisting}
\subsection{Fields}{
\begin{itemize}
\item{
\index{NAME}
\label{de.sswis.model.conditions.Delta.NAME}\hypertarget{de.sswis.model.conditions.Delta.NAME}{\texttt{public static final java.lang.String\ {\bf  NAME}}
}
}
\item{
\index{DESCRIPTION}
\label{de.sswis.model.conditions.Delta.DESCRIPTION}\hypertarget{de.sswis.model.conditions.Delta.DESCRIPTION}{\texttt{public static final java.lang.String\ {\bf  DESCRIPTION}}
}
}
\end{itemize}
}
\subsection{Constructors}{
\vskip -2em
\begin{itemize}
\item{ 
\index{Delta(double)}
\hypertarget{de.sswis.model.conditions.Delta(double)}{{\bf  Delta}\\}
\begin{lstlisting}[frame=none]
public Delta(double delta)\end{lstlisting} %end signature
\begin{itemize}
\item{
{\bf  Description}

Konstruktor
}
\item{
{\bf  Parameters}
  \begin{itemize}
   \item{
\texttt{delta} -- maximaler Betrag der Differenz der Gesamtpunktzahlen, der die Bedingung erfüllt}
  \end{itemize}
}%end item
\end{itemize}
}%end item
\end{itemize}
}
\subsection{Methods}{
\vskip -2em
\begin{itemize}
\item{ 
\index{fulfillsCondition(Agent, Agent)}
\hypertarget{de.sswis.model.conditions.Delta.fulfillsCondition(de.sswis.model.Agent, de.sswis.model.Agent)}{{\bf  fulfillsCondition}\\}
\begin{lstlisting}[frame=none]
public abstract boolean fulfillsCondition(de.sswis.model.Agent agent1,de.sswis.model.Agent agent2)\end{lstlisting} %end signature
\begin{itemize}
\item{
{\bf  Description copied from \hyperlink{de.sswis.model.conditions.Condition}{Condition}{\small \refdefined{de.sswis.model.conditions.Condition}} }

Überprüft ob die zwei Agenten die Bedingung erfüllen.
}
\item{
{\bf  Parameters}
  \begin{itemize}
   \item{
\texttt{agent1} -- Agent dessen kombinierte Strategie die Bedingung behinhaltet}
   \item{
\texttt{agent2} -- Gegenspieler}
  \end{itemize}
}%end item
\item{{\bf  Returns} -- 
\texttt{\small true}, wenn die Bedingung erfüllt ist, \texttt{\small false} sonst 
}%end item
\end{itemize}
}%end item
\end{itemize}
}
\subsection{Members inherited from class Condition }{
\texttt{de.sswis.model.conditions.Condition} {\small 
\refdefined{de.sswis.model.conditions.Condition}}
{\small 

fulfillsCondition}
}
\section{\label{de.sswis.model.conditions.OwnGroup}Class OwnGroup}{
\hypertarget{de.sswis.model.conditions.OwnGroup}{}\vskip .1in 
Eine Bedinung die erfüllt ist, wenn beide Agenten der gleichen Gruppe zugehörig sind.\vskip .1in 
\subsection{Declaration}{
\begin{lstlisting}[frame=none]
public class OwnGroup
 extends de.sswis.model.conditions.Condition\end{lstlisting}
\subsection{Fields}{
\begin{itemize}
\item{
\index{NAME}
\label{de.sswis.model.conditions.OwnGroup.NAME}\hypertarget{de.sswis.model.conditions.OwnGroup.NAME}{\texttt{public static final java.lang.String\ {\bf  NAME}}
}
}
\item{
\index{DESCRIPTION}
\label{de.sswis.model.conditions.OwnGroup.DESCRIPTION}\hypertarget{de.sswis.model.conditions.OwnGroup.DESCRIPTION}{\texttt{public static final java.lang.String\ {\bf  DESCRIPTION}}
}
}
\end{itemize}
}
\subsection{Constructors}{
\vskip -2em
\begin{itemize}
\item{ 
\index{OwnGroup()}
\hypertarget{de.sswis.model.conditions.OwnGroup()}{{\bf  OwnGroup}\\}
\begin{lstlisting}[frame=none]
public OwnGroup()\end{lstlisting} %end signature
}%end item
\end{itemize}
}
\subsection{Methods}{
\vskip -2em
\begin{itemize}
\item{ 
\index{fulfillsCondition(Agent, Agent)}
\hypertarget{de.sswis.model.conditions.OwnGroup.fulfillsCondition(de.sswis.model.Agent, de.sswis.model.Agent)}{{\bf  fulfillsCondition}\\}
\begin{lstlisting}[frame=none]
public abstract boolean fulfillsCondition(de.sswis.model.Agent agent1,de.sswis.model.Agent agent2)\end{lstlisting} %end signature
\begin{itemize}
\item{
{\bf  Description copied from \hyperlink{de.sswis.model.conditions.Condition}{Condition}{\small \refdefined{de.sswis.model.conditions.Condition}} }

Überprüft ob die zwei Agenten die Bedingung erfüllen.
}
\item{
{\bf  Parameters}
  \begin{itemize}
   \item{
\texttt{agent1} -- Agent dessen kombinierte Strategie die Bedingung behinhaltet}
   \item{
\texttt{agent2} -- Gegenspieler}
  \end{itemize}
}%end item
\item{{\bf  Returns} -- 
\texttt{\small true}, wenn die Bedingung erfüllt ist, \texttt{\small false} sonst 
}%end item
\end{itemize}
}%end item
\end{itemize}
}
\subsection{Members inherited from class Condition }{
\texttt{de.sswis.model.conditions.Condition} {\small 
\refdefined{de.sswis.model.conditions.Condition}}
{\small 

fulfillsCondition}
}
\section{\label{de.sswis.model.conditions.Poorer}Class Poorer}{
\hypertarget{de.sswis.model.conditions.Poorer}{}\vskip .1in 
Eine Bedingung die erfüllt ist wenn der Gegenspieler ärmer ist. Ein Agent ist ärmer als ein anderer, wenn er weniger Punkte hat.\vskip .1in 
\subsection{Declaration}{
\begin{lstlisting}[frame=none]
public class Poorer
 extends de.sswis.model.conditions.Condition\end{lstlisting}
\subsection{Fields}{
\begin{itemize}
\item{
\index{NAME}
\label{de.sswis.model.conditions.Poorer.NAME}\hypertarget{de.sswis.model.conditions.Poorer.NAME}{\texttt{public static final java.lang.String\ {\bf  NAME}}
}
}
\item{
\index{DESCRIPTION}
\label{de.sswis.model.conditions.Poorer.DESCRIPTION}\hypertarget{de.sswis.model.conditions.Poorer.DESCRIPTION}{\texttt{public static final java.lang.String\ {\bf  DESCRIPTION}}
}
}
\end{itemize}
}
\subsection{Constructors}{
\vskip -2em
\begin{itemize}
\item{ 
\index{Poorer()}
\hypertarget{de.sswis.model.conditions.Poorer()}{{\bf  Poorer}\\}
\begin{lstlisting}[frame=none]
public Poorer()\end{lstlisting} %end signature
}%end item
\end{itemize}
}
\subsection{Methods}{
\vskip -2em
\begin{itemize}
\item{ 
\index{fulfillsCondition(Agent, Agent)}
\hypertarget{de.sswis.model.conditions.Poorer.fulfillsCondition(de.sswis.model.Agent, de.sswis.model.Agent)}{{\bf  fulfillsCondition}\\}
\begin{lstlisting}[frame=none]
public abstract boolean fulfillsCondition(de.sswis.model.Agent agent1,de.sswis.model.Agent agent2)\end{lstlisting} %end signature
\begin{itemize}
\item{
{\bf  Description copied from \hyperlink{de.sswis.model.conditions.Condition}{Condition}{\small \refdefined{de.sswis.model.conditions.Condition}} }

Überprüft ob die zwei Agenten die Bedingung erfüllen.
}
\item{
{\bf  Parameters}
  \begin{itemize}
   \item{
\texttt{agent1} -- Agent dessen kombinierte Strategie die Bedingung behinhaltet}
   \item{
\texttt{agent2} -- Gegenspieler}
  \end{itemize}
}%end item
\item{{\bf  Returns} -- 
\texttt{\small true}, wenn die Bedingung erfüllt ist, \texttt{\small false} sonst 
}%end item
\end{itemize}
}%end item
\end{itemize}
}
\subsection{Members inherited from class Condition }{
\texttt{de.sswis.model.conditions.Condition} {\small 
\refdefined{de.sswis.model.conditions.Condition}}
{\small 

fulfillsCondition}
}
\section{\label{de.sswis.model.conditions.Probability}Class Probability}{
\hypertarget{de.sswis.model.conditions.Probability}{}\vskip .1in 
Eine Bedingung die mit einer gewissen Wahrscheinlichkeit \texttt{\small alpha} erfüllt ist.\vskip .1in 
\subsection{Declaration}{
\begin{lstlisting}[frame=none]
public class Probability
 extends de.sswis.model.conditions.Condition\end{lstlisting}
\subsection{Fields}{
\begin{itemize}
\item{
\index{NAME}
\label{de.sswis.model.conditions.Probability.NAME}\hypertarget{de.sswis.model.conditions.Probability.NAME}{\texttt{public static final java.lang.String\ {\bf  NAME}}
}
}
\item{
\index{DESCRIPTION}
\label{de.sswis.model.conditions.Probability.DESCRIPTION}\hypertarget{de.sswis.model.conditions.Probability.DESCRIPTION}{\texttt{public static final java.lang.String\ {\bf  DESCRIPTION}}
}
}
\end{itemize}
}
\subsection{Constructors}{
\vskip -2em
\begin{itemize}
\item{ 
\index{Probability(double)}
\hypertarget{de.sswis.model.conditions.Probability(double)}{{\bf  Probability}\\}
\begin{lstlisting}[frame=none]
public Probability(double alpha)\end{lstlisting} %end signature
\begin{itemize}
\item{
{\bf  Description}

Konstruktor
}
\item{
{\bf  Parameters}
  \begin{itemize}
   \item{
\texttt{alpha} -- Wahrscheinlichkeit mit der die Bedingung erfüllt ist}
  \end{itemize}
}%end item
\end{itemize}
}%end item
\end{itemize}
}
\subsection{Methods}{
\vskip -2em
\begin{itemize}
\item{ 
\index{fulfillsCondition(Agent, Agent)}
\hypertarget{de.sswis.model.conditions.Probability.fulfillsCondition(de.sswis.model.Agent, de.sswis.model.Agent)}{{\bf  fulfillsCondition}\\}
\begin{lstlisting}[frame=none]
public abstract boolean fulfillsCondition(de.sswis.model.Agent agent1,de.sswis.model.Agent agent2)\end{lstlisting} %end signature
\begin{itemize}
\item{
{\bf  Description copied from \hyperlink{de.sswis.model.conditions.Condition}{Condition}{\small \refdefined{de.sswis.model.conditions.Condition}} }

Überprüft ob die zwei Agenten die Bedingung erfüllen.
}
\item{
{\bf  Parameters}
  \begin{itemize}
   \item{
\texttt{agent1} -- Agent dessen kombinierte Strategie die Bedingung behinhaltet}
   \item{
\texttt{agent2} -- Gegenspieler}
  \end{itemize}
}%end item
\item{{\bf  Returns} -- 
\texttt{\small true}, wenn die Bedingung erfüllt ist, \texttt{\small false} sonst 
}%end item
\end{itemize}
}%end item
\end{itemize}
}
\subsection{Members inherited from class Condition }{
\texttt{de.sswis.model.conditions.Condition} {\small 
\refdefined{de.sswis.model.conditions.Condition}}
{\small 

fulfillsCondition}
}
\section{\label{de.sswis.model.conditions.Richer}Class Richer}{
\hypertarget{de.sswis.model.conditions.Richer}{}\vskip .1in 
Eine Bedingung die erfüllt ist wenn der Gegenspieler reicher ist. Ein Agent ist ärmer als ein anderer, wenn er mehr Punkte hat.\vskip .1in 
\subsection{Declaration}{
\begin{lstlisting}[frame=none]
public class Richer
 extends de.sswis.model.conditions.Condition\end{lstlisting}
\subsection{Fields}{
\begin{itemize}
\item{
\index{NAME}
\label{de.sswis.model.conditions.Richer.NAME}\hypertarget{de.sswis.model.conditions.Richer.NAME}{\texttt{public static final java.lang.String\ {\bf  NAME}}
}
}
\item{
\index{DESCRIPTION}
\label{de.sswis.model.conditions.Richer.DESCRIPTION}\hypertarget{de.sswis.model.conditions.Richer.DESCRIPTION}{\texttt{public static final java.lang.String\ {\bf  DESCRIPTION}}
}
}
\end{itemize}
}
\subsection{Constructors}{
\vskip -2em
\begin{itemize}
\item{ 
\index{Richer()}
\hypertarget{de.sswis.model.conditions.Richer()}{{\bf  Richer}\\}
\begin{lstlisting}[frame=none]
public Richer()\end{lstlisting} %end signature
}%end item
\end{itemize}
}
\subsection{Methods}{
\vskip -2em
\begin{itemize}
\item{ 
\index{fulfillsCondition(Agent, Agent)}
\hypertarget{de.sswis.model.conditions.Richer.fulfillsCondition(de.sswis.model.Agent, de.sswis.model.Agent)}{{\bf  fulfillsCondition}\\}
\begin{lstlisting}[frame=none]
public abstract boolean fulfillsCondition(de.sswis.model.Agent agent1,de.sswis.model.Agent agent2)\end{lstlisting} %end signature
\begin{itemize}
\item{
{\bf  Description copied from \hyperlink{de.sswis.model.conditions.Condition}{Condition}{\small \refdefined{de.sswis.model.conditions.Condition}} }

Überprüft ob die zwei Agenten die Bedingung erfüllen.
}
\item{
{\bf  Parameters}
  \begin{itemize}
   \item{
\texttt{agent1} -- Agent dessen kombinierte Strategie die Bedingung behinhaltet}
   \item{
\texttt{agent2} -- Gegenspieler}
  \end{itemize}
}%end item
\item{{\bf  Returns} -- 
\texttt{\small true}, wenn die Bedingung erfüllt ist, \texttt{\small false} sonst 
}%end item
\end{itemize}
}%end item
\end{itemize}
}
\subsection{Members inherited from class Condition }{
\texttt{de.sswis.model.conditions.Condition} {\small 
\refdefined{de.sswis.model.conditions.Condition}}
{\small 

fulfillsCondition}
}
\section{\label{de.sswis.model.conditions.SpecificGroup}Class SpecificGroup}{
\hypertarget{de.sswis.model.conditions.SpecificGroup}{}\vskip .1in 
Eine Bedingung die erfüllt ist, wenn der Gegenspieler Mitglied einer spezifischen Gruppe ist.\vskip .1in 
\subsection{Declaration}{
\begin{lstlisting}[frame=none]
public class SpecificGroup
 extends de.sswis.model.conditions.Condition\end{lstlisting}
\subsection{Fields}{
\begin{itemize}
\item{
\index{NAME}
\label{de.sswis.model.conditions.SpecificGroup.NAME}\hypertarget{de.sswis.model.conditions.SpecificGroup.NAME}{\texttt{public static final java.lang.String\ {\bf  NAME}}
}
}
\item{
\index{DESCRIPTION}
\label{de.sswis.model.conditions.SpecificGroup.DESCRIPTION}\hypertarget{de.sswis.model.conditions.SpecificGroup.DESCRIPTION}{\texttt{public static final java.lang.String\ {\bf  DESCRIPTION}}
}
}
\end{itemize}
}
\subsection{Constructors}{
\vskip -2em
\begin{itemize}
\item{ 
\index{SpecificGroup(int)}
\hypertarget{de.sswis.model.conditions.SpecificGroup(int)}{{\bf  SpecificGroup}\\}
\begin{lstlisting}[frame=none]
public SpecificGroup(int groupID)\end{lstlisting} %end signature
\begin{itemize}
\item{
{\bf  Description}

Konstruktor
}
\item{
{\bf  Parameters}
  \begin{itemize}
   \item{
\texttt{groupID} -- Gruppen-ID der Gruppe deren Mitglieder die Bedingung erfüllen}
  \end{itemize}
}%end item
\end{itemize}
}%end item
\end{itemize}
}
\subsection{Methods}{
\vskip -2em
\begin{itemize}
\item{ 
\index{fulfillsCondition(Agent, Agent)}
\hypertarget{de.sswis.model.conditions.SpecificGroup.fulfillsCondition(de.sswis.model.Agent, de.sswis.model.Agent)}{{\bf  fulfillsCondition}\\}
\begin{lstlisting}[frame=none]
public abstract boolean fulfillsCondition(de.sswis.model.Agent agent1,de.sswis.model.Agent agent2)\end{lstlisting} %end signature
\begin{itemize}
\item{
{\bf  Description copied from \hyperlink{de.sswis.model.conditions.Condition}{Condition}{\small \refdefined{de.sswis.model.conditions.Condition}} }

Überprüft ob die zwei Agenten die Bedingung erfüllen.
}
\item{
{\bf  Parameters}
  \begin{itemize}
   \item{
\texttt{agent1} -- Agent dessen kombinierte Strategie die Bedingung behinhaltet}
   \item{
\texttt{agent2} -- Gegenspieler}
  \end{itemize}
}%end item
\item{{\bf  Returns} -- 
\texttt{\small true}, wenn die Bedingung erfüllt ist, \texttt{\small false} sonst 
}%end item
\end{itemize}
}%end item
\end{itemize}
}
\subsection{Members inherited from class Condition }{
\texttt{de.sswis.model.conditions.Condition} {\small 
\refdefined{de.sswis.model.conditions.Condition}}
{\small 

fulfillsCondition}
}
}
\chapter{Package de.sswis.model.strategies}{
\label{de.sswis.model.strategies}\hypertarget{de.sswis.model.strategies}{}
\hskip -.05in
\hbox to \hsize{\textit{ Package Contents\hfil Page}}
\vskip .13in
\hbox{{\bf  Classes}}
\entityintro{AlwaysCooperate}{de.sswis.model.strategies.AlwaysCooperate}{Eine Basisstrategie, bei der der Agent immer kooperiert.}
\entityintro{BaseStrategy}{de.sswis.model.strategies.BaseStrategy}{Eine Basisstrategie die einem Agenten und seinem Gegenspieler eine Aktion zuordnet.}
\entityintro{Grim1}{de.sswis.model.strategies.Grim1}{Eine Basisstrategie, bei der der Agent kooperiert, wenn der Gegenspieler bei allen vorherigen gemeinsamen Spielen kooperiert hat.}
\entityintro{Grim2}{de.sswis.model.strategies.Grim2}{Eine Basisstrategie, bei der der Agent kooperiert, wenn der Gegenspieler bei allen vorherigen Spielen kooperiert hat.}
\entityintro{GroupGrim}{de.sswis.model.strategies.GroupGrim}{Eine Basisstrategie, bei der der Agent, aus Gruppe G, kooperiert, wenn der Gegenspieler bei allen vorherigen Spielen gegen einen Agenten aus Gruppe G kooperiert hat.}
\entityintro{GroupTitForTat}{de.sswis.model.strategies.GroupTitForTat}{Eine Basisstrategie, bei der der Agent, aus Gruppe G, kooperiert, wenn der Gegenspieler beim letzten Spiel gegen einen Agenten aus Gruppe G kooperiert hat.}
\entityintro{NeverCooperate}{de.sswis.model.strategies.NeverCooperate}{Eine Basisstrategie, bei der der Agent nie kooperiert.}
\entityintro{Random}{de.sswis.model.strategies.Random}{Eine Basisstrategie, bei der die Aktion des Agenten zufällig ist.}
\entityintro{TitForTat1}{de.sswis.model.strategies.TitForTat1}{Eine Basisstrategie, bei der der Agent kooperiert, wenn der Gegenspieler beim letzten gemeinsamen Spiel kooperiert hat.}
\entityintro{TitForTat2}{de.sswis.model.strategies.TitForTat2}{Eine Basisstrategie, bei der der Agent kooperiert, wenn der Gegenspieler aus dem letzten Spiel kooperiert hat.}
\vskip .1in
\vskip .1in
\section{\label{de.sswis.model.strategies.AlwaysCooperate}Class AlwaysCooperate}{
\hypertarget{de.sswis.model.strategies.AlwaysCooperate}{}\vskip .1in 
Eine Basisstrategie, bei der der Agent immer kooperiert.\vskip .1in 
\subsection{Declaration}{
\begin{lstlisting}[frame=none]
public class AlwaysCooperate
 extends de.sswis.model.strategies.BaseStrategy\end{lstlisting}
\subsection{Fields}{
\begin{itemize}
\item{
\index{NAME}
\label{de.sswis.model.strategies.AlwaysCooperate.NAME}\hypertarget{de.sswis.model.strategies.AlwaysCooperate.NAME}{\texttt{public static final java.lang.String\ {\bf  NAME}}
}
}
\item{
\index{DESCRIPTION}
\label{de.sswis.model.strategies.AlwaysCooperate.DESCRIPTION}\hypertarget{de.sswis.model.strategies.AlwaysCooperate.DESCRIPTION}{\texttt{public static final java.lang.String\ {\bf  DESCRIPTION}}
}
}
\end{itemize}
}
\subsection{Constructors}{
\vskip -2em
\begin{itemize}
\item{ 
\index{AlwaysCooperate()}
\hypertarget{de.sswis.model.strategies.AlwaysCooperate()}{{\bf  AlwaysCooperate}\\}
\begin{lstlisting}[frame=none]
public AlwaysCooperate()\end{lstlisting} %end signature
}%end item
\end{itemize}
}
\subsection{Methods}{
\vskip -2em
\begin{itemize}
\item{ 
\index{calculateAction(Agent, Agent)}
\hypertarget{de.sswis.model.strategies.AlwaysCooperate.calculateAction(de.sswis.model.Agent, de.sswis.model.Agent)}{{\bf  calculateAction}\\}
\begin{lstlisting}[frame=none]
public abstract de.sswis.model.Action calculateAction(de.sswis.model.Agent agent1,de.sswis.model.Agent agent2)\end{lstlisting} %end signature
\begin{itemize}
\item{
{\bf  Description copied from \hyperlink{de.sswis.model.strategies.BaseStrategy}{BaseStrategy}{\small \refdefined{de.sswis.model.strategies.BaseStrategy}} }

Berechnet die Aktion des Agenten entsprechend der Basisstrategie.
}
\item{
{\bf  Parameters}
  \begin{itemize}
   \item{
\texttt{agent1} -- Agent dessen kombinierte Strategie die Basisstrategie behinhaltet}
   \item{
\texttt{agent2} -- Gegenspieler}
  \end{itemize}
}%end item
\item{{\bf  Returns} -- 
eine \texttt{\small Action} die entweder Kooperation oder Defektion ist 
}%end item
\end{itemize}
}%end item
\end{itemize}
}
\subsection{Members inherited from class BaseStrategy }{
\texttt{de.sswis.model.strategies.BaseStrategy} {\small 
\refdefined{de.sswis.model.strategies.BaseStrategy}}
{\small 

calculateAction}
}
\section{\label{de.sswis.model.strategies.BaseStrategy}Class BaseStrategy}{
\hypertarget{de.sswis.model.strategies.BaseStrategy}{}\vskip .1in 
Eine Basisstrategie die einem Agenten und seinem Gegenspieler eine Aktion zuordnet. Die Basisstrategie kann sowohl vom Agenten abhängen dessen kombinierte Strategie sie beinhaltet, als auch von dessem Gegenspieler.\vskip .1in 
\subsection{Declaration}{
\begin{lstlisting}[frame=none]
public abstract class BaseStrategy
 extends java.lang.Object\end{lstlisting}
\subsection{All known subclasses}{TitForTat2\small{\refdefined{de.sswis.model.strategies.TitForTat2}}, TitForTat1\small{\refdefined{de.sswis.model.strategies.TitForTat1}}, Random\small{\refdefined{de.sswis.model.strategies.Random}}, NeverCooperate\small{\refdefined{de.sswis.model.strategies.NeverCooperate}}, GroupTitForTat\small{\refdefined{de.sswis.model.strategies.GroupTitForTat}}, GroupGrim\small{\refdefined{de.sswis.model.strategies.GroupGrim}}, Grim2\small{\refdefined{de.sswis.model.strategies.Grim2}}, Grim1\small{\refdefined{de.sswis.model.strategies.Grim1}}, AlwaysCooperate\small{\refdefined{de.sswis.model.strategies.AlwaysCooperate}}}
\subsection{Constructors}{
\vskip -2em
\begin{itemize}
\item{ 
\index{BaseStrategy()}
\hypertarget{de.sswis.model.strategies.BaseStrategy()}{{\bf  BaseStrategy}\\}
\begin{lstlisting}[frame=none]
public BaseStrategy()\end{lstlisting} %end signature
}%end item
\end{itemize}
}
\subsection{Methods}{
\vskip -2em
\begin{itemize}
\item{ 
\index{calculateAction(Agent, Agent)}
\hypertarget{de.sswis.model.strategies.BaseStrategy.calculateAction(de.sswis.model.Agent, de.sswis.model.Agent)}{{\bf  calculateAction}\\}
\begin{lstlisting}[frame=none]
public abstract de.sswis.model.Action calculateAction(de.sswis.model.Agent agent1,de.sswis.model.Agent agent2)\end{lstlisting} %end signature
\begin{itemize}
\item{
{\bf  Description}

Berechnet die Aktion des Agenten entsprechend der Basisstrategie.
}
\item{
{\bf  Parameters}
  \begin{itemize}
   \item{
\texttt{agent1} -- Agent dessen kombinierte Strategie die Basisstrategie behinhaltet}
   \item{
\texttt{agent2} -- Gegenspieler}
  \end{itemize}
}%end item
\item{{\bf  Returns} -- 
eine \texttt{\small Action} die entweder Kooperation oder Defektion ist 
}%end item
\end{itemize}
}%end item
\end{itemize}
}
}
\section{\label{de.sswis.model.strategies.Grim1}Class Grim1}{
\hypertarget{de.sswis.model.strategies.Grim1}{}\vskip .1in 
Eine Basisstrategie, bei der der Agent kooperiert, wenn der Gegenspieler bei allen vorherigen gemeinsamen Spielen kooperiert hat. Handelt es sich um das erste gemeinsame Spiel, so kooperiert der Agent.\vskip .1in 
\subsection{Declaration}{
\begin{lstlisting}[frame=none]
public class Grim1
 extends de.sswis.model.strategies.BaseStrategy\end{lstlisting}
\subsection{Fields}{
\begin{itemize}
\item{
\index{NAME}
\label{de.sswis.model.strategies.Grim1.NAME}\hypertarget{de.sswis.model.strategies.Grim1.NAME}{\texttt{public static final java.lang.String\ {\bf  NAME}}
}
}
\item{
\index{DESCRIPTION}
\label{de.sswis.model.strategies.Grim1.DESCRIPTION}\hypertarget{de.sswis.model.strategies.Grim1.DESCRIPTION}{\texttt{public static final java.lang.String\ {\bf  DESCRIPTION}}
}
}
\end{itemize}
}
\subsection{Constructors}{
\vskip -2em
\begin{itemize}
\item{ 
\index{Grim1()}
\hypertarget{de.sswis.model.strategies.Grim1()}{{\bf  Grim1}\\}
\begin{lstlisting}[frame=none]
public Grim1()\end{lstlisting} %end signature
}%end item
\end{itemize}
}
\subsection{Methods}{
\vskip -2em
\begin{itemize}
\item{ 
\index{calculateAction(Agent, Agent)}
\hypertarget{de.sswis.model.strategies.Grim1.calculateAction(de.sswis.model.Agent, de.sswis.model.Agent)}{{\bf  calculateAction}\\}
\begin{lstlisting}[frame=none]
public abstract de.sswis.model.Action calculateAction(de.sswis.model.Agent agent1,de.sswis.model.Agent agent2)\end{lstlisting} %end signature
\begin{itemize}
\item{
{\bf  Description copied from \hyperlink{de.sswis.model.strategies.BaseStrategy}{BaseStrategy}{\small \refdefined{de.sswis.model.strategies.BaseStrategy}} }

Berechnet die Aktion des Agenten entsprechend der Basisstrategie.
}
\item{
{\bf  Parameters}
  \begin{itemize}
   \item{
\texttt{agent1} -- Agent dessen kombinierte Strategie die Basisstrategie behinhaltet}
   \item{
\texttt{agent2} -- Gegenspieler}
  \end{itemize}
}%end item
\item{{\bf  Returns} -- 
eine \texttt{\small Action} die entweder Kooperation oder Defektion ist 
}%end item
\end{itemize}
}%end item
\end{itemize}
}
\subsection{Members inherited from class BaseStrategy }{
\texttt{de.sswis.model.strategies.BaseStrategy} {\small 
\refdefined{de.sswis.model.strategies.BaseStrategy}}
{\small 

calculateAction}
}
\section{\label{de.sswis.model.strategies.Grim2}Class Grim2}{
\hypertarget{de.sswis.model.strategies.Grim2}{}\vskip .1in 
Eine Basisstrategie, bei der der Agent kooperiert, wenn der Gegenspieler bei allen vorherigen Spielen kooperiert hat. Handelt es sich um das erste Spiel, so kooperiert der Agent.\vskip .1in 
\subsection{Declaration}{
\begin{lstlisting}[frame=none]
public class Grim2
 extends de.sswis.model.strategies.BaseStrategy\end{lstlisting}
\subsection{Fields}{
\begin{itemize}
\item{
\index{NAME}
\label{de.sswis.model.strategies.Grim2.NAME}\hypertarget{de.sswis.model.strategies.Grim2.NAME}{\texttt{public static final java.lang.String\ {\bf  NAME}}
}
}
\item{
\index{DESCRIPTION}
\label{de.sswis.model.strategies.Grim2.DESCRIPTION}\hypertarget{de.sswis.model.strategies.Grim2.DESCRIPTION}{\texttt{public static final java.lang.String\ {\bf  DESCRIPTION}}
}
}
\end{itemize}
}
\subsection{Constructors}{
\vskip -2em
\begin{itemize}
\item{ 
\index{Grim2()}
\hypertarget{de.sswis.model.strategies.Grim2()}{{\bf  Grim2}\\}
\begin{lstlisting}[frame=none]
public Grim2()\end{lstlisting} %end signature
}%end item
\end{itemize}
}
\subsection{Methods}{
\vskip -2em
\begin{itemize}
\item{ 
\index{calculateAction(Agent, Agent)}
\hypertarget{de.sswis.model.strategies.Grim2.calculateAction(de.sswis.model.Agent, de.sswis.model.Agent)}{{\bf  calculateAction}\\}
\begin{lstlisting}[frame=none]
public abstract de.sswis.model.Action calculateAction(de.sswis.model.Agent agent1,de.sswis.model.Agent agent2)\end{lstlisting} %end signature
\begin{itemize}
\item{
{\bf  Description copied from \hyperlink{de.sswis.model.strategies.BaseStrategy}{BaseStrategy}{\small \refdefined{de.sswis.model.strategies.BaseStrategy}} }

Berechnet die Aktion des Agenten entsprechend der Basisstrategie.
}
\item{
{\bf  Parameters}
  \begin{itemize}
   \item{
\texttt{agent1} -- Agent dessen kombinierte Strategie die Basisstrategie behinhaltet}
   \item{
\texttt{agent2} -- Gegenspieler}
  \end{itemize}
}%end item
\item{{\bf  Returns} -- 
eine \texttt{\small Action} die entweder Kooperation oder Defektion ist 
}%end item
\end{itemize}
}%end item
\end{itemize}
}
\subsection{Members inherited from class BaseStrategy }{
\texttt{de.sswis.model.strategies.BaseStrategy} {\small 
\refdefined{de.sswis.model.strategies.BaseStrategy}}
{\small 

calculateAction}
}
\section{\label{de.sswis.model.strategies.GroupGrim}Class GroupGrim}{
\hypertarget{de.sswis.model.strategies.GroupGrim}{}\vskip .1in 
Eine Basisstrategie, bei der der Agent, aus Gruppe G, kooperiert, wenn der Gegenspieler bei allen vorherigen Spielen gegen einen Agenten aus Gruppe G kooperiert hat. Hat der Gegenspieler noch nicht gegen einen Agenten aus Gruppe G gespielt, so kooperiert der Agent.\vskip .1in 
\subsection{Declaration}{
\begin{lstlisting}[frame=none]
public class GroupGrim
 extends de.sswis.model.strategies.BaseStrategy\end{lstlisting}
\subsection{Fields}{
\begin{itemize}
\item{
\index{NAME}
\label{de.sswis.model.strategies.GroupGrim.NAME}\hypertarget{de.sswis.model.strategies.GroupGrim.NAME}{\texttt{public static final java.lang.String\ {\bf  NAME}}
}
}
\item{
\index{DESCRIPTION}
\label{de.sswis.model.strategies.GroupGrim.DESCRIPTION}\hypertarget{de.sswis.model.strategies.GroupGrim.DESCRIPTION}{\texttt{public static final java.lang.String\ {\bf  DESCRIPTION}}
}
}
\end{itemize}
}
\subsection{Constructors}{
\vskip -2em
\begin{itemize}
\item{ 
\index{GroupGrim()}
\hypertarget{de.sswis.model.strategies.GroupGrim()}{{\bf  GroupGrim}\\}
\begin{lstlisting}[frame=none]
public GroupGrim()\end{lstlisting} %end signature
}%end item
\end{itemize}
}
\subsection{Methods}{
\vskip -2em
\begin{itemize}
\item{ 
\index{calculateAction(Agent, Agent)}
\hypertarget{de.sswis.model.strategies.GroupGrim.calculateAction(de.sswis.model.Agent, de.sswis.model.Agent)}{{\bf  calculateAction}\\}
\begin{lstlisting}[frame=none]
public abstract de.sswis.model.Action calculateAction(de.sswis.model.Agent agent1,de.sswis.model.Agent agent2)\end{lstlisting} %end signature
\begin{itemize}
\item{
{\bf  Description copied from \hyperlink{de.sswis.model.strategies.BaseStrategy}{BaseStrategy}{\small \refdefined{de.sswis.model.strategies.BaseStrategy}} }

Berechnet die Aktion des Agenten entsprechend der Basisstrategie.
}
\item{
{\bf  Parameters}
  \begin{itemize}
   \item{
\texttt{agent1} -- Agent dessen kombinierte Strategie die Basisstrategie behinhaltet}
   \item{
\texttt{agent2} -- Gegenspieler}
  \end{itemize}
}%end item
\item{{\bf  Returns} -- 
eine \texttt{\small Action} die entweder Kooperation oder Defektion ist 
}%end item
\end{itemize}
}%end item
\end{itemize}
}
\subsection{Members inherited from class BaseStrategy }{
\texttt{de.sswis.model.strategies.BaseStrategy} {\small 
\refdefined{de.sswis.model.strategies.BaseStrategy}}
{\small 

calculateAction}
}
\section{\label{de.sswis.model.strategies.GroupTitForTat}Class GroupTitForTat}{
\hypertarget{de.sswis.model.strategies.GroupTitForTat}{}\vskip .1in 
Eine Basisstrategie, bei der der Agent, aus Gruppe G, kooperiert, wenn der Gegenspieler beim letzten Spiel gegen einen Agenten aus Gruppe G kooperiert hat. Hat der Gegenspieler noch nicht gegen einen Agenten aus Gruppe G gespielt, so kooperiert der Agent.\vskip .1in 
\subsection{Declaration}{
\begin{lstlisting}[frame=none]
public class GroupTitForTat
 extends de.sswis.model.strategies.BaseStrategy\end{lstlisting}
\subsection{Fields}{
\begin{itemize}
\item{
\index{NAME}
\label{de.sswis.model.strategies.GroupTitForTat.NAME}\hypertarget{de.sswis.model.strategies.GroupTitForTat.NAME}{\texttt{public static final java.lang.String\ {\bf  NAME}}
}
}
\item{
\index{DESCRIPTION}
\label{de.sswis.model.strategies.GroupTitForTat.DESCRIPTION}\hypertarget{de.sswis.model.strategies.GroupTitForTat.DESCRIPTION}{\texttt{public static final java.lang.String\ {\bf  DESCRIPTION}}
}
}
\end{itemize}
}
\subsection{Constructors}{
\vskip -2em
\begin{itemize}
\item{ 
\index{GroupTitForTat()}
\hypertarget{de.sswis.model.strategies.GroupTitForTat()}{{\bf  GroupTitForTat}\\}
\begin{lstlisting}[frame=none]
public GroupTitForTat()\end{lstlisting} %end signature
}%end item
\end{itemize}
}
\subsection{Methods}{
\vskip -2em
\begin{itemize}
\item{ 
\index{calculateAction(Agent, Agent)}
\hypertarget{de.sswis.model.strategies.GroupTitForTat.calculateAction(de.sswis.model.Agent, de.sswis.model.Agent)}{{\bf  calculateAction}\\}
\begin{lstlisting}[frame=none]
public abstract de.sswis.model.Action calculateAction(de.sswis.model.Agent agent1,de.sswis.model.Agent agent2)\end{lstlisting} %end signature
\begin{itemize}
\item{
{\bf  Description copied from \hyperlink{de.sswis.model.strategies.BaseStrategy}{BaseStrategy}{\small \refdefined{de.sswis.model.strategies.BaseStrategy}} }

Berechnet die Aktion des Agenten entsprechend der Basisstrategie.
}
\item{
{\bf  Parameters}
  \begin{itemize}
   \item{
\texttt{agent1} -- Agent dessen kombinierte Strategie die Basisstrategie behinhaltet}
   \item{
\texttt{agent2} -- Gegenspieler}
  \end{itemize}
}%end item
\item{{\bf  Returns} -- 
eine \texttt{\small Action} die entweder Kooperation oder Defektion ist 
}%end item
\end{itemize}
}%end item
\end{itemize}
}
\subsection{Members inherited from class BaseStrategy }{
\texttt{de.sswis.model.strategies.BaseStrategy} {\small 
\refdefined{de.sswis.model.strategies.BaseStrategy}}
{\small 

calculateAction}
}
\section{\label{de.sswis.model.strategies.NeverCooperate}Class NeverCooperate}{
\hypertarget{de.sswis.model.strategies.NeverCooperate}{}\vskip .1in 
Eine Basisstrategie, bei der der Agent nie kooperiert.\vskip .1in 
\subsection{Declaration}{
\begin{lstlisting}[frame=none]
public class NeverCooperate
 extends de.sswis.model.strategies.BaseStrategy\end{lstlisting}
\subsection{Fields}{
\begin{itemize}
\item{
\index{NAME}
\label{de.sswis.model.strategies.NeverCooperate.NAME}\hypertarget{de.sswis.model.strategies.NeverCooperate.NAME}{\texttt{public static final java.lang.String\ {\bf  NAME}}
}
}
\item{
\index{DESCRIPTION}
\label{de.sswis.model.strategies.NeverCooperate.DESCRIPTION}\hypertarget{de.sswis.model.strategies.NeverCooperate.DESCRIPTION}{\texttt{public static final java.lang.String\ {\bf  DESCRIPTION}}
}
}
\end{itemize}
}
\subsection{Constructors}{
\vskip -2em
\begin{itemize}
\item{ 
\index{NeverCooperate()}
\hypertarget{de.sswis.model.strategies.NeverCooperate()}{{\bf  NeverCooperate}\\}
\begin{lstlisting}[frame=none]
public NeverCooperate()\end{lstlisting} %end signature
}%end item
\end{itemize}
}
\subsection{Methods}{
\vskip -2em
\begin{itemize}
\item{ 
\index{calculateAction(Agent, Agent)}
\hypertarget{de.sswis.model.strategies.NeverCooperate.calculateAction(de.sswis.model.Agent, de.sswis.model.Agent)}{{\bf  calculateAction}\\}
\begin{lstlisting}[frame=none]
public abstract de.sswis.model.Action calculateAction(de.sswis.model.Agent agent1,de.sswis.model.Agent agent2)\end{lstlisting} %end signature
\begin{itemize}
\item{
{\bf  Description copied from \hyperlink{de.sswis.model.strategies.BaseStrategy}{BaseStrategy}{\small \refdefined{de.sswis.model.strategies.BaseStrategy}} }

Berechnet die Aktion des Agenten entsprechend der Basisstrategie.
}
\item{
{\bf  Parameters}
  \begin{itemize}
   \item{
\texttt{agent1} -- Agent dessen kombinierte Strategie die Basisstrategie behinhaltet}
   \item{
\texttt{agent2} -- Gegenspieler}
  \end{itemize}
}%end item
\item{{\bf  Returns} -- 
eine \texttt{\small Action} die entweder Kooperation oder Defektion ist 
}%end item
\end{itemize}
}%end item
\end{itemize}
}
\subsection{Members inherited from class BaseStrategy }{
\texttt{de.sswis.model.strategies.BaseStrategy} {\small 
\refdefined{de.sswis.model.strategies.BaseStrategy}}
{\small 

calculateAction}
}
\section{\label{de.sswis.model.strategies.Random}Class Random}{
\hypertarget{de.sswis.model.strategies.Random}{}\vskip .1in 
Eine Basisstrategie, bei der die Aktion des Agenten zufällig ist.\vskip .1in 
\subsection{Declaration}{
\begin{lstlisting}[frame=none]
public class Random
 extends de.sswis.model.strategies.BaseStrategy\end{lstlisting}
\subsection{Fields}{
\begin{itemize}
\item{
\index{NAME}
\label{de.sswis.model.strategies.Random.NAME}\hypertarget{de.sswis.model.strategies.Random.NAME}{\texttt{public static final java.lang.String\ {\bf  NAME}}
}
}
\item{
\index{DESCRIPTION}
\label{de.sswis.model.strategies.Random.DESCRIPTION}\hypertarget{de.sswis.model.strategies.Random.DESCRIPTION}{\texttt{public static final java.lang.String\ {\bf  DESCRIPTION}}
}
}
\end{itemize}
}
\subsection{Constructors}{
\vskip -2em
\begin{itemize}
\item{ 
\index{Random()}
\hypertarget{de.sswis.model.strategies.Random()}{{\bf  Random}\\}
\begin{lstlisting}[frame=none]
public Random()\end{lstlisting} %end signature
}%end item
\end{itemize}
}
\subsection{Methods}{
\vskip -2em
\begin{itemize}
\item{ 
\index{calculateAction(Agent, Agent)}
\hypertarget{de.sswis.model.strategies.Random.calculateAction(de.sswis.model.Agent, de.sswis.model.Agent)}{{\bf  calculateAction}\\}
\begin{lstlisting}[frame=none]
public abstract de.sswis.model.Action calculateAction(de.sswis.model.Agent agent1,de.sswis.model.Agent agent2)\end{lstlisting} %end signature
\begin{itemize}
\item{
{\bf  Description copied from \hyperlink{de.sswis.model.strategies.BaseStrategy}{BaseStrategy}{\small \refdefined{de.sswis.model.strategies.BaseStrategy}} }

Berechnet die Aktion des Agenten entsprechend der Basisstrategie.
}
\item{
{\bf  Parameters}
  \begin{itemize}
   \item{
\texttt{agent1} -- Agent dessen kombinierte Strategie die Basisstrategie behinhaltet}
   \item{
\texttt{agent2} -- Gegenspieler}
  \end{itemize}
}%end item
\item{{\bf  Returns} -- 
eine \texttt{\small Action} die entweder Kooperation oder Defektion ist 
}%end item
\end{itemize}
}%end item
\end{itemize}
}
\subsection{Members inherited from class BaseStrategy }{
\texttt{de.sswis.model.strategies.BaseStrategy} {\small 
\refdefined{de.sswis.model.strategies.BaseStrategy}}
{\small 

calculateAction}
}
\section{\label{de.sswis.model.strategies.TitForTat1}Class TitForTat1}{
\hypertarget{de.sswis.model.strategies.TitForTat1}{}\vskip .1in 
Eine Basisstrategie, bei der der Agent kooperiert, wenn der Gegenspieler beim letzten gemeinsamen Spiel kooperiert hat. Handelt es sich um das erste gemeinsame Spiel, so kooperiert der Agent.\vskip .1in 
\subsection{Declaration}{
\begin{lstlisting}[frame=none]
public class TitForTat1
 extends de.sswis.model.strategies.BaseStrategy\end{lstlisting}
\subsection{Fields}{
\begin{itemize}
\item{
\index{NAME}
\label{de.sswis.model.strategies.TitForTat1.NAME}\hypertarget{de.sswis.model.strategies.TitForTat1.NAME}{\texttt{public static final java.lang.String\ {\bf  NAME}}
}
}
\item{
\index{DESCRIPTION}
\label{de.sswis.model.strategies.TitForTat1.DESCRIPTION}\hypertarget{de.sswis.model.strategies.TitForTat1.DESCRIPTION}{\texttt{public static final java.lang.String\ {\bf  DESCRIPTION}}
}
}
\end{itemize}
}
\subsection{Constructors}{
\vskip -2em
\begin{itemize}
\item{ 
\index{TitForTat1()}
\hypertarget{de.sswis.model.strategies.TitForTat1()}{{\bf  TitForTat1}\\}
\begin{lstlisting}[frame=none]
public TitForTat1()\end{lstlisting} %end signature
}%end item
\end{itemize}
}
\subsection{Methods}{
\vskip -2em
\begin{itemize}
\item{ 
\index{calculateAction(Agent, Agent)}
\hypertarget{de.sswis.model.strategies.TitForTat1.calculateAction(de.sswis.model.Agent, de.sswis.model.Agent)}{{\bf  calculateAction}\\}
\begin{lstlisting}[frame=none]
public abstract de.sswis.model.Action calculateAction(de.sswis.model.Agent agent1,de.sswis.model.Agent agent2)\end{lstlisting} %end signature
\begin{itemize}
\item{
{\bf  Description copied from \hyperlink{de.sswis.model.strategies.BaseStrategy}{BaseStrategy}{\small \refdefined{de.sswis.model.strategies.BaseStrategy}} }

Berechnet die Aktion des Agenten entsprechend der Basisstrategie.
}
\item{
{\bf  Parameters}
  \begin{itemize}
   \item{
\texttt{agent1} -- Agent dessen kombinierte Strategie die Basisstrategie behinhaltet}
   \item{
\texttt{agent2} -- Gegenspieler}
  \end{itemize}
}%end item
\item{{\bf  Returns} -- 
eine \texttt{\small Action} die entweder Kooperation oder Defektion ist 
}%end item
\end{itemize}
}%end item
\end{itemize}
}
\subsection{Members inherited from class BaseStrategy }{
\texttt{de.sswis.model.strategies.BaseStrategy} {\small 
\refdefined{de.sswis.model.strategies.BaseStrategy}}
{\small 

calculateAction}
}
\section{\label{de.sswis.model.strategies.TitForTat2}Class TitForTat2}{
\hypertarget{de.sswis.model.strategies.TitForTat2}{}\vskip .1in 
Eine Basisstrategie, bei der der Agent kooperiert, wenn der Gegenspieler aus dem letzten Spiel kooperiert hat. Handelt es sich um das erste Spiel, so kooperiert der Agent.\vskip .1in 
\subsection{Declaration}{
\begin{lstlisting}[frame=none]
public class TitForTat2
 extends de.sswis.model.strategies.BaseStrategy\end{lstlisting}
\subsection{Fields}{
\begin{itemize}
\item{
\index{NAME}
\label{de.sswis.model.strategies.TitForTat2.NAME}\hypertarget{de.sswis.model.strategies.TitForTat2.NAME}{\texttt{public static final java.lang.String\ {\bf  NAME}}
}
}
\item{
\index{DESCRIPTION}
\label{de.sswis.model.strategies.TitForTat2.DESCRIPTION}\hypertarget{de.sswis.model.strategies.TitForTat2.DESCRIPTION}{\texttt{public static final java.lang.String\ {\bf  DESCRIPTION}}
}
}
\end{itemize}
}
\subsection{Constructors}{
\vskip -2em
\begin{itemize}
\item{ 
\index{TitForTat2()}
\hypertarget{de.sswis.model.strategies.TitForTat2()}{{\bf  TitForTat2}\\}
\begin{lstlisting}[frame=none]
public TitForTat2()\end{lstlisting} %end signature
}%end item
\end{itemize}
}
\subsection{Methods}{
\vskip -2em
\begin{itemize}
\item{ 
\index{calculateAction(Agent, Agent)}
\hypertarget{de.sswis.model.strategies.TitForTat2.calculateAction(de.sswis.model.Agent, de.sswis.model.Agent)}{{\bf  calculateAction}\\}
\begin{lstlisting}[frame=none]
public abstract de.sswis.model.Action calculateAction(de.sswis.model.Agent agent1,de.sswis.model.Agent agent2)\end{lstlisting} %end signature
\begin{itemize}
\item{
{\bf  Description copied from \hyperlink{de.sswis.model.strategies.BaseStrategy}{BaseStrategy}{\small \refdefined{de.sswis.model.strategies.BaseStrategy}} }

Berechnet die Aktion des Agenten entsprechend der Basisstrategie.
}
\item{
{\bf  Parameters}
  \begin{itemize}
   \item{
\texttt{agent1} -- Agent dessen kombinierte Strategie die Basisstrategie behinhaltet}
   \item{
\texttt{agent2} -- Gegenspieler}
  \end{itemize}
}%end item
\item{{\bf  Returns} -- 
eine \texttt{\small Action} die entweder Kooperation oder Defektion ist 
}%end item
\end{itemize}
}%end item
\end{itemize}
}
\subsection{Members inherited from class BaseStrategy }{
\texttt{de.sswis.model.strategies.BaseStrategy} {\small 
\refdefined{de.sswis.model.strategies.BaseStrategy}}
{\small 

calculateAction}
}
}
\chapter{Package de.sswis.view}{
\label{de.sswis.view}\hypertarget{de.sswis.view}{}
\hskip -.05in
\hbox to \hsize{\textit{ Package Contents\hfil Page}}
\vskip .13in
\hbox{{\bf  Interfaces}}
\entityintro{AbstractMainView}{de.sswis.view.AbstractMainView}{}
\entityintro{AbstractManageCominedStartegiesView}{de.sswis.view.AbstractManageCominedStartegiesView}{}
\entityintro{AbstractManageConfigurationsView}{de.sswis.view.AbstractManageConfigurationsView}{}
\entityintro{AbstractManageGamesView}{de.sswis.view.AbstractManageGamesView}{}
\entityintro{AbstractManageInitializationsView}{de.sswis.view.AbstractManageInitializationsView}{}
\entityintro{AbstractManageResultsView}{de.sswis.view.AbstractManageResultsView}{}
\entityintro{AbstractManageStrategiesView}{de.sswis.view.AbstractManageStrategiesView}{}
\entityintro{AbstractNewCombinedStrategyView}{de.sswis.view.AbstractNewCombinedStrategyView}{}
\entityintro{AbstractNewConfigurationView}{de.sswis.view.AbstractNewConfigurationView}{}
\entityintro{AbstractNewGameView}{de.sswis.view.AbstractNewGameView}{}
\entityintro{AbstractNewInitializationView}{de.sswis.view.AbstractNewInitializationView}{}
\entityintro{AbstractNewStrategyView}{de.sswis.view.AbstractNewStrategyView}{}
\entityintro{AbstractShowCompareView}{de.sswis.view.AbstractShowCompareView}{}
\entityintro{AbstractShowMultiResultView}{de.sswis.view.AbstractShowMultiResultView}{}
\entityintro{AbstractShowResultView}{de.sswis.view.AbstractShowResultView}{}
\vskip .13in
\hbox{{\bf  Classes}}
\entityintro{MainView}{de.sswis.view.MainView}{}
\entityintro{ManageGamesView}{de.sswis.view.ManageGamesView}{}
\entityintro{ManageStrategiesView}{de.sswis.view.ManageStrategiesView}{}
\entityintro{NewConfigurationView}{de.sswis.view.NewConfigurationView}{}
\entityintro{NewInitializationView}{de.sswis.view.NewInitializationView}{}
\entityintro{ShowCompareView}{de.sswis.view.ShowCompareView}{}
\entityintro{ShowMultiResultView}{de.sswis.view.ShowMultiResultView}{}
\entityintro{ShowResultView}{de.sswis.view.ShowResultView}{}
\vskip .1in
\vskip .1in
\section{\label{de.sswis.view.AbstractMainView}Interface AbstractMainView}{
\hypertarget{de.sswis.view.AbstractMainView}{}\vskip .1in 
\subsection{Declaration}{
\begin{lstlisting}[frame=none]
public interface AbstractMainView
\end{lstlisting}
\subsection{All known subinterfaces}{MainView\small{\refdefined{de.sswis.view.MainView}}}
\subsection{All classes known to implement interface}{MainView\small{\refdefined{de.sswis.view.MainView}}}
}
\section{\label{de.sswis.view.AbstractManageCominedStartegiesView}Interface AbstractManageCominedStartegiesView}{
\hypertarget{de.sswis.view.AbstractManageCominedStartegiesView}{}\vskip .1in 
\subsection{Declaration}{
\begin{lstlisting}[frame=none]
public interface AbstractManageCominedStartegiesView
\end{lstlisting}
}
\section{\label{de.sswis.view.AbstractManageConfigurationsView}Interface AbstractManageConfigurationsView}{
\hypertarget{de.sswis.view.AbstractManageConfigurationsView}{}\vskip .1in 
\subsection{Declaration}{
\begin{lstlisting}[frame=none]
public interface AbstractManageConfigurationsView
\end{lstlisting}
}
\section{\label{de.sswis.view.AbstractManageGamesView}Interface AbstractManageGamesView}{
\hypertarget{de.sswis.view.AbstractManageGamesView}{}\vskip .1in 
\subsection{Declaration}{
\begin{lstlisting}[frame=none]
public interface AbstractManageGamesView
\end{lstlisting}
\subsection{All known subinterfaces}{ManageGamesView\small{\refdefined{de.sswis.view.ManageGamesView}}}
\subsection{All classes known to implement interface}{ManageGamesView\small{\refdefined{de.sswis.view.ManageGamesView}}}
}
\section{\label{de.sswis.view.AbstractManageInitializationsView}Interface AbstractManageInitializationsView}{
\hypertarget{de.sswis.view.AbstractManageInitializationsView}{}\vskip .1in 
\subsection{Declaration}{
\begin{lstlisting}[frame=none]
public interface AbstractManageInitializationsView
\end{lstlisting}
}
\section{\label{de.sswis.view.AbstractManageResultsView}Interface AbstractManageResultsView}{
\hypertarget{de.sswis.view.AbstractManageResultsView}{}\vskip .1in 
\subsection{Declaration}{
\begin{lstlisting}[frame=none]
public interface AbstractManageResultsView
\end{lstlisting}
}
\section{\label{de.sswis.view.AbstractManageStrategiesView}Interface AbstractManageStrategiesView}{
\hypertarget{de.sswis.view.AbstractManageStrategiesView}{}\vskip .1in 
\subsection{Declaration}{
\begin{lstlisting}[frame=none]
public interface AbstractManageStrategiesView
\end{lstlisting}
\subsection{All known subinterfaces}{ManageStrategiesView\small{\refdefined{de.sswis.view.ManageStrategiesView}}}
\subsection{All classes known to implement interface}{ManageStrategiesView\small{\refdefined{de.sswis.view.ManageStrategiesView}}}
}
\section{\label{de.sswis.view.AbstractNewCombinedStrategyView}Interface AbstractNewCombinedStrategyView}{
\hypertarget{de.sswis.view.AbstractNewCombinedStrategyView}{}\vskip .1in 
\subsection{Declaration}{
\begin{lstlisting}[frame=none]
public interface AbstractNewCombinedStrategyView
\end{lstlisting}
}
\section{\label{de.sswis.view.AbstractNewConfigurationView}Interface AbstractNewConfigurationView}{
\hypertarget{de.sswis.view.AbstractNewConfigurationView}{}\vskip .1in 
\subsection{Declaration}{
\begin{lstlisting}[frame=none]
public interface AbstractNewConfigurationView
\end{lstlisting}
\subsection{All known subinterfaces}{NewConfigurationView\small{\refdefined{de.sswis.view.NewConfigurationView}}}
\subsection{All classes known to implement interface}{NewConfigurationView\small{\refdefined{de.sswis.view.NewConfigurationView}}}
}
\section{\label{de.sswis.view.AbstractNewGameView}Interface AbstractNewGameView}{
\hypertarget{de.sswis.view.AbstractNewGameView}{}\vskip .1in 
\subsection{Declaration}{
\begin{lstlisting}[frame=none]
public interface AbstractNewGameView
\end{lstlisting}
}
\section{\label{de.sswis.view.AbstractNewInitializationView}Interface AbstractNewInitializationView}{
\hypertarget{de.sswis.view.AbstractNewInitializationView}{}\vskip .1in 
\subsection{Declaration}{
\begin{lstlisting}[frame=none]
public interface AbstractNewInitializationView
\end{lstlisting}
\subsection{All known subinterfaces}{NewInitializationView\small{\refdefined{de.sswis.view.NewInitializationView}}}
\subsection{All classes known to implement interface}{NewInitializationView\small{\refdefined{de.sswis.view.NewInitializationView}}}
}
\section{\label{de.sswis.view.AbstractNewStrategyView}Interface AbstractNewStrategyView}{
\hypertarget{de.sswis.view.AbstractNewStrategyView}{}\vskip .1in 
\subsection{Declaration}{
\begin{lstlisting}[frame=none]
public interface AbstractNewStrategyView
\end{lstlisting}
}
\section{\label{de.sswis.view.AbstractShowCompareView}Interface AbstractShowCompareView}{
\hypertarget{de.sswis.view.AbstractShowCompareView}{}\vskip .1in 
\subsection{Declaration}{
\begin{lstlisting}[frame=none]
public interface AbstractShowCompareView
\end{lstlisting}
\subsection{All known subinterfaces}{ShowCompareView\small{\refdefined{de.sswis.view.ShowCompareView}}}
\subsection{All classes known to implement interface}{ShowCompareView\small{\refdefined{de.sswis.view.ShowCompareView}}}
}
\section{\label{de.sswis.view.AbstractShowMultiResultView}Interface AbstractShowMultiResultView}{
\hypertarget{de.sswis.view.AbstractShowMultiResultView}{}\vskip .1in 
\subsection{Declaration}{
\begin{lstlisting}[frame=none]
public interface AbstractShowMultiResultView
\end{lstlisting}
\subsection{All known subinterfaces}{ShowMultiResultView\small{\refdefined{de.sswis.view.ShowMultiResultView}}}
\subsection{All classes known to implement interface}{ShowMultiResultView\small{\refdefined{de.sswis.view.ShowMultiResultView}}}
}
\section{\label{de.sswis.view.AbstractShowResultView}Interface AbstractShowResultView}{
\hypertarget{de.sswis.view.AbstractShowResultView}{}\vskip .1in 
\subsection{Declaration}{
\begin{lstlisting}[frame=none]
public interface AbstractShowResultView
\end{lstlisting}
\subsection{All known subinterfaces}{ShowResultView\small{\refdefined{de.sswis.view.ShowResultView}}}
\subsection{All classes known to implement interface}{ShowResultView\small{\refdefined{de.sswis.view.ShowResultView}}}
}
\section{\label{de.sswis.view.MainView}Class MainView}{
\hypertarget{de.sswis.view.MainView}{}\vskip .1in 
\subsection{Declaration}{
\begin{lstlisting}[frame=none]
public class MainView
 extends java.lang.Object implements AbstractMainView\end{lstlisting}
\subsection{Constructors}{
\vskip -2em
\begin{itemize}
\item{ 
\index{MainView()}
\hypertarget{de.sswis.view.MainView()}{{\bf  MainView}\\}
\begin{lstlisting}[frame=none]
public MainView()\end{lstlisting} %end signature
}%end item
\end{itemize}
}
}
\section{\label{de.sswis.view.ManageGamesView}Class ManageGamesView}{
\hypertarget{de.sswis.view.ManageGamesView}{}\vskip .1in 
\subsection{Declaration}{
\begin{lstlisting}[frame=none]
public class ManageGamesView
 extends java.lang.Object implements AbstractManageGamesView\end{lstlisting}
\subsection{Constructors}{
\vskip -2em
\begin{itemize}
\item{ 
\index{ManageGamesView()}
\hypertarget{de.sswis.view.ManageGamesView()}{{\bf  ManageGamesView}\\}
\begin{lstlisting}[frame=none]
public ManageGamesView()\end{lstlisting} %end signature
}%end item
\end{itemize}
}
}
\section{\label{de.sswis.view.ManageStrategiesView}Class ManageStrategiesView}{
\hypertarget{de.sswis.view.ManageStrategiesView}{}\vskip .1in 
\subsection{Declaration}{
\begin{lstlisting}[frame=none]
public class ManageStrategiesView
 extends java.lang.Object implements AbstractManageStrategiesView\end{lstlisting}
\subsection{Constructors}{
\vskip -2em
\begin{itemize}
\item{ 
\index{ManageStrategiesView()}
\hypertarget{de.sswis.view.ManageStrategiesView()}{{\bf  ManageStrategiesView}\\}
\begin{lstlisting}[frame=none]
public ManageStrategiesView()\end{lstlisting} %end signature
}%end item
\end{itemize}
}
}
\section{\label{de.sswis.view.NewConfigurationView}Class NewConfigurationView}{
\hypertarget{de.sswis.view.NewConfigurationView}{}\vskip .1in 
\subsection{Declaration}{
\begin{lstlisting}[frame=none]
public class NewConfigurationView
 extends java.lang.Object implements AbstractNewConfigurationView\end{lstlisting}
\subsection{Constructors}{
\vskip -2em
\begin{itemize}
\item{ 
\index{NewConfigurationView()}
\hypertarget{de.sswis.view.NewConfigurationView()}{{\bf  NewConfigurationView}\\}
\begin{lstlisting}[frame=none]
public NewConfigurationView()\end{lstlisting} %end signature
}%end item
\end{itemize}
}
}
\section{\label{de.sswis.view.NewInitializationView}Class NewInitializationView}{
\hypertarget{de.sswis.view.NewInitializationView}{}\vskip .1in 
\subsection{Declaration}{
\begin{lstlisting}[frame=none]
public class NewInitializationView
 extends java.lang.Object implements AbstractNewInitializationView\end{lstlisting}
\subsection{Constructors}{
\vskip -2em
\begin{itemize}
\item{ 
\index{NewInitializationView()}
\hypertarget{de.sswis.view.NewInitializationView()}{{\bf  NewInitializationView}\\}
\begin{lstlisting}[frame=none]
public NewInitializationView()\end{lstlisting} %end signature
}%end item
\end{itemize}
}
\subsection{Methods}{
\vskip -2em
\begin{itemize}
\item{ 
\index{\$\$\$getRootComponent\$\$\$()}
\hypertarget{de.sswis.view.NewInitializationView.$$$getRootComponent$$$()}{{\bf  \$\$\$getRootComponent\$\$\$}\\}
\begin{lstlisting}[frame=none]
public javax.swing.JComponent $$$getRootComponent$$$()\end{lstlisting} %end signature
\begin{itemize}
\item{{\bf  Returns} -- 
die Komponente 
}%end item
\end{itemize}
}%end item
\end{itemize}
}
}
\section{\label{de.sswis.view.ShowCompareView}Class ShowCompareView}{
\hypertarget{de.sswis.view.ShowCompareView}{}\vskip .1in 
\subsection{Declaration}{
\begin{lstlisting}[frame=none]
public class ShowCompareView
 extends java.lang.Object implements AbstractShowCompareView\end{lstlisting}
\subsection{Constructors}{
\vskip -2em
\begin{itemize}
\item{ 
\index{ShowCompareView()}
\hypertarget{de.sswis.view.ShowCompareView()}{{\bf  ShowCompareView}\\}
\begin{lstlisting}[frame=none]
public ShowCompareView()\end{lstlisting} %end signature
}%end item
\end{itemize}
}
}
\section{\label{de.sswis.view.ShowMultiResultView}Class ShowMultiResultView}{
\hypertarget{de.sswis.view.ShowMultiResultView}{}\vskip .1in 
\subsection{Declaration}{
\begin{lstlisting}[frame=none]
public class ShowMultiResultView
 extends java.lang.Object implements AbstractShowMultiResultView\end{lstlisting}
\subsection{Constructors}{
\vskip -2em
\begin{itemize}
\item{ 
\index{ShowMultiResultView()}
\hypertarget{de.sswis.view.ShowMultiResultView()}{{\bf  ShowMultiResultView}\\}
\begin{lstlisting}[frame=none]
public ShowMultiResultView()\end{lstlisting} %end signature
}%end item
\end{itemize}
}
}
\section{\label{de.sswis.view.ShowResultView}Class ShowResultView}{
\hypertarget{de.sswis.view.ShowResultView}{}\vskip .1in 
\subsection{Declaration}{
\begin{lstlisting}[frame=none]
public class ShowResultView
 extends java.lang.Object implements AbstractShowResultView\end{lstlisting}
\subsection{Constructors}{
\vskip -2em
\begin{itemize}
\item{ 
\index{ShowResultView()}
\hypertarget{de.sswis.view.ShowResultView()}{{\bf  ShowResultView}\\}
\begin{lstlisting}[frame=none]
public ShowResultView()\end{lstlisting} %end signature
}%end item
\end{itemize}
}
}
}
\chapter{Package de.sswis.view.model}{
\label{de.sswis.view.model}\hypertarget{de.sswis.view.model}{}
\hskip -.05in
\hbox to \hsize{\textit{ Package Contents\hfil Page}}
\vskip .13in
\hbox{{\bf  Classes}}
\entityintro{CombinedStrategy}{de.sswis.view.model.CombinedStrategy}{}
\entityintro{Configuration}{de.sswis.view.model.Configuration}{}
\entityintro{Game}{de.sswis.view.model.Game}{}
\entityintro{Group}{de.sswis.view.model.Group}{}
\entityintro{Initialization}{de.sswis.view.model.Initialization}{}
\entityintro{Result}{de.sswis.view.model.Result}{}
\entityintro{Strategy}{de.sswis.view.model.Strategy}{}
\vskip .1in
\vskip .1in
\section{\label{de.sswis.view.model.CombinedStrategy}Class CombinedStrategy}{
\hypertarget{de.sswis.view.model.CombinedStrategy}{}\vskip .1in 
\subsection{Declaration}{
\begin{lstlisting}[frame=none]
public class CombinedStrategy
 extends java.lang.Object\end{lstlisting}
\subsection{Constructors}{
\vskip -2em
\begin{itemize}
\item{ 
\index{CombinedStrategy()}
\hypertarget{de.sswis.view.model.CombinedStrategy()}{{\bf  CombinedStrategy}\\}
\begin{lstlisting}[frame=none]
public CombinedStrategy()\end{lstlisting} %end signature
}%end item
\end{itemize}
}
\subsection{Methods}{
\vskip -2em
\begin{itemize}
\item{ 
\index{isCorrect()}
\hypertarget{de.sswis.view.model.CombinedStrategy.isCorrect()}{{\bf  isCorrect}\\}
\begin{lstlisting}[frame=none]
public boolean isCorrect()\end{lstlisting} %end signature
}%end item
\end{itemize}
}
}
\section{\label{de.sswis.view.model.Configuration}Class Configuration}{
\hypertarget{de.sswis.view.model.Configuration}{}\vskip .1in 
\subsection{Declaration}{
\begin{lstlisting}[frame=none]
public class Configuration
 extends java.lang.Object\end{lstlisting}
\subsection{Constructors}{
\vskip -2em
\begin{itemize}
\item{ 
\index{Configuration()}
\hypertarget{de.sswis.view.model.Configuration()}{{\bf  Configuration}\\}
\begin{lstlisting}[frame=none]
public Configuration()\end{lstlisting} %end signature
}%end item
\end{itemize}
}
\subsection{Methods}{
\vskip -2em
\begin{itemize}
\item{ 
\index{isCorrect()}
\hypertarget{de.sswis.view.model.Configuration.isCorrect()}{{\bf  isCorrect}\\}
\begin{lstlisting}[frame=none]
public boolean isCorrect()\end{lstlisting} %end signature
}%end item
\end{itemize}
}
}
\section{\label{de.sswis.view.model.Game}Class Game}{
\hypertarget{de.sswis.view.model.Game}{}\vskip .1in 
\subsection{Declaration}{
\begin{lstlisting}[frame=none]
public class Game
 extends java.lang.Object\end{lstlisting}
\subsection{Constructors}{
\vskip -2em
\begin{itemize}
\item{ 
\index{Game()}
\hypertarget{de.sswis.view.model.Game()}{{\bf  Game}\\}
\begin{lstlisting}[frame=none]
public Game()\end{lstlisting} %end signature
}%end item
\end{itemize}
}
}
\section{\label{de.sswis.view.model.Group}Class Group}{
\hypertarget{de.sswis.view.model.Group}{}\vskip .1in 
\subsection{Declaration}{
\begin{lstlisting}[frame=none]
public class Group
 extends java.lang.Object\end{lstlisting}
\subsection{Constructors}{
\vskip -2em
\begin{itemize}
\item{ 
\index{Group()}
\hypertarget{de.sswis.view.model.Group()}{{\bf  Group}\\}
\begin{lstlisting}[frame=none]
public Group()\end{lstlisting} %end signature
}%end item
\end{itemize}
}
}
\section{\label{de.sswis.view.model.Initialization}Class Initialization}{
\hypertarget{de.sswis.view.model.Initialization}{}\vskip .1in 
\subsection{Declaration}{
\begin{lstlisting}[frame=none]
public class Initialization
 extends java.lang.Object\end{lstlisting}
\subsection{Constructors}{
\vskip -2em
\begin{itemize}
\item{ 
\index{Initialization()}
\hypertarget{de.sswis.view.model.Initialization()}{{\bf  Initialization}\\}
\begin{lstlisting}[frame=none]
public Initialization()\end{lstlisting} %end signature
}%end item
\end{itemize}
}
}
\section{\label{de.sswis.view.model.Result}Class Result}{
\hypertarget{de.sswis.view.model.Result}{}\vskip .1in 
\subsection{Declaration}{
\begin{lstlisting}[frame=none]
public class Result
 extends java.lang.Object\end{lstlisting}
\subsection{Constructors}{
\vskip -2em
\begin{itemize}
\item{ 
\index{Result()}
\hypertarget{de.sswis.view.model.Result()}{{\bf  Result}\\}
\begin{lstlisting}[frame=none]
public Result()\end{lstlisting} %end signature
}%end item
\end{itemize}
}
}
\section{\label{de.sswis.view.model.Strategy}Class Strategy}{
\hypertarget{de.sswis.view.model.Strategy}{}\vskip .1in 
\subsection{Declaration}{
\begin{lstlisting}[frame=none]
public class Strategy
 extends java.lang.Object\end{lstlisting}
\subsection{Constructors}{
\vskip -2em
\begin{itemize}
\item{ 
\index{Strategy()}
\hypertarget{de.sswis.view.model.Strategy()}{{\bf  Strategy}\\}
\begin{lstlisting}[frame=none]
public Strategy()\end{lstlisting} %end signature
}%end item
\end{itemize}
}
\subsection{Methods}{
\vskip -2em
\begin{itemize}
\item{ 
\index{isCorrect()}
\hypertarget{de.sswis.view.model.Strategy.isCorrect()}{{\bf  isCorrect}\\}
\begin{lstlisting}[frame=none]
public boolean isCorrect()\end{lstlisting} %end signature
}%end item
\end{itemize}
}
}
}
