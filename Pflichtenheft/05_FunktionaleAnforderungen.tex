\section{Funktionale Anforderungen}

\subsection{Pflichtkriterien}

\subsubsection{Allgemein}

\textbf{/F0000/ Simulation erstellen}

Es können Simulationen erstellt werden.

\textbf{/F0010/ Simulation starten}

Die Simulation kann gestartet werden.

\textbf{/F0020/ Ergebnisse anzeigen}

Die Ergebnisse der Simulation können angezeigt werden.

\subsubsection{Simulation erstellen}

\textbf{/F0100/ Anzahl der Agenten}

Die Anzahl der Agenten kann festgelegt werden.

\textbf{/F0110/ Initiale Strategie}

Die initiale Strategie der Agenten kann festgelegt werden.

\textbf{/F0120/ Spiel}

Das zu simulierende Spiel kann festgelegt werden.

\textbf{/F0130/ Rundenanzahl}

Die Anzahl der Runden kann festgelegt werden.

\textbf{/F0140/ Zyklenanzahl}

Die Anzahl der Zyklen kann festgelegt werden.

\textbf{/F0150/ Wiederholungen}

Die Anzahl der Wiederholungen der Simulation kann festgelegt werden.

\textbf{/F0160/ Strategieanpassung}

Die Art der Strategieanpassung kann festgelegt werden. 

\textbf{/F0170/ Wahrscheinlichkeit für Strategieanpassung}

Die Wahrscheinlichkeit für eine Strategieanpassung am Ende eines Zyklus kann festgelegt werden.

\textbf{/F0180 Gruppen}

Die Agenten können in Gruppen eingeteilt werden.

\textbf{/F0190 Standardwerte}

Es werden Standartwerte für die Konfiguration aus einer Config-Datei geladen.

\textbf{/F0200/ Menge von Konfigurationen}

Für die Konfigurationswerte aus /F0100/ bis /F0170/ kann man mehrere Werte angeben. Für Zahlenwerte kann man einen Wertebereich der Form \emph{Startwert} - \emph{Endwert} - \emph{Schrittweite} angeben. Daraus entsteht eine Menge von Konfigurationen. Alle Konfigurationen werden simuliert.

\subsubsection{Spiele}

\textbf{/F0300/ Gefangenendilemma}

\textbf{/F0310/ Feiglingsspiel}

\textbf{/F0320/ Hirschjagd}

\textbf{/F0330/ Falke-Taube}

\textbf{/F0340/ Kampf der Geschlechter}

\textbf{/F0350/ Vertrauensspiel}

\textbf{/F0360/ Eigenes Spiel}

Erstellen eines Spiels durch Angabe der Payoffs. 

\subsection{Wunschkriterien}
