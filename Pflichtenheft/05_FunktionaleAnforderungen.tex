\section{Funktionale Anforderungen}

\subsection{Pflichtkriterien}

\subsubsection{Allgemein}

\textbf{/F0000/}
Es können Simulationen erstellt werden.

\textbf{/F0010/} 
Die Simulation kann gestartet werden.

\textbf{/F0020/} 
Die Ergebnisse der Simulation können angezeigt werden.

\subsubsection{Simulation erstellen}

\textbf{/F0100/} 
Die Anzahl der Agenten kann festgelegt werden. Es werden nur gerade Zahlen zugelassen.

\textbf{/F0110/} 
Die initiale Strategie der Agenten kann festgelegt werden. Dabei wird der prozentuale Anteil der der Strategie im Startzustand angegeben.  

\textbf{/F0120/} 
Das zu simulierende Spiel kann festgelegt werden.

\textbf{/F0130/} 
Die Anzahl der Runden kann festgelegt werden.

\textbf{/F0140/} 
Die Anzahl der Zyklen kann festgelegt werden.

\textbf{/F0150/} 
Die Anzahl der Wiederholungen der Simulation kann festgelegt werden.

\textbf{/F0160/} 
Die Art der Strategieanpassung kann festgelegt werden. 

\textbf{/F0170/} 
Die Wahrscheinlichkeit für eine Strategieanpassung am Ende eines Zyklus kann festgelegt werden. Es sind Werte zwischen 0.01 und 1 zugelassen.

\textbf{/F0180/} 
Die Bewertungsart mit der die Agenten verglichen werden kann festgelegt werden.

\textbf{/F0190/} 
Die Agenten können in Gruppen eingeteilt werden. Die Gruppen müssen nicht disjunkt sein.

\textbf{/F0200/} 
Es werden Standartwerte für die Konfiguration aus einer Config-Datei geladen.

\textbf{/F0210/} 
Für die Anteile der initialen Strategien kann man mehrere Werte in der Form \emph{Startwert} - \emph{Endwert} - \emph{Schrittweite} angeben. Daraus entsteht eine Menge von Konfigurationen. Jede Konfiguration erhält eine eigene Simulation. Die prozentuale Anteile der aller Strategien in einer Konfiguration muss zusammen immer 100 ergeben. 

\subsubsection{Spiele}
Es werden alle implementierten Spiele mit ihren Payoffs angegeben. $K$ steht für die Aktion \emph{Kooperation} und $D$ für \emph{Defektion}.

\textbf{/F0300/} 
Gefangenendilemma
\begin{table}[H]
\centering
\setlength{\extrarowheight}{2pt}
\begin{tabular}{cc|c|c|}
  & \multicolumn{1}{c}{} & \multicolumn{2}{c}{Spieler $2$} \\
  & \multicolumn{1}{c}{} & \multicolumn{1}{c}{$K$} & \multicolumn{1}{c}{$D$} \\\cline{3-4}
  \multirow{2}*{Spieler $1$} & $K$ & $-1,-1$ & $-3,0$ \\\cline{3-4} 
  & $D$ & $0,-3$ & $-2,-2$ \\\cline{3-4}
\end{tabular}
\end{table}

\textbf{/F0310/} 
Feiglingsspiel
\begin{table}[H]
\centering
\setlength{\extrarowheight}{2pt}
\begin{tabular}{cc|c|c|}
  & \multicolumn{1}{c}{} & \multicolumn{2}{c}{Spieler $2$} \\
  & \multicolumn{1}{c}{} & \multicolumn{1}{c}{$K$} & \multicolumn{1}{c}{$D$} \\\cline{3-4}
  \multirow{2}*{Spieler $1$} & $K$ & $4,4$ & $2,6$ \\\cline{3-4} 
  & $D$ & $6,2$ & $0,0$ \\\cline{3-4}
\end{tabular}
\end{table}

\textbf{/F0320/} 
Hirschjagd
\begin{table}[H]
\centering
\setlength{\extrarowheight}{2pt}
\begin{tabular}{cc|c|c|}
  & \multicolumn{1}{c}{} & \multicolumn{2}{c}{Spieler $2$} \\
  & \multicolumn{1}{c}{} & \multicolumn{1}{c}{$K$} & \multicolumn{1}{c}{$D$} \\\cline{3-4}
  \multirow{2}*{Spieler $1$} & $K$ & $4,4$ & $0,3$ \\\cline{3-4} 
  & $D$ & $3,0$ & $3,3$ \\\cline{3-4}
\end{tabular}
\end{table}

\textbf{/F0330/} 
Falke-Taube
\begin{table}[H]
\centering
\setlength{\extrarowheight}{2pt}
\begin{tabular}{cc|c|c|}
  & \multicolumn{1}{c}{} & \multicolumn{2}{c}{Spieler $2$} \\
  & \multicolumn{1}{c}{} & \multicolumn{1}{c}{$K$} & \multicolumn{1}{c}{$D$} \\\cline{3-4}
  \multirow{2}*{Spieler $1$} & $K$ & $2,2$ & $1,4$ \\\cline{3-4} 
  & $D$ & $4,1$ & $0,0$ \\\cline{3-4}
\end{tabular}
\end{table}

\textbf{/F0340/} 
Kampf der Geschlechter
\begin{table}[H]
\centering
\setlength{\extrarowheight}{2pt}
\begin{tabular}{cc|c|c|}
  & \multicolumn{1}{c}{} & \multicolumn{2}{c}{Spieler $2$} \\
  & \multicolumn{1}{c}{} & \multicolumn{1}{c}{$K$} & \multicolumn{1}{c}{$D$} \\\cline{3-4}
  \multirow{2}*{Spieler $1$} & $K$ & $0,0$ & $1,3$ \\\cline{3-4} 
  & $D$ & $3,1$ & $0,0$ \\\cline{3-4}
\end{tabular}
\end{table}

\textbf{/F0350/} 
Vertrauensspiel
\begin{table}[H]
\centering
\setlength{\extrarowheight}{2pt}
\begin{tabular}{cc|c|c|}
  & \multicolumn{1}{c}{} & \multicolumn{2}{c}{Spieler $2$} \\
  & \multicolumn{1}{c}{} & \multicolumn{1}{c}{$K$} & \multicolumn{1}{c}{$D$} \\\cline{3-4}
  \multirow{2}*{Spieler $1$} & $K$ & $1,1$ & $-1,2$ \\\cline{3-4} 
  & $D$ & $0,0$ & $0,0$ \\\cline{3-4}
\end{tabular}
\end{table}

\textbf{/F0360/} 
Erstellen eines Spiels durch Angabe der Payoffs.

\subsubsection{Strategien}

\textbf{/F0400/}
Die Strategien /F0410/ bis /F0450/ können mit einer Bedingung aus /F0460/ bis /F0510/ und einer weiteren Strategie kombiniert werden. Daraus entsteht eine gemischte Strategie. Ist die Bedingung erfüllt wird die erste Strategie ausgeführt, ansonsten die zweite Strategie.  

\textbf{/F0410/} 
Strategie: Tit-for-Tat. Der Agent kooperiert, wenn der andere Agent das letzte mal mit ihm kooperiert hat. Beim ersten mal kooperiert der Agent.

\textbf{/F0420/} 
Strategie: Grim. Der Agent kooperiert, wenn der andere Agent immer mit im Kooperiert hat. Beim ersten mal kooperiert der Agent.

\textbf{/F0430/} 
Strategie: der Agent kooperiert immer.

\textbf{/F0440/}
Strategie: der Agent kooperiert nie.

\textbf{/F0450/}
Strategie: die Aktion des Agenten ist zufällig.

\textbf{/F0460/}
Bedingung: Wähle die erste Strategie mit Wahrscheinlichkeit $\alpha$. $\alpha$ wird beim erstellen der Simulation angegeben.

\textbf{/F0470/}
Bedingung: Wähle die erste Strategie, wenn der andere Agent zur Gruppe $G$ gehört. $G$ wird beim erstellen der Simulation angegeben.

\textbf{/F0480/}
Bedingung: Wähle die erste Strategie, wenn der andere Agent zu meiner Gruppe gehört. Ist der Agent in keiner Gruppe, wählt er die erste Strategie, wenn der anderer Agent ebenfalls in keiner Gruppe ist.

\textbf{/F0490/}
Bedingung: Wähle die erste Strategie, wenn der andere Agent reicher ist.

\textbf{/F0500/}
Bedingung: Wähle die erste Strategie, wenn der andere Agent ärmer ist.

\textbf{/F0510/}
Bedingung: Wähle die erste Strategie, wenn der andere Agent ungefähr gleich reich ist, d.h. wenn er $\pm$ 50 Punkte hat. 

\subsubsection{Anpassung der Strategie}


\subsection{Wunschkriterien}

\subsubsection{Konfiguration einer Simulation}

\textbf{/W0000/ Konfiguration einer simulation  speichern}

Die Konfiguration einer Simulation kann bei Bedarf gespeichert werden.

\textbf{/W0001/ Konfiguration für eine simulation laden}

Die gespeicherte Konfiguration kann bei Bedarf für eine Simulation geladen werden.

\subsubsection{Ergebnis einer Simulation}

\textbf{/W0100/ Ergebnis einer Simulation speichern}

Das Ergebnis einer Simulation kann bei Bedarf gespeichert werden.

\textbf{/W0101/ Ergebnis einer Simulation laden}

Das gespeicherte Ergebnis einer Simulation kann bei Bedarf geladen werden.

\subsubsection{Startkapital für Agenten}

\textbf{/W0200/ Startkapital für Agenten festlegen}

Das Startkapital für Agenten kann für jede Runde festgelegt werden.

\subsubsection{weitere Strategien}

\textbf{/W0300/ Strategien für die Paarung von Agenten}

Die Strategien für die Paarung von Agenten können festgelegt werden.

\subsubsection{Zustände der Agenten}

\textbf{/W0400/ Zustände der Agenten während der Simulation bei der Auswertung anzeigen}

Zustände der Agenten während der Simulatoon können bei der Auswertung angezeigt werden.

\subsubsection{Algorithmen für Strategieanpassung}

\textbf{/W0500/ Algorithmen für Strategieanpassung der Agenten}

Die Algorithmen für Strategieanpassung der Agenten können festgelegt werden.