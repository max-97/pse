\section{Glossar}

\textbf{Spiel}
Ein spieltheoretisches Stufenspiel, welches sich als 2x2 Matrix darstellen lässt, wobei sich an jeder der 4 Einträge die Payoffs für beide Spieler befinden.

\textbf{Strategie}
Eine Strategie eines Agenten A bestimmt die Aktion von A im Verlauf eines Spiels mit einem Agenten B. Die Strategie kann von den vergangenen Aktionen des Agenten B, den Rängen von A und B, sowie deren Gruppenzugehörigkeit abhängen. Die Strategie eines Agenten kann sich während der Simulation ändern.

\textbf{Runde}
Eine Runde umfasst die Paarung aller Agenten und das einmalige Spielen des ausgewählten Stufenspiels.

\textbf{Zyklus}
Ein Zyklus umfasst $k$ Runden sowie das Auswerten und Anpassen der Strategien.

\textbf{Gruppe}
Eine Gruppe ist eine Menge von Agenten, die vom Benutzer festgelegt werden kann.

\textbf{Gleichgewicht}
Ein Gleichgewicht bezeichnet den Zustand, in dem die Agenten in den letzten 500 Zyklen ihre Strategie nicht mehr angepasst haben.

\textbf{reicher und ärmer}
Ein Agent A ist reicher bzw. ärmer als ein Agent B, wenn er mehr bzw. weniger Punkte hat.

\textbf{ungefähr gleich reich}
Zwei Agenten sind ungefähr gleich reich, wenn die Differenz ihrer Punkte $\pm 50$ Punkte beträgt.

\textbf{Konfiguration}
Eine Menge von Parametern, die eine Simulation oder eine Menge von Simulationen\footnote{vgl. /F0210/} beschreibt.

\textbf{kombinierte Strategie}
Ein kombinierte Strategie besteht aus $k$ Strategie-Bedingungspaaren\footnote{siehe /F0400/}. 

\textbf{gemischte Strategie}
Eine gemischte Strategie besteht aus 

\textbf{Initialisierung}

\textbf{Simulation}

\textbf{}

\textbf{}