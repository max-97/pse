
\section{Globale Testfälle und Szenarien}

  
% usage: \testfall{szenario}{ablauf}{ergebnis} oder
\newcommand{\testfall}[4][]{
  \begin{description}
    \item[Stand] #2
    \item[Aktion] #3
    \item[Reaktion] #4
  \end{description}
}

\subsection{Testfälle}



\textbf{/T001/} Konfiguration erstellen
\begin{enumerate}
\item \testfall{Das Startfenster ist geöffnet und es wird gerade keine Simulation durchgeführt.}
{Der Benutzer drückt auf den Knopf 'Konfiguration erstellen'.}
{Es öffnet sich die erste Seite des Konfigurations-Assistenten.}

\item \testfall{Die erste oder zweiten Seite des Konfigurations-Assistenten ist geöffnet.}
{Der Benutzer tätigt alle Eingaben auf dieser Seite und bestätigt diese mit klicken auf 'weiter'.}
{Es kommt eine Fehlermeldung zu einer fehlerhaften Eingabe.}

\item \testfall{Die erste oder zweiten Seite des Konfigurations-Assistenten ist geöffnet.}
{Der Benutzer tätigt alle Eingaben auf dieser Seite korrekt und bestätigt diese mit klicken auf 'weiter'.}
{Es wird die nächste Seite des Konfigurations-Assistenten geöffnet.}

\item \testfall{Eine Seite des Konfigurations-Assistenten ist geöffnet.}
{Der Benutzer tätigt alle Eingaben auf allen Seiten. Der Knopf 'fertig stellen' wird aktiviert und der Benutzer bestätigt mit klicken auf 'fertig stellen'.}
{Es kommt eine Fehlermeldung zu einer fehlerhaften Eingabe.}

\item \testfall{Eine Seite des Assistenten ist geöffnet.}
{Der Benutzer tätigt alle Eingaben auf allen Seiten korrekt und bestätigt diese mit klicken auf 'fertig stellen'.}
{Das Fenster schließt sich und die erstellten Konfigurationen werden auf die Startseite geladen.}


\end{enumerate}

\textbf{/T002/} Konfigurationen speichern und laden
\begin{enumerate}
\item \testfall{Das Startfenster ist geöffnet und es wird gerade keine Simulation durchgeführt}
		{Der Benutzer klickt auf den Knopf 'Konfiguration laden' und wählt eine in der Liste genannten Konfigurationen. Er bestätigt mit 'OK'}
		{Es öffnet sich der Konfigurations-Assistent. Alle unter dem gewählten Namen gespeicherten Parameter sind bereits eingegeben.}
		
\item \testfall{Das Startfenster ist geöffnet und es wird gerade keine Simulation durchgeführt}
		{Der Benutzer wählt in dem Menü 'Optionen' das Untermenü 'Konfigurationen' und wählt eine der aufgelisteten Konfigurationen.}
		{Es öffnet sich der Konfigurations-Assistent. Alle unter dem gewählten Namen gespeicherten Parameter sind bereits eingegeben.}

\item \testfall{Eine Seite des Konfigurations-Assistenten zum erstellen einer Konfiguration ist geöffnet.}
		{Der Benutzer klickt auf den Knopf 'Konfiguration speichern'. In dem sich geöffneten Pop-Up-Fenster gibt er einen Namen für die Konfiguration ein und bestätigt mit 'OK'}
		{Das Fenster sowie der Assistent schließen sich.}
\end{enumerate}

\textbf{/T003/} Simulationen starten
\begin{enumerate}
\item \testfall{Das Startfenster ist geöffnet und es wird mindestens eine Simulation in der Liste 'Simulationen' angezeigt, die aktuell nicht durchgeführt wird.}
		{Der Benutzer klickt auf den Knopf 'start' neben einer Simulation.}
		{Der Knopf 'start' wird deaktiviert. Nach einigen Sekunden wird der Knopf 'start' und 'Ergebnisse anzeigen' aktiviert.}
		
\item \testfall{Das Startfenster ist geöffnet und es wird mindestens eine Simulation in der Liste 'Simulationen' angezeigt, die aktuell nicht durchgeführt wird.}
		{Der Benutzer klickt auf den Knopf 'Alle Simulationen starten'}
		{Alle 'start' Knöpfe und der Knopf 'Alle Simulationen starten' werden deaktiviert. Nach einigen Sekunden werden alle 'start' Knöpfe, alle 'Ergebnisse anzeigen' und der Knopf 'Alle Simulationen starten' aktiviert.}
\end{enumerate}

\textbf{/T004/} Ergebnisse anzeigen
\begin{enumerate}
\item \testfall{Das Startfenster ist geöffnet und es wird mindestens eine Simulation in der Liste 'Simulationen' angezeigt, die bereits durchgeführt wurde.}
		{Der Benutzer klickt auf 'Ergebnisse anzeigen'.}
		{Es öffnet sich das Ergebnisfenster der Simulation.}
\end{enumerate}

\textbf{/T005/} Ergebnisse speichern und laden
\begin{enumerate}
\item \testfall{Das Ergebnisfenster einer Simulation ist geöffnet.}
		{Der Benutzer scrollt bei den Ergebnissen ganz nach unten und drückt auf Ergebnisse speichern. Im geöffneten Pop-Up-Fenster gibt er einen Namen ein und bestätigt auf 'OK'.}
		{Das Pop-Up-Fenster schließt sich.}
\end{enumerate}

\item \testfall{Das Startfenster ist geöffnet.}
		{Der Benutzer wählt in dem Menü 'Optionen' das Untermenü 'Konfigurationen' und wählt eine der aufgelisteten Ergebnisse.}
		{Es öffnet sich das Ergebnisfenster des gespeicherten Ergebniszustands.}
\end{enumerate}


