\section{Einleitung}

Sswis ist ein forschungsorientiertes Softwareprodukt, mit dem wiederholte Spiele (``repeated games") als Teilgebiet der Spieltheorie näher untersucht werden können.

Dem Anwender wird dabei erlaubt, eine Vielzahl unterschiedlicher Konfigurationen zu bestimmen, anhand derer Simulationen auszuführen und die vom Programm ausgegebenen Ergebinsse zu begutachten bzw. (wissenschaftlich) weiterzuverarbeiten.

Konfigurationen bieten die Möglichkeit, Art und Größe des Spiels sowie die für die Simulation verfügbaren Strategien festzulegen. Es können aber auch abstrake Effekte, wie z.B. Kastenbildung, mittels vordefinierter oder während der Simulation entstandener Gruppen festgelegt werden.

Die Simulationen bauen immer auf der grundlegenden Annahme auf, dass die Akteure innerhalb der Spielabfolge eine möglichst gute relative Platzierung gegenüber ihrer Mitspieler erzielen möchten.
Dies stellt eine Modifikation der klassischen Spieltheorie dar, bei der Agenten ausschließlich von dem Ziel angetrieben sind, ihr eigenes Kapital zu maximieren.

Im Versuch, ihre Platzierung zu verbessern, können Agenten in regelmäßigen Abständen ihre Stategien anpassen.
Das angestrebte Softwareprodukt gibt an, ob im Zuge der Simulation ein Gleichgewicht eingetreten ist, d.h. ein Zustand, in dem Agenten ihre Stratgie nicht mehr\footnote{oder ggf. nur noch geringfügig} ändern.
Weitere Ergebnisse umfassen Ranglisten der Agenten und ihrer Stategien.
Um die Belastbarkeit der Daten zu erhöhen, werden Zufälligkeiten der Simulation erkannt und herausdestilliert.

Swiss unterliegt einer hohen Modularität, was Benutzern und Entwicklern das Hinzufügen neuer Funktionalität erleichtert.