\section{Einleitung}

Sswis ist ein forschungsorientiertes Softwareprodukt, mit dem wiederholte Spiele ("`repeated games"') als Teilgebiet der Spieltheorie näher untersucht werden können.

Dem Anwender wird dabei erlaubt, durch eine Vielzahl unterschiedlicher Eingabemöglichkeiten Rahmenbedingungen festzulegen, anhand ihrer Simulationen durchzuführen und die vom Programm ausgegebenen Ergebnisse zu begutachten bzw. (wissenschaftlich) weiterzuverarbeiten.

Benutzereingaben bieten u.a. die Möglichkeit, Art und Größe des Spiels sowie die für die Simulation verfügbaren Strategien festzulegen. Es können aber auch abstrakte Effekte, wie z.B. Kastenbildung, mittels vordefinierter oder entstehender Gruppen berücksichtigt werden.

Die Simulationen bauen immer auf der grundlegenden Annahme auf, dass die Agenten innerhalb der Spielabfolge eine möglichst gute relative Platzierung gegenüber ihren Mitspielern erzielen möchten.
Dies stellt eine Modifikation der klassischen Spieltheorie dar, bei der Agenten ausschließlich von dem Ziel angetrieben sind, ihr eigenes Kapital zu maximieren.

Im Versuch, ihre Platzierung zu verbessern, können Agenten in regelmäßigen Abständen ihre Strategien anpassen.
Das angestrebte Softwareprodukt gibt hierbei an, ob im Zuge der Simulation ein Gleichgewicht eingetreten ist, d.h. ein Zustand, in dem Agenten ihre Strategie nicht mehr\footnote{oder ggf. nur noch geringfügig} ändern.\\
Weitere Ergebnisse umfassen z.B. Ranglisten der Agenten und ihrer Strategien.

Sswis unterliegt einer hohen Modularität, was Benutzern und Entwicklern das Hinzufügen neuer Funktionalität erleichtert.
