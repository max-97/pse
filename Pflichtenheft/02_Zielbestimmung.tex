\section{Zielbestimmung}

\subsection{Pflichtkriterien}

\subsubsection{Simulation erstellen}
%TODO diese Punkte sind für funktionale Anforderungen
%Folgende Werte werden beim Erstellen einer Simulation vom Benutzer festgelegt:
%\begin{itemize}
%\item Anzahl der Agenten 
%\item Initiale Strategien für Agenten
%\item Das zu simulierende Spiel
%\item Rundenanzahl für das Spiel
%\item Maximale Rundenanzahl für die Simulation
%\item Wiederholungen der gesamten Simulation
%\item Auswertungsart der Agenten
%\item Wahrscheinlichkeit für Anpassung der Strategie eines Agenten
%\end{itemize}

\begin{itemize}
\item Erstellen von Konfigurationen für eine Simulation
\item Mehrere Konfigurationen gleichzeitig erstellen und simulieren
\item Agenten gruppieren
\end{itemize}


%\subsubsection{Ablauf der Simulation}
%\begin{itemize}
%\item Agenten werden gepaart
%\item Agenten spielen gegeneinander
%\item Agenten passen ihre Strategie an
%\item Simulation endet bei einem Gleichgewicht oder nach maximaler Rundenanzahl
%\end{itemize}

\subsubsection{Simulation auswerten}
\begin{itemize}
\item Agenten werden in einer Rangliste angezeigt
\item Verteilung der Strategien wird in einem Diagramm angezeigt
\item Punkteverteilung wird in einem Diagramm dargestellt
\end{itemize}

%\subsubsection{Sonstiges}
%\begin{itemize}
%
%\end{itemize}

\subsection{Wunschkriterien}
\begin{itemize}
\item Konfiguration einer Simulation speichern
\item Konfiguration für eine Simulation laden
\item Ergebnis einer Simulation speichern
\item Ergebnis einer Simulation laden
\item Agenten besitzen gemischte Strategien
\item Startkapital für Agenten
\item Strategien für die Paarung von Agenten
\item Zustand der Agenten während der Simulation speichern
\item Zustände der Agenten während der Simulation in Diagramm darstellen
\item Genetischer Algorithmus für Strategieanpassung
\end{itemize}

\subsection{Abgrenzungskriterien}
\begin{itemize}
\item Simulator ist ausschließlich auf Deutsch verfügbar
\item Simulator ist nur für wissenschaftlichen Gebrauch gedacht
\item Simulator liefert keine Beweise, sondern nur empirische Daten
%TODO gehört zu funktionalen Anforderungen
%\item Korrektheit der Software wird bei einer Simulation mit bis zu X Agenten gewährleistet
%\item Es werden nur Spiele mit den Aktionen \emph{Kooperation} und \emph{Defektion} simuliert 
\end{itemize}