\section{Zielbestimmung}

\subsection{Pflichtkriterien}

\subsubsection{Simulation erstellen}
Folgende Werte werden beim Erstellen einer Simulation vom Benutzer festgelegt:
\begin{itemize}
\item Anzahl der Agenten 
\item Initiale Strategien für Agenten
\item Das zu simulierende Spiel
\item Rundenanzahl für das Spiel
\item Maximale Rundenanzahl für die Simulation
\item Wiederholungen der gesamten Simulation
\item Auswertungsart der Agenten
\item Wahrscheinlichkeit für Anpassung der Strategie eines Agenten
\end{itemize}

\subsubsection{Ablauf der Simulation}
\begin{itemize}
\item Agenten werden gepaart
\item Agenten spielen gegeneinander
\item Agenten passen ihre Strategie an
\item Simulation endet bei einem Gleichgewicht oder nach maximaler Rundenanzahl
\end{itemize}

\subsubsection{Simulation auswerten}
\begin{itemize}
\item Agenten werden in einer Rangliste angezeigt
\item Verteilung der Strategien wird in einem Diagramm angezeigt
\item Punkteverteilung wird in einem Diagramm dargestellt
\item Ergebnis kann gespeichert werden
\end{itemize}

\subsubsection{Sonstiges}
\begin{itemize}
\item Konfiguration einer Simulation speichern
\item Konfiguration für eine Simulation laden
\item Ergebnis einer Simulation laden
\end{itemize}

\subsection{Wunschkriterien}
\begin{itemize}
\item Agenten besitzen gemischte Strategien
\item Startkapital für Agenten
\item Strategien für die Paarung von Agenten
\item Agenten besitzen Merkmale
\item Zustand der Agenten während der Simulation speichern
\item Zustände der Agenten während der Simulation in Diagramm darstellen
\item Genetischer Algorithmus für Strategieanpassung
\end{itemize}

\subsection{Abgrenzungskriterien}